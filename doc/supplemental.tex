\documentclass[final,1p]{elsarticle}
\usepackage{amssymb}
\usepackage{stackrel}
\usepackage{amsfonts}
\usepackage{amsmath}
\usepackage[english]{babel}
\usepackage{graphicx}
\usepackage{float}
\usepackage{rotating}
\usepackage{color}
%\usepackage{ulem}
\usepackage{tikz}
\usepackage{epic}
% \usepackage{caption,subcaption}
%\usepackage{subeqnarray}
\usepackage{titlesec}
\usepackage{etoolbox}

\makeatletter
\patchcmd{\ttlh@hang}{\parindent\z@}{\parindent\z@\leavevmode}{}{}
\patchcmd{\ttlh@hang}{\noindent}{}{}{}
\makeatother

\setcounter{secnumdepth}{4}

\titleformat{\paragraph}
{\normalfont\small\scshape}{\theparagraph}{1em}{}
%{\normalfont\normalsize\bfseries}{\theparagraph}{1em}{}
\titlespacing*{\paragraph}
{0pt}{3.25ex plus 1ex minus .2ex}{1.5ex plus .2ex}


\newcommand\encircle[1]{%
  \tikz[baseline=(X.base)] 
    \node (X) [draw, shape=circle, inner sep=0] {\strut #1};}

\newcommand{\I}{\mathrm{i}}
\newcommand{\D}{\mathrm{d}}

\newcommand{\PGR}[1]{{\color{blue} #1}}
\newcommand{\PGRcomm}[1]{{\color{blue}\small\em PGR2all: #1}}
\newcommand{\PGRfoot}[1]{{\color{blue}\footnote{\color{blue} #1}}}
\newcommand{\ES}[1]{{\color{red} #1}}
\newcommand{\EScomm}[1]{{\color{red}\small\em ES2all: #1}}
\newcommand{\ESfoot}[1]{{\color{red}\footnote{\color{red} #1}}}
\newcommand{\bmu}{{\mbox{\boldmath{$\mu$}}}}

\newcounter{denselistcounter}
\newenvironment{denselist}[1]
{ \begin{list}
  {#1{denselistcounter})}{\usecounter{denselistcounter}
  \setlength{\topsep}{-0pt}
  \setlength{\partopsep}{-0pt}
  \setlength{\itemsep}{-0pt}
  \setlength{\parsep}{-0pt}
  \setlength{\labelwidth}{6pt}
  \setlength{\labelsep}{4pt}
  \setlength{\leftmargin}{20pt}
  \setlength{\rightmargin}{20pt}
  }
}
{\end{list}}

\unitlength 1mm
\thicklines

\begin{document}


\begin{frontmatter}

\title{Quantum Dissipative Dynamics\\
A real time real space approach to far off equilibrium dynamics in finite electron systems}

\author{F.~Coppens$^{a,b}$}
\author{P. M. Dinh$^{a,b}$}
\author{J. Heraud$^{a,b}$}
\author{P.-G.~Reinhard$^c$}
\author{E.~Suraud$^{a,b,d}$}
\author{M.~Vincendon$^{a,b}$}
\cortext[author]{Corresponding author:
  coppens@irsamc.ups-tlse.fr} 
\address{$^a$Universit\'e de Toulouse; UPS; Laboratoire de Physique
             Th\'{e}orique, IRSAMC; F-31062 Toulouse Cedex, France}
\address{$^b$Laboratoire de Physique Th\'eorique, Universit\'e Paul
  Sabatier, CNRS, F-31062 Toulouse C\'edex, France}
\address{$^c$Institut f{\"u}r Theoretische Physik, Universit{\"a}t
  Erlangen, D-91058 Erlangen, Germany}
\address{$^d$School of Mathematics and Physics, Queen's University Belfast, Belfast, UK}

\date{Status: 12. January 2019}
\begin{abstract}
\PGR{\em This is a first draft for the CPC presenting the TDLDA+RTA code
  to the public. The layout of presentation has yet to be discussed.
Presently it mixes theory and algorithm with code. We may also
consider collecting all theory \& numerics together then followed by a
huge block detailing the code.}
\end{abstract}

\begin{keyword}
%\PACS{05.30.Fk,31.70.Hq,34.10.+x,36.40.Cg}
electronic dissipation, time-dependent
density functional theory, ...
\end{keyword}
\end{frontmatter}

\tableofcontents

\noindent
\PGRcomm{Two general points yet to be clarified:
\\
The code is still capable of coupling to a dielectric
  environment. We should skip that for the CPC publication.
\\
Another question to be discussed is whether we maintain the option to
run finite differences instead of FFT. If we do so, then we have to
test that branch carefully.
}


\subsection{The structure of the TDLDA  and RTA packages}
\label{sec:TDLDA_RTAnum}

%\subsection{Storage of wavefunctions, densities and potentials}

Naturally, wavefunctions and other fields in spatial 3D representation
are would be handled as three-dimensional arrays, e.g. {\tt
  chpcoul3D(1:kxbox,1:kybox,1:kzbox)} for the Coulomb potential. However, in
order to simplify loops, all 3D fields are addressed as linear
arrays, e.g., for the Coulomb field {\tt chpcoul(1:kxbox*kybox*kzbox)}
where {\tt kxbox, kybox, kzbox} are the number of grid points in
$x$, $y$, and $z$ direction. The parameters defining and handling the grid 
are:
\begin{center}
\begin{tabular}{lcl}
{\tt kxbox} &:& total number of grid points in $x$-direction
\\
{\tt kybox} &:& total number of grid points in $y$-direction
\\
{\tt kzbox} &:& total number of grid points in $z$-direction
\\
{\tt nx2}   &:& equivalent to {\tt kxbox}
\\
{\tt ny2}   &:& equivalent to {\tt kybox}
\\
{\tt nz2}   &:& equivalent to {\tt kzbox}
\\
{\tt nx} &:& ={\tt nx2/2}, number of grid points with $x>0$, see eq. (\ref{eq:gridpoints}) 
\\
{\tt ny} &:& ={\tt ny2/2}, number of grid points with $y>0$, see eq. (\ref{eq:gridpoints}) 
\\
{\tt nz} &:& ={\tt nz2/2}, number of grid points with $z>0$, see eq. (\ref{eq:gridpoints}) 
\\
{\tt kdfull2} &:& ={\tt kxbox*kybox*kzbox}, total number of grid points
\\
{\tt nxyz} &:& equivalent to {\tt kdfull2}
\\
{\tt ix} or {\tt i1} &:& running index for $x$ values
\\
{\tt iy} or {\tt i2} &:& running index for $y$ values
\\
{\tt iz} or {\tt i3} &:& running index for $z$ values
\\
{\tt ind} &:& ={\tt ix}+{\tt nx2}*({\tt iy}-1)+{\tt nx2}*{\tt ny2}*({\tt iz}-1),
               running index for linear storage
%\\
%{\tt } &:&
\end{tabular}
\end{center}
The basic arrays concern densities, potentials, and
s.p. wavefunctions.  Densities and potentials distinguish spin which
is denoted by $\uparrow=$ spin-up and $\downarrow=$ spin-down.
Each s.p. wavefunction is associated with one unique spin
carried in the array {\tt ispin(1:kstate)}.
The fields are arranged the following way:
%\begin{center}
%\begin{tabular}{lcl}
\begin{description}
\item{\tt rho(2*kdfull2)} = density,
linearly stored in two blocks of length {\tt kdfull2}:
\\
 {\tt rho(1:kdfull2)} = total density 
  $\rho(\mathbf{r})=\rho_\uparrow(\mathbf{r})+\rho_\downarrow(\mathbf{r})$
\\
 {\tt rho(kdfull2+1:2*kdfull2)} = difference
  $\rho(\mathbf{r})=\rho_\uparrow(\mathbf{r})-\rho_\downarrow(\mathbf{r})$

\item{\tt aloc(2*kdfull2)} local Kohn-Sham potential,
linearly stored in two blocks of length {\tt kdfull2}:
\\
 {\tt aloc(1:kdfull2)} = local KS potential for spin-up
  $=U_\uparrow(\mathbf{r})$
\\
 {\tt aloc(kdfull2+1:2*kdfull2)} = local KS potential for spin-down
  $=U_\downarrow(\mathbf{r})$

\item{\tt chpcoul(kdfull2)} = Coulomb potential, linearly stored

\item{\tt psi(kdfull2,kstate)} = set of complex s.p. wavefunctions 
\\ 1. index for spatial disribution, 2. index counts the states
\\ complex array {\tt psi} for dynamics, real array {\tt psir} for static case
\\ each s.p. state has unique spin given in the array {\tt ispsin(1:kstate)}
\end{description}
%\end{tabular}
%\end{center}


\EScomm{TDLDA and RTA calling trees to be simplified with keywords only; 
Present figs to be put in supplemental material together with details on subroutines. 
NB Supplemental material to be made as separate file. }

\subsubsection{The TDLDA calling tree}
\label{sec:TDLDAtree}

The TDLDA packages is a rather complex collection of routines.  Thus
the tree structure of the code is sketched only at the major level of
callings and is presented in three separate diagrams: the main routine
with all initializations and two calls to the major drivers for static
and dynamic calculations in diagram \ref{fig:tree_main}, the static
driver in diagram \ref{fig:tree_static}, and the dynamic driver in
\ref{fig:tree_dyn}.
\begin{figure}
\centerline{\fbox{
\begin{picture}(122,158)(5,-146)
\put(10,8){\mbox{\tt main}}
\put(11,7){\line(0,-1){147}}
 \put(50,1){\mbox{\large general initializations}}
   \put(20,-4){\mbox{\tt cpu\_time,  init\_parallele, init\_simann}}
   \put(11,-3){\line(1,0){8}}
   \put(20,-8){\mbox{\tt  initnamelists, checkoptions}}
   \put(11,-7){\line(1,0){8}}
   \put(20,-12){\mbox{\tt init\_baseparams, iparams}}
   \put(11,-11){\line(1,0){8}}
   \put(20,-16){\mbox{\tt iperio, changeperio}}
        \put(90,-16){\mbox{$\leftrightarrow$ PsP tables}}
   \put(11,-15){\line(1,0){8}}
   \put(20,-20){\mbox{\tt init\_grid, init\_fields}}
   \put(11,-19){\line(1,0){8}}
   \put(20,-24){\mbox{\tt init\_radmatrix}}
        \put(90,-24){\mbox{$\leftrightarrow$ tables for SIC}}
   \put(11,-23){\line(1,0){8}}
   \put(20,-28){\mbox{\tt ocoption, init\_output}}
   \put(11,-27){\line(1,0){8}}
   \put(20,-32){\mbox{\tt init\_jellium, initions}}
   \put(11,-31){\line(1,0){8}}
   \put(20,-36){\mbox{\tt pseudo\_external}}
        \put(90,-36){\mbox{$\leftrightarrow$ read explicit PsP}}
   \put(11,-35){\line(1,0){8}}
   \put(20,-40){\mbox{\tt initwf}}
   \put(11,-39){\line(1,0){8}}
   \put(20,-44){\mbox{\tt init\_homfield}}
   \put(11,-43){\line(1,0){8}}
   \put(20,-48){\mbox{\tt timer, init\_boxpara}}
   \put(11,-47){\line(1,0){8}}
 \put(50,-55){\mbox{\large static part}}
   \put(20,-60){\mbox{\tt init\_fsicr}}
   \put(11,-59){\line(1,0){8}}
   \put(20,-64){\fbox{\tt statit}}
   \put(11,-63){\line(1,0){8}}
   \put(20,-68){\mbox{\tt simann}}
   \put(11,-67){\line(1,0){8}}
   \put(20,-72){\mbox{\tt afterburn}}
   \put(11,-72){\line(1,0){8}}
 \put(50,-79){\mbox{\large dynamic part}}
   \put(20,-84){\mbox{\tt init\_absbc, init\_abs\_accum, initmeasurepoints}}
   \put(11,-83){\line(1,0){8}}
   \put(20,-88){\mbox{\tt init\_dynprotocol, evaluate}}
   \put(11,-87){\line(1,0){8}}
   \put(20,-92){\mbox{\tt init\_fsic, init\_scattel{\rm(??)}}}
   \put(11,-91){\line(1,0){8}}
   \put(20,-96){\mbox{\tt init\_dynwf}}
   \put(11,-95){\line(1,0){8}}
   \put(20,-100){\mbox{\tt restart2, addcluster}}
   \put(11,-99){\line(1,0){8}}
   \put(20,-104){\mbox{\tt calclocal, calc\_sic, calcpseudo}}
   \put(11,-103){\line(1,0){8}}
   \put(20,-108){\mbox{\tt dyn\_mfield, fermi\_init, info}}
   \put(11,-107){\line(1,0){8}}
   \put(20,-112){\mbox{\tt mergetabs, ordo\_per\_spin}}
   \put(11,-111){\line(1,0){8}}
   \put(20,-116){\mbox{\tt calc\_proj, calc\_projFine}}
   \put(11,-115){\line(1,0){8}}
   \put(20,-120){\mbox{\tt mpi\_barrier}}
   \put(11,-119){\line(1,0){8}}
   \put(20,-124){\mbox{\tt rhointxy, rhointxz, rhointyz}}
   \put(11,-123){\line(1,0){8}}
   \put(20,-128){\mbox{\tt lffirststep}}
   \put(11,-127){\line(1,0){8}}
   \put(20,-132){\mbox{\tt open\_protok\_el, analyze\_elect}}
   \put(11,-131){\line(1,0){8}}
   \put(20,-136){\fbox{\tt dyn\_propag}}
   \put(11,-135){\line(1,0){8}}
   \put(20,-141){\mbox{\tt mpi\_finalize}}
   \put(11,-140){\line(1,0){8}}
\end{picture}
}}
\caption{\label{fig:tree_main}
Schematic calling tree for the main routine in {\tt main.F90}.
The calling trees for the two major 
subroutines in framed boxes are explained in
subsequent figures \ref{fig:tree_static} and  \ref{fig:tree_dyn}.
}
\end{figure}


\begin{figure}
\centerline{\fbox{
\begin{picture}(122,66)(5,-62)
\put(10,0){\mbox{\tt static}}
\put(11,-1){\line(0,-1){58}}
   \put(20,-4){\mbox{\tt calcrhor}}
   \put(11,-3){\line(1,0){8}}
   \put(20,-8){\mbox{\tt falr, solv\_possion}}
   \put(11,-7){\line(1,0){8}}
   \put(20,-12){\mbox{\tt calcpseudo, static\_mfield}}
   \put(11,-11){\line(1,0){8}}
   \put(20,-16){\mbox{\tt pricm, infor, prifld}}
   \put(11,-15){\line(1,0){8}}
   \put(20,-20){\mbox{\tt sstep}}
   \put(11,-19){\line(1,0){8}}
   \put(21,-21){\line(0,-1){22}}
      \put(30,-24){\mbox{\tt cpu\_time, system\_clock}}
      \put(21,-23){\line(1,0){8}}
      \put(30,-28){\mbox{\tt exchgr, subtr\_sicpot}}
      \put(21,-27){\line(1,0){8}}
      \put(30,-32){\mbox{\tt nonlocalr}}
      \put(21,-31){\line(1,0){8}}
      \put(30,-36){\mbox{\tt rftf, rfftback, rkin3D}}
      \put(21,-35){\line(1,0){8}}
      \put(30,-40){\mbox{\tt project, givens}}
      \put(21,-39){\line(1,0){8}}
      \put(30,-44){\mbox{\tt schmidt, reocc}}
      \put(21,-43){\line(1,0){8}}
   \put(20,-48){\mbox{\tt prifldz, pri\_pstat, printfield}}
   \put(11,-47){\line(1,0){8}}
   \put(20,-52){\mbox{\tt localizer, mtv\_field}}
   \put(11,-51){\line(1,0){8}}
   \put(20,-56){\mbox{\tt resume, rsave}}
   \put(11,-55){\line(1,0){8}}
   \put(20,-60){\mbox{\tt infor\_sic, diag\_lagr}}
   \put(11,-59){\line(1,0){8}}
\end{picture}
}}
\caption{\label{fig:tree_static}
Schematic calling tree for the static driver routine in {\tt static.F90}.
}
\end{figure}


\begin{figure}
\centerline{\fbox{
\begin{picture}(122,70)(5,-66)
\put(10,0){\mbox{\tt dyn\_propag}}
\put(11,-1){\line(0,-1){62}}
   \put(20,-4){\mbox{\tt stimer, print\_densdiff, savings}}
   \put(11,-3){\line(1,0){8}}
   \put(20,-8){\fbox{\tt rta}}
   \put(11,-7){\line(1,0){8}}
        \put(90,-8){\mbox{$\leftrightarrow$ see section \ref{sec:RTA}}}
   \put(20,-12){\mbox{\tt init\_occ\_target, init\_psitarget}}
   \put(11,-11){\line(1,0){8}}
   \put(20,-16){\mbox{\tt tstep\_exp, CrankNicholson\_exp}}
   \put(11,-15){\line(1,0){8}}
   \put(20,-20){\mbox{\tt tstep}}
   \put(11,-19){\line(1,0){8}}
   \put(21,-21){\line(0,-1){18}}
      \put(30,-24){\mbox{\tt cpu\_time, system\_clock}}
      \put(21,-23){\line(1,0){8}}
      \put(30,-28){\mbox{\tt nonlocstep}}
      \put(21,-27){\line(1,0){8}}
      \put(30,-32){\mbox{\tt kinprop, fftf, fftback}}
      \put(21,-31){\line(1,0){8}}
      \put(30,-36){\mbox{\tt eval\_unitrot}}
      \put(21,-35){\line(1,0){8}}
      \put(30,-40){\mbox{\tt dyn\_meanfield, escmask, checkzeroforce}}
      \put(21,-39){\line(1,0){8}}
   \put(20,-44){\mbox{\tt rhointxy, rhointxz, rhointyz, testcurrent}}
   \put(11,-43){\line(1,0){8}}
   \put(20,-48){\mbox{\tt itstep, itstepv, enerkin\_ions, reset\_ions}}
   \put(11,-47){\line(1,0){8}}
   \put(20,-52){\mbox{\tt calc\_pseudo, calclocal, calc\_sic}}
   \put(11,-51){\line(1,0){8}}
   \put(20,-56){\mbox{\tt calc\_proj, calc\_projFine}}
   \put(11,-55){\line(1,0){8}}
   \put(20,-60){\mbox{\tt analyze\_elect, analyze\_ions, energ\_ions}}
   \put(11,-59){\line(1,0){8}}
   \put(20,-64){\mbox{\tt mpi\_barrier}}
   \put(11,-63){\line(1,0){8}}
\end{picture}
}}
\caption{\label{fig:tree_dyn}
Schematic calling tree for the static driver routine in {\tt dynamic.F90}.
The routine in the framed box is explained in great detail in section
\ref{sec:RTA}. 
}
\end{figure}

\subsubsection{The structure of the RTA package} % in {\tt rta.F90}}
\label{eq:RTApack}

\paragraph{Input and output related to RTA}
\label{sec:IO}

\PGRcomm{This subsection needs yet to be worked out in detail.}
\EScomm{Section to be put in supplemental material}


\medskip

The {\tt NAMELIST dynamic} contains the following variables
used in the RTA procedure:\PGRfoot{We should define a new {\tt
    NAMELIST} and shift the RTA variables to there.}
\begin{description}
\item[\tt jrtaint:]
   Modulus for calling the RTA subroutine, i.e., nr. of TDLDA steps
   per one RTA step. Course time step $\Delta t$ for RTA and fine
   time step for TDLDA {\tt dt1} are related as
   $\Delta t=${\tt jrtaint}$*${\tt dt1}.
\item[\tt rtamu:]
   Parameter $\mu$ in front of the quadratic density constraint in the
   DCMF Hamiltonian (\ref{eq:hDCMF}).
\item[\tt rtamuj:]
   Parameter $\mu_j$ in front of the quadratic current constraint in the
   DCMF Hamiltonian (\ref{eq:hDCMF}).
\item[\tt rtasumvar2max:] Termination criterion $\epsilon_0$ in the
   RTA step as used
   in figure \ref{fig:summaryDCMF}. 
\item[\tt rtaeps:]
  Step size $\delta$ in the damping operator \PGRcomm{(cross ref to be
    defined)} $\mathcal{D}$ for the RTA step.
\item[\tt rtae0dmp:]
  Energy offset $E_00$ in the damping operator \PGRcomm{(cross ref to be
    defined)} $\mathcal{D}$ for the RTA step.
\item[\tt rtasigee:]
  In medium $e^-$-$e^-$ cross section used for the relaxation time 
  (\ref{eq:relaxtime}).
\item[\tt rtars:]
  Effective  Wigner-Seitz radius $r_s$ used for the relaxation time 
  (\ref{eq:relaxtime}).
\item[\tt rtatempinit:]
  The value {\tt rtatempinit}/10 is used as lower value for the search
  of temperature in {\tt SUBROUTINE ferm1}.
\item[\tt rtaforcetemperature:]
  \PGRcomm{Seems to be obsolete?}
\end{description}

Most observables were already defined at TDLDA stage and thus
are returned in the standard output files as explained in the TDLDA
section\PGRfoot{Cross reference yet to be defined.}. New output files
specific to observables from RTA are:\PGRfoot{These
  files seem to carry only protocol for numerics. Where do we
  print energy balance ($E_\mathrm{intr}^*$ etc)?}
\begin{description}
\item[\tt prta:]
\item[\tt peqstat:]
\item[\tt pspeed:]
\item[\tt prhov:]
\end{description}

\paragraph{The calling tree}
\label{eq:treeRTA}

Here is an oversight over the tree structure of the RTA routines.
Those subroutines contained in  {\tt rta.F90} are explained in detail
in section \ref{eq:details}. Subroutines coming from the TDLDA package
or external sources are marked\PGRfoot{The {\tt HEigensystem} seems
  copied from some library. This could cause copyright problems if we
  publish the code. Is it from BLAS/LINPACK? Then we could
replace the Fortran source by a library call.}
\\
\centerline{\fbox{
\begin{picture}(122,159)(5,-155)
\put(10,0){\mbox{\tt RTA}}
\put(11,-1){\line(0,-1){150}}
   \put(20,-4){\mbox{\tt srhomat}}
   \put(11,-3){\line(1,0){8}}
   \put(21,-5){\line(0,-1){6}}
      \put(30,-8){\mbox{\tt scalar}}
      \put(21,-7){\line(1,0){8}}
      \put(30,-12){\mbox{\tt cdiagmat}}
      \put(21,-11){\line(1,0){8}}
      \put(31,-13){\line(0,-1){2}}
         \put(40,-16){\mbox{\tt HEigensystem}}
         \put(31,-15){\line(1,0){8}}
             \put(90,-16){\mbox{$\leftrightarrow$ library routine}}
   \put(20,-20){\mbox{\tt eqstate}}
   \put(21,-21){\line(0,-1){74}}
   \put(11,-19){\line(1,0){8}}
      \put(30,-24){\mbox{\tt calcrhotot}}
      \put(21,-23){\line(1,0){8}}
      \put(30,-28){\mbox{\tt calc\_current}}
      \put(21,-27){\line(1,0){8}}
      \put(30,-32){\mbox{\tt calc\_Eref}}
      \put(21,-31){\line(1,0){8}}
      \put(30,-36){\mbox{\tt fermi1}}
      \put(31,-37){\line(0,-1){2}}
      \put(21,-35){\line(1,0){8}}
         \put(40,-40){\mbox{\tt occT1}}
         \put(31,-39){\line(1,0){8}}
      \put(30,-44){\mbox{\tt calc\_psi1}}
      \put(31,-45){\line(0,-1){38}}
      \put(21,-43){\line(1,0){8}}
         \put(40,-48){\mbox{\tt calc\_hamiltonien}}
         \put(31,-47){\line(1,0){8}}
         \put(40,-52){\mbox{\tt calc\_var}}
         \put(41,-53){\line(0,-1){2}}
            \put(50,-56){\mbox{\tt cproject}}
            \put(41,-55){\line(1,0){8}}
            \put(90,-56){\mbox{\PGR{$\leftrightarrow$ \footnotesize obsolete?}}}
         \put(40,-60){\mbox{\tt calcrhotot}}
         \put(31,-59){\line(1,0){8}}
         \put(40,-64){\mbox{\tt calc\_current}}
         \put(31,-63){\line(1,0){8}}
         \put(40,-68){\mbox{\tt cschmidt}}
         \put(31,-67){\line(1,0){8}}
             \put(90,-68){\mbox{$\leftrightarrow$ TDLDA package}}
         \put(40,-72){\mbox{\tt calc\_ekin}}
         \put(31,-71){\line(1,0){8}}
             \put(90,-72){\mbox{$\leftrightarrow$ TDLDA package}}
         \put(40,-76){\mbox{\tt nonlocalc}}
         \put(31,-75){\line(1,0){8}}
             \put(90,-76){\mbox{$\leftrightarrow$ TDLDA package}}
         \put(40,-80){\mbox{\tt fftf}}
         \put(31,-79){\line(1,0){8}}
             \put(90,-80){\mbox{$\leftrightarrow$ FFTW3 package}}
         \put(40,-84){\mbox{\tt fftback}}
         \put(31,-83){\line(1,0){8}}
             \put(90,-84){\mbox{$\leftrightarrow$ FFTW3 package}}
      \put(30,-88){\mbox{\tt coul\_mfield}}
      \put(21,-87){\line(1,0){8}}
             \put(90,-88){\mbox{$\leftrightarrow$ TDLDA package}}
      \put(30,-92){\mbox{\tt dyn\_mfield}}
      \put(21,-91){\line(1,0){8}}
             \put(90,-92){\mbox{$\leftrightarrow$ TDLDA package}}
      \put(30,-96){\mbox{\tt info}}
      \put(21,-95){\line(1,0){8}}
             \put(90,-96){\mbox{$\leftrightarrow$ TDLDA package}}
   \put(20,-100){\mbox{\tt occupT0}}
   \put(21,-101){\line(0,-1){2}}
   \put(11,-99){\line(1,0){8}}
      \put(30,-104){\mbox{\tt indexx}}
      \put(21,-103){\line(1,0){8}}
   \put(20,-108){\mbox{\tt calcrhoeq}}
   \put(21,-109){\line(0,-1){10}}
   \put(11,-107){\line(1,0){8}}
      \put(30,-112){\mbox{\tt cdiagmat}}
      \put(31,-113){\line(0,-1){2}}
      \put(21,-111){\line(1,0){8}}
         \put(40,-116){\mbox{\tt HEigensystem}}
         \put(31,-115){\line(1,0){8}}
             \put(90,-116){\mbox{$\leftrightarrow$ library routine}}
      \put(30,-120){\mbox{\tt  indexx}}
      \put(21,-119){\line(1,0){8}}
   \put(20,-124){\mbox{\tt calc\_Eref}}
   \put(11,-123){\line(1,0){8}}
   \put(20,-128){\mbox{\tt CorrectEnergy2}}
   \put(11,-127){\line(1,0){8}}
   \put(20,-132){\mbox{\tt OccupPerSpin}}
   \put(11,-131){\line(1,0){8}}
   \put(20,-136){\mbox{\tt temperature}}
   \put(11,-135){\line(1,0){8}}
   \put(21,-137){\line(0,-1){2}}
      \put(30,-140){\mbox{\tt lmdif1}}
      \put(21,-139){\line(1,0){8}}
   \put(20,-144){\mbox{\tt dyn\_mfield}}
   \put(11,-143){\line(1,0){8}}
             \put(90,-144){\mbox{$\leftrightarrow$ TDLDA package}}
   \put(20,-148){\mbox{\tt info}}
   \put(11,-147){\line(1,0){8}}
             \put(90,-148){\mbox{$\leftrightarrow$ TDLDA package}}
   \put(20,-152){\mbox{\tt analyze\_elect}}
   \put(11,-151){\line(1,0){8}}
             \put(90,-152){\mbox{$\leftrightarrow$ TDLDA package}}
\end{picture}
}}


\section{Supplemental material}

\subsection{Storage of wavefunctions, densities and potentials}

Naturally, wavefunctions and other fields in spatial 3D representation
are would be handled as three-dimensional arrays, e.g. {\tt
  chpcoul3D(1:kxbox,1:kybox,1:kzbox)} for the Coulomb potential. However, in
order to simplify loops, all 3D fields are addressed as linear
arrays, e.g., for the Coulomb field {\tt chpcoul(1:kxbox*kybox*kzbox)}
where {\tt kxbox, kybox, kzbox} are the number of grid points in
$x$, $y$, and $z$ direction. The parameters defining and handling the grid 
are:
\begin{center}
\begin{tabular}{lcl}
{\tt kxbox} &:& total number of grid points in $x$-direction
\\
{\tt kybox} &:& total number of grid points in $y$-direction
\\
{\tt kzbox} &:& total number of grid points in $z$-direction
\\
{\tt nx2}   &:& equivalent to {\tt kxbox}
\\
{\tt ny2}   &:& equivalent to {\tt kybox}
\\
{\tt nz2}   &:& equivalent to {\tt kzbox}
\\
{\tt nx} &:& ={\tt nx2/2}, number of grid points with $x>0$, see eq. (\ref{eq:gridpoints}) 
\\
{\tt ny} &:& ={\tt ny2/2}, number of grid points with $y>0$, see eq. (\ref{eq:gridpoints}) 
\\
{\tt nz} &:& ={\tt nz2/2}, number of grid points with $z>0$, see eq. (\ref{eq:gridpoints}) 
\\
{\tt kdfull2} &:& ={\tt kxbox*kybox*kzbox}, total number of grid points
\\
{\tt nxyz} &:& equivalent to {\tt kdfull2}
\\
{\tt ix} or {\tt i1} &:& running index for $x$ values
\\
{\tt iy} or {\tt i2} &:& running index for $y$ values
\\
{\tt iz} or {\tt i3} &:& running index for $z$ values
\\
{\tt ind} &:& ={\tt ix}+{\tt nx2}*({\tt iy}-1)+{\tt nx2}*{\tt ny2}*({\tt iz}-1),
               running index for linear storage
%\\
%{\tt } &:&
\end{tabular}
\end{center}
The basic arrays concern densities, potentials, and
s.p. wavefunctions.  Densities and potentials distinguish spin which
is denoted by $\uparrow=$ spin-up and $\downarrow=$ spin-down.
Each s.p. wavefunction is associated with one unique spin
carried in the array {\tt ispin(1:kstate)}.
The fields are arranged the following way:
%\begin{center}
%\begin{tabular}{lcl}
\begin{description}
\item{\tt rho(2*kdfull2)} = density,
linearly stored in two blocks of length {\tt kdfull2}:
\\
 {\tt rho(1:kdfull2)} = total density 
  $\rho(\mathbf{r})=\rho_\uparrow(\mathbf{r})+\rho_\downarrow(\mathbf{r})$
\\
 {\tt rho(kdfull2+1:2*kdfull2)} = difference
  $\rho(\mathbf{r})=\rho_\uparrow(\mathbf{r})-\rho_\downarrow(\mathbf{r})$

\item{\tt aloc(2*kdfull2)} local Kohn-Sham potential,
linearly stored in two blocks of length {\tt kdfull2}:
\\
 {\tt aloc(1:kdfull2)} = local KS potential for spin-up
  $=U_\uparrow(\mathbf{r})$
\\
 {\tt aloc(kdfull2+1:2*kdfull2)} = local KS potential for spin-down
  $=U_\downarrow(\mathbf{r})$

\item{\tt chpcoul(kdfull2)} = Coulomb potential, linearly stored

\item{\tt psi(kdfull2,kstate)} = set of complex s.p. wavefunctions 
\\ 1. index for spatial disribution, 2. index counts the states
\\ complex array {\tt psi} for dynamics, real array {\tt psir} for static case
\\ each s.p. state has unique spin given in the array {\tt ispsin(1:kstate)}
\end{description}
%\end{tabular}
%\end{center}



\subsection{The TDLDA subroutines in detail}
\label{eq:detailsTDLDA}


\PGRcomm{Here is to come a detailed description of the subroutines
  similar as for RTA in \ref{eq:details}, but here for the LDA part.
Only one example is given as appetizer. The problem is that it cost
enormous amounts of space to present all routines and functions at that
level of detail. We should shift that to the
supplementary material. It will comes there anyway if we decide to
transfer all these details to {\tt doxygen}.
}

\subsubsection*{\tt SUBROUTINE coul\_mfield(rho)}
\begin{tabular}{lcl}
 {\tt rho(1:2*kdfull2)} & in/out & \\ density for which Coulomb field
 is computed
\end{tabular}
\\[4pt]
Computes Coulomb potential for given density by invoking
Poisson solver.
Th emerging coulomb potential is communicated as {\tt chpcoul}
via module {\tt params}.
In case of dielectric external media, adds pseudo-density for image
charge.
\\
\PGRcomm{Why is {\tt rho} also {\tt INTENT OUT}? Is all {\tt
    1:2*kdfull2} used or only the first block {\tt 1:kdfull2}?}



%\subsubsection*{\tt }
%\begin{tabular}{lcl}
% {\tt } & &\\
%\end{tabular}
%\\[4pt]


\subsection{The RTA subroutines in detail}
\label{eq:details}

\subsubsection*{\tt SUBROUTINE rta(psi,aloc,rho,iterat)}
\begin{tabular}{lcl}
 {\tt iterat} & in & external iteration number (TDLDA time step)\\
 {\tt psi(1:kdfull2,1:kstate)} & in/out& set of s.p. wavefunctions\\
 {\tt rho(1:2*kdfull2)}& in/out & local densities for spin up and down\\
 {\tt aloc(1:2*kdfull2)} & in/out& local potentials for spin up and down\\
\end{tabular}
\\[4pt]
Basic RTA routine performing density constrained mean-field (DCMF)
iterations, energy adjustment, admixing of local equilibrium states by
calls to subroutines (see calling tree).



\subsubsection*{\tt SUBROUTINE calcrhoeq(psiorthloc,psieqloc,psiloc,occuporthloc,occuploc,nstateloc)}
\begin{tabular}{lcl}
 {\tt nstateloc} & in & number of s.p. states in spin block\\
 {\tt psiorthloc(1:kdfull2,1:nstateloc),} & in & set of TDLDA
 wavefunctions (natural orbitals)\\
 {\tt occuporthloc(1:nstateloc)} & in & occupations of TDLDA states\\
 {\tt psieqloc(1:kdfull2,1:nstateloc)} & in& set of local-equilibrium wavefunctions\\
 {\tt psiloc(1:kdfull2,1:nstateloc)} & out& set of final mixed wavefunctions\\
 {\tt occuploc(1:nstateloc)} & in/out & \\
\end{tabular}
\\[4pt]
Encapsulated in {\tt SUBROUTINE rta}. Performs the mixing of TDLDA
states with local-equilibrium state according to relaxation rate
for one spin block. The mixed densty matrix is expanded in a
representation by both sets of s.p. states. 



\subsubsection*{\tt SUBROUTINE calc\_Eref(occup,ispin,Ei,Eref)}
\begin{tabular}{lcl}
 {\tt occup(1:nstate)} & in & occupation number for s.p. states.\\
 {\tt ispin(1:nstate)} & in & spin assignement for s.p. states.\\
 {\tt Ei(1:nstate)} & in & spin assignement for s.p. states.\\
 {\tt Eref(1:2)} & out & sum of s.p. energies per spin.\\
\end{tabular}
\\
Computes the weighted sum of s.p. energies as reference energy
for DCMF. The sum is accumulated for each spin separately.


\subsubsection*{\tt SUBROUTINE fermi1(ekmod,eref,occup,ispinact,T0i,T1i,T2,mu)}
\begin{tabular}{lcl}
 {\tt ekmod(1:kstate)} & in & given s.p. energies, spin up block first, then
 spin down\\
 {\tt eref}& in & reference energy = wanted sum of s.p. energies\\
 {\tt ispinact}& in & spin for which routine is run\\
 {\tt T0i, T1i}& in & lower and upper temperature for search\\
 {\tt occup(1:kstate)}& in/out & occupation numbers, spin block-wise\\
 {\tt T2} & out & final temperature for which Fermi distribution
 matches {\tt eref} \\
 {\tt mu} & out & final chemical potential\\
\end{tabular}
\\[4pt]
Determines thermal Fermi occupation such that given sum of
s.p. energies {\tt eref} and particle number is matched. Is done for
each spin separately. Solution scheme is bracketing. Refers to 
{\tt SUBROUTINE OccT1} while iterating temperatur {\tt T2}.
\\
\PGRcomm{Nr. of spin-up/spin-down states comes through {\tt m\_params}. 
We should protocol all such entries. First step is to augment each
{\tt USE} by {\tt ONLY} such that the explicitely communicated
variables becomes visible. Important variables may then be listed
explicitely.
}
\\
\PGRcomm{Routine requires that arrays are sorted in continuous blocks of
  spin. Do we have an initial check for that? And we need to address
  that in the general part which explains the layout of arrays.}

\subsubsection*{\tt SUBROUTINE OccT1(occrefloc,enerloc,Etotloc,muloc,occtotloc,n,T,occuploc)}
\begin{tabular}{lcl}
 {\tt enerloc(1:n)} & in & s.p. energies for actual spin\\
 {\tt n} & in & number of s.p. states treated here\\
 {\tt T} & in & temperature\\
 {\tt occrefloc} & in & wanted total number of particles \\
 {\tt occuploc(1:n)} & out & thermal occupation numbers for given {\tt
   T} and s.p. energies\\
 {\tt muloc} & out & chemical potential (Fermi energy)\\
 {\tt occtotloc} & out& final total number of particles \\
 {\tt Etotloc} & out& sum of s.p. energies \\
\end{tabular}
\\[4pt]
Determines by bracketing chemical potential {\tt muloc} for given array of
s.p. energies, temperature {\tt T}, and wanted number of particles
{\tt occrefloc} with precision {\tt 1D-12}. Delivers with it
thermal occupation numbers and corresponding total particle number and
sum of s.p. energies.
\\
\PGRcomm{This routine is specific to {/tt SUBROUTINE ferm1}. Could we
encapsulate it by a {/tt CONTAINS}?}

\subsubsection*{\tt SUBROUTINE
  Calc\_psi1(psi1,aloc,rhotot0,rhototloc,curr0,curr1,j,lambda,mu,
\\
lambdaj,muj,sumvar2,eal,ekmod)}
\noindent
combined with encapsulated {\tt SUBROUTINE calc\_hamiltonien}.
\\
\begin{tabular}{lcl}
 {\tt j} & in & number of DCMF iteration, used here for print\\
 {\tt lambda(1:kdfull2,1:2)} & in & Lagrange parameter for density for
 spin up\&down\\
 {\tt lambdaj(1:kdfull2,1:3)} & in & Lagrange parameter for current\\
 {\tt mu, muj} & in & driving parameter for augmented Lagrangian\\
 {\tt aloc(1:2*kdfull2)} & in & local potentials for spin up and down\\
 {\tt rhoto0(1:kdfull2,1:2)} & in & initial density \PGRcomm{not used ??}\\
 {\tt curr0(1:kdfull2,1:3)} & in & wanted current \\
 {\tt psi1(1:kdfull2,1:kstate)} & in/out & set of s.p. wavefunctions iterated\\
 {\tt rhototloc(1:kdfull2,1:2)} & out & actual density according to {\tt psi1}\\
 {\tt curr1(1:kdfull2,1:3)} & out & actual current from {\tt psi1}\\
 {\tt ekmod(1:nstate)} & out & final s.p. energies\\
 {\tt eal} & out & final sum of s.p. energies\\
 {\tt sumvar2} & out & variance of s.p. energies\\
\end{tabular}
\\[4pt]
Performs one damped gradient step of with density \& current
constrained Hamiltonian.
\\
\PGRcomm{The density array distinguishes spin up/down while the
  current array does not. Reason?}
\\
PGRcomm{The IN \& OUT assignments in this subroutine have to be updated.}



\subsubsection*{\tt SUBROUTINE eqstate(psi,aloc,rho,psi1,occuporth,iterat)}
\begin{tabular}{lcl}
 {\tt iterat} & in & actual iteration number (for printing)\\
 {\tt psi(1:kdfull2,1:kstate)} & in & initial set of
 s.p. wavefunctions\\
 {\tt psi1(1:kdfull2,1:kstate)} & out & final set of s.p. wavefunctions\\
 {\tt aloc(1:2*kdfull2)} & in/out & local part of potential, spin
 up/down stacked in blocks\\
 {\tt rho(1:2*kdfull2)} & in & initial density, spin
 up/down stacked in blocks \\
 {\tt occuporth(1:kstate)} & in & occupation numbers for {\tt psi} and
 still the same for {\tt psi1}.\\
\end{tabular}
\\[4pt]
DCMF iterations by reapeatedly calling {\tt Calc\_psi1},
updating Lagrangian parameters for density \& current constraints, and
occassionally tuning temperature to achieve correct energy. The
latter is done by calling {\tt fermi1}. The local potential
is kept constant during DCMF iteration and updated only at the very end.
\\
\PGRcomm{Fetches nr. of spin up/down from {\tt m\_params}.}
\\
\PGRcomm{Lagrange parameters are started from scratch. May it be
  faster to recycle the previous Lagrange parameters?
}
\\
\PGRcomm{Density {\tt rho} is entered via list and still recomputed
as {\tt rhotot0}. Unnecessary doubling?}


\subsubsection*{\tt SUBROUTINE OccupT0(occloc,esploc,Estar)}
\begin{tabular}{lcl}
 {\tt esploc(1:nstate)} & in & given s.p. energies\\
 {\tt occloc(1:nstate)} & in & given occupation numbers\\
 {\tt Estar} & out & excitation energy relative to T=0 distribution\\
\end{tabular}
\\[4pt]
Computes thermal excitation energy as difference of actual energy to
the energy obtained by Fermi distribution for $T=0$. The latter
distributions  is computed for the given s.p. energies which are the
same as used for the thermal state.



\subsubsection*{\tt SUBROUTINE calcrhotot(rho,q0)}
\begin{tabular}{lcl}
 {\tt q0(1:kdfull2,1:kstate) } & in & set of s.p. wavefunctions for
 which density is accumulated\\
 {\tt rho(kdfull2,2)} & out & resulting density\\
\end{tabular}
\\[4pt]
Computes local density for set of wavefunctions {\tt q0}. Note that
two crucial information is communicated via module {\tt params}, namely
{\tt occup}, the array of occupation numbers, {\tt ispin} the
array assigning spin top each s.p. state, and {\tt nstate}, the number
of s.p. states.
\\
\PGRcomm{Exploiting the sorting of spin in blocks of s.p. states, we
  could rewrite the code with to {\tt SUM} statements.}


\subsubsection*{\tt SUBROUTINE calc\_var(hpsi,psi1,sumvar2)}
\begin{tabular}{lcl}
 {\tt psi1(kdfull2,kstate)}& in & set of s.p. states for which
 variance of s.p. energies of calculated\\
 {\tt hpsi(kdfull2,kstate)} & in/out & array
 $H\rightarrow\psi_\alpha$, on input in $k$-space, on output in $r$-space
\\
 {\tt sumvar2} & out  & summed variance of s.p. energies\\
\end{tabular}
\\[4pt]
Computes the sum of variances of the s.p. energies,
$\langle\hat{\Delta h}^2|rangle$. 
\\
PGRcomm{The routine projects from each $hat{h}\psi_\alpha$
all s.p. states $\psi_\beta$ from the pool of states. That is too
much. The s.p. variance should be
$\sum_\alpha\langle|\psi_\alpha|(\hat{h}-\varepsilon_\alpha)^2|\psi_\alpha\rangle$
where $\varepsilon_\alpha=\langle|\psi_\alpha|\hat{h}|\psi_\alpha\rangle$.}




\subsubsection*{\tt SUBROUTINE forceTemp(amoy,occup,n,temp,mu)}
\begin{tabular}{lcl}
 {\tt amoy(1:n)} & in & given s.p. energies\\
 {\tt occup(1:n)} & in & given thermal occupation\\
 {\tt n} & in & number of s.p. states\\
 {\tt temp} & in & temperature\\
 {\tt mu} & out & emerging chemical potential\\
\end{tabular}
\\[4pt]
Determines chemical potential for given s.p. energies and temperature
by call to {\tt OccT1}.
\\
\PGRcomm{Obsolete and never used.}




\subsubsection*{\tt SUBROUTINE fermi\_init(ekmod,T,occup,ispinact)}
\begin{tabular}{lcl}
 {\tt ekmod(1:nstate)} & in & given s.p. energies\\
 {\tt T} & in & given temperature\\
 {\tt ispinact} & in & actual spin\\
 {\tt occup(1:kstate)} & in/out & initial occupation and resulting
 Fermi distribution for {\tt T}.\\
\end{tabular}
\\[4pt]
Determines Fermi distribution for given s.p. energies and
temperature. Searches appropriate chemical potential {\tt mu} by
bracketing. Use for repreated calls to {\tt FUNCTION occ}.
\\
\PGRcomm{This routine {\tt fermi\_init} and the related
{\tt FUNCTION occ} are never used, thus obsolete. May be removed.}



\subsubsection*{\tt SUBROUTINE srhomat(psi,aloc,psiorth,occuporth)}
\begin{tabular}{lcl}
 {\tt psi(1:kdfull2,1:kstate)} & in & set of s.p. wavefunctions, not orth-normalized\\
 {\tt psiorth(1:kdfull2,1:kstate)} & out & ortho-normalized natural orbitals\\
 {\tt aloc(1:2*kdfull2)} & in & actual local potential\\
 {\tt occuporth(1:kstate)} & out & occupation numbers for
 ortho-normalized states\\
\end{tabular}
\\[4pt] Computes the density matrix of initial state goiven by set of
wavefunctions {\tt psi} together with their occupations {\tt occup},
the latter communicated through module {\tt params}.  Then
diagonalizes the density matrix and computes on {\tt psiorth} the new
wavefunctions associated with diagonal representation of the density
matrix.  
\\
Finally updates running transformation matrix {\tt psitophi} which is
communicated and stored through module {\tt params}.
\\
\PGRcomm{Usage and propagation of  {\tt psitophi} is somewhat hidden
  because it is handled through a module. Needs to be explained somwhere.}


\subsubsection*{\tt SUBROUTINE scalar(tab1,tab2,scal,ispin, mess)}
\begin{tabular}{lcl}
 {\tt tab1(1:kdfull2,1:kstate)} & in & 1. set of s.p. wavefunctions\\
 {\tt tab2(1:kdfull2,1:kstate)} & in & 2. set of s.p. wavefunctions\\
 {\tt ispin(1:nstate)} & in & spin of s.p. states\\
 {\tt mess} & in & message for print inside routine\\
 {\tt scal(nstate,nstate)} & out & matrix of wavefunction overlaps\\
\end{tabular}
\\[4pt]


\subsubsection*{\tt SUBROUTINE cdiagspin(mat, eigen, vect, N)}
\begin{tabular}{lcl}
 {\tt mat(N,N)} & in & complex Hermitean matrix to be diagonalized\\
 {\tt N} & in & dimension of matrix\\
 {\tt eigen(N)} & out & resulting eigenbvalues\\
 {\tt Vect(N,N)} & out & resulting eigenstates\\
\end{tabular}
\\[4pt]
Driver routine for diagonalization of a complex Hermitean matrix of
dimension {\tt N} which consists in a two blocks for separate spin.
Refers for each single block to routine {\tt cdiag} and subsequent
library routines contained therein.


\subsubsection*{\tt SUBROUTINE indexx (n,arrin,indx)}
\begin{tabular}{lcl}
 {\tt n} & in & length of array\\
 {\tt arrin(1:n)} & in & array to be sorted\\
 {\tt indx(1:n)} & out & pointer array \\
\end{tabular}
\\[4pt]
Evaluates sorting of an array in ascending order.



\subsubsection*{\tt SUBROUTINE occupPerSpin(mess,Occ)}
\begin{tabular}{lcl}
 {\tt mess} & in & character variable with comment printed inside routine\\
 {\tt Occ(1:2)} & out & total number of particles in each spin\\
\end{tabular}
\\[4pt]
Computes number of particles in each sin block. Uses {\tt nstate} and
occupations {\tt occup} from module {\tt params}.



\subsubsection*{\tt CorrectEnergy2(Wref,Eref,w,E,Wout,nloc)}
\begin{tabular}{lcl}
 {\tt W(1:nloc)} & in & initial occupations numbers\\
 {\tt E(1:nloc)} & in & given s.p. energies\\
 {\tt Wref} & in & reference particle number to be reached\\
 {\tt Eref} & in &  reference sum of s.p. energies to be reached\\
 {\tt nloc} & in & actual number of states\\
 {\tt Wout(nloc)} & out  & readjusted occupation numbers\\
\end{tabular}
\\[4pt]
Final energy correction by one step along Fermi distribution
(using Taylor expansion about actual distribution),
see appendix \ref{sec:corriter}).



\subsubsection*{\tt SUBROUTINE ordo\_per\_spin(psi)}
\begin{tabular}{lcl}
 {\tt psi(1:kdfull2,1:kstate)} & in/out & s.p. wavefunctions before
 and after reordering\\
\end{tabular}
\\[4pt]
Reorder states  in two blocks of spin up and down.
Applies that reshuffling to all relevant field of states,
s.p. wavefunctions {\tt psi}, spin per state {\tt ispin},
and occupations {\tt occup}. 
\\
\PGRcomm{Routine has been rendered obsolete by new initialization of
  states which produces immediately the correct sorting. But routine
  should be kept for possible later use (e.g., mixing states from
  different sources.}

\subsubsection*{\tt SUBROUTINE temperature(mu,T)}
\begin{tabular}{lcl}
 {\tt mu} & out & resulting chemical potential\\
 {\tt T} & out & resulting temperature\\
\end{tabular}
\\[4pt]
Takes s.p. energies {\tt amoy} and occupations {\tt occup} from module
{\tt params} and fits a Fermi distribution to it. Temperature and
chemical potentials of the fitted distribution are returned via list.
Calls a fitting routine {\tt lmdif1} using subroutine {\tt ff} as argument.

\subsubsection*{\tt SUBROUTINE ff(m,n,X,FVEC,IFLAG)}
\begin{tabular}{lcl}
 {\tt X(1:n)} & in & array handling chemical potential and temperature\\
 {\tt Fvec(1:m)} & out & array of mismatches of distributions\\
 {\tt n} & in & number of parameters of model, actually 2\\
 {\tt m} & in & number of entries in array\\
 {\tt iflag} & in & flaf possibly written (actually not used)\\
\end{tabular}
\\[4pt]
Mismatch of {\tt occup} (via modules {\tt params}) from Fermi
distribution to given chemical potential and temperature. To be used
in fitting routine {\tt lmdef1}.


\subsubsection*{\tt SUBROUTINE cproject(qin,qout,ispact,q0)}
\begin{tabular}{lcl}
 {\tt qin(1:kdfull2)} & in & s.p. wvaefunction to be projected\\
 {\tt q0(1:kdfull2,1:kstate)} & in & set of s.p. wavefunctions which
 is projected out from {\tt qin}\\
 {\tt ispact} & in & spin associated with {\tt qin}\\
 {\tt qout(1:kdfull2)} & out & projected s.p. wavefunction \\
\end{tabular}
\\[4pt] Projects away from {\tt qin} all contributions of the set 
{\tt q0}.  
\\ 
\PGRcomm{This routine may become obsolete if we recode the
  the variance in routine {\tt calc\_var} to meet the
  standard definition.}










\bibliographystyle{elsarticle-num}                                                        
\bibliography{dissip,cluster,relaxtime}             

\end{document}
