
%%%%%%%%%%%%%%%%%%%%%%%%%%%%%%%%%%%%
%%% PLEASE COMPILE WITH PDFLATEX %%%
%%%%%%%%%%%%%%%%%%%%%%%%%%%%%%%%%%%%
\documentclass[11pt,a4paper]{article}
%%%%%%%%%%%%%%%%%%%%%%%%%%%%%%%%%%%%
%%% PLEASE COMPILE WITH PDFLATEX %%%
%%%%%%%%%%%%%%%%%%%%%%%%%%%%%%%%%%%%
%\usepackage[xetex]{geometry}
\usepackage[T1]{fontenc}
\usepackage{geometry}
\usepackage[utf8]{inputenc}
\usepackage{xcolor}
\usepackage{parskip}
\usepackage{mathtools}
%\usepackage{physics}
\usepackage{xr-hyper}
%\usepackage[xetex]{hyperref}
\usepackage{bm}
\usepackage{hyperref}
\usepackage{hypertoc}
\usepackage{xr}
\usepackage{enumitem}
\usepackage{float}
\usepackage{fancyvrb}
%\usepackage{mathspec}
\usepackage{titlesec,titletoc}
\usepackage[singlelinecheck=false,tableposition=top]{caption}
\usepackage{tikz}
\usepackage[nottoc]{tocbibind}
\usepackage{longtable}
%\usepackage{setspace,tocloft}

\geometry{top=1in,bottom=1in,left=1in,right=1.375in}
\graphicspath{{User_manual.figures/},{figures/}}


%\titlespacing*{10pt}{10pt}{10pt}
\def\theequation{M.\arabic{equation}}
\def\thefigure{M.\arabic{figure}}
\def\thetable{M.\arabic{table}}
%\def\thesection{M.\arabic{section}\hspace*{1em}\textcolor{white}{.}}
\def\thesection{M.\arabic{section}}
%\setlength\cftsecnumwidth{8em}
%\cftsetindents{section}{4em}{40em}
%\renewcommand{\thesection}{M.\thesection}
%\setlength\cftbeforesecskip{1.3pt}
%\setlength\cftaftertoctitleskip{2pt}
%\renewcommand{\cftsecafterpnum}{\hspace*{7.5em}}
%\renewcommand{\cftsubsecafterpnum}{\hspace*{7.5em}}

%\newcounter{denselistcounter}
%\newenvironment{denselist}[1]
%{ \begin{list}
%  {#1{denselistcounter})}{\usecounter{denselistcounter}
%  \setlength{\topsep}{-0pt}
%  \setlength{\partopsep}{-0pt}
%  \setlength{\itemsep}{-0pt}
%  \setlength{\parsep}{-0pt}
%  \setlength{\labelwidth}{6pt}
%  \setlength{\labelsep}{4pt}
%  \setlength{\leftmargin}{20pt}
%  \setlength{\rightmargin}{20pt}
%  }
%}
%{\end{list}}
\setlength{\parskip}{4pt}
\unitlength 1mm
\thicklines

%\setmathfont(Latin){Minion Pro}
%\setmathrm{Minion Pro}
%\setmathsf{Myriad Pro Light} %% Make this match the headline font
%\exchangeforms{phi}			 %%   to get math in the headings right
%\defaultfontfeatures{
%	Ligatures={Required,Common,Contextual,Rare,Historic,TeX},
%	Style=Swash,
%	Contextuals={Swash,WordInitial,WordFinal}
%}
%\setmainfont{Minion Pro}
%[
%	UprightFeatures={
%		SizeFeatures={
%			{Size={-8.4},Font=* Caption},
%			{Size={8.4-13},Font=*},
%			{Size={13-19.9},Font=* Subhead},
%			{Size={19.9-},Font=* Display}
%		},
%	},
%	BoldFeatures={
%		SizeFeatures={ 
%			{Size={-8.4},Font=* Bold Caption},
%			{Size={8.4-13},Font=* Bold},
%			{Size={13-19.9},Font=* Bold Subhead},
%			{Size={19.9-},Font=* Bold Display}
%		},
%	},
%	ItalicFeatures={
%		SizeFeatures={ 
%			{Size={-8.4},Font=* Italic Caption},
%			{Size={8.4-13},Font=* Italic},
%			{Size={13-19.9},Font=* Italic Subhead},
%			{Size={19.9-},Font=* Italic Display}
%		},
%	},
%	BoldItalicFeatures={
%		SizeFeatures={ 
%			{Size={-8.4},Font=* Bold Italic Caption},
%			{Size={8.4-13},Font=* Bold Italic},
%			{Size={13-19.9},Font=* Bold Italic Subhead},
%			{Size={19.9-},Font=* Bold Italic Display}
%		},
%	},
%]
%\setsansfont{Myriad Pro}
%\newfontfamily\headlinefont{Myriad Pro Light}
%\newfontfamily\headlinefont{Latin Modern Sans}

%\linespread{1.3} % Only touch if the linespacing is insufficient

\definecolor{niceBlue2}{RGB}{16,117,175}
\colorlet{activeColor}{niceBlue2}
\hypersetup{
	colorlinks=true,
	linkcolor=activeColor,
	filecolor=activeColor,
	urlcolor=activeColor,
	citecolor=activeColor,
}\urlstyle{same}

\titleformat{\section}[hang]
%	{\headlinefont\LARGE}
	{\sffamily\LARGE}
	{\thesection}
	{12pt}
	{\color{activeColor}}
\titlespacing{\section}{0em}{4ex}{1ex}
\titleformat{\subsection}[hang]
%	{\headlinefont\large}
	{\sffamily\large}
	{\thesubsection}
	{12pt}
	{\color{activeColor}}
\titlespacing{\subsection}{0em}{4ex}{1ex}
\titleformat{\subsubsection}[hang]
%	{\headlinefont\itshape}
	{\sffamily\itshape}
	{\thesubsubsection}
	{12pt}
	{\color{activeColor}}
\titlespacing{\subsubsection}{0em}{4ex}{1ex}

\titlecontents{section}[0em]
%	{\addvspace{2ex}\headlinefont\large}
	{\addvspace{2ex}\sffamily\large}
	{\hspace*{1.5em}{\contentslabel{2.5em}}}
	{\hspace*{1.5em}{\contentslabel{2.5em}}}
	{\titlerule*[0.7em]{.}\contentspage}
	[]
\titlecontents{subsection}[2.025em]
%	{\addvspace{1ex}\headlinefont\normalsize}
	{\addvspace{1ex}\sffamily\normalsize}
	{\hspace*{3.35em}{\contentslabel{3.5em}}\color{activeColor}}
	{\hspace*{3.35em}{\contentslabel{3.5em}}\color{activeColor}}
	{\titlerule*[0.7em]{.}\contentspage}
	[]
\titlecontents{subsubsection}[4.24em]
%	{\headlinefont\itshape\small}
	{\sffamily\itshape\small}
	{\hspace*{5.5em}{\contentslabel{4em}}}
	{\hspace*{5em}{\contentslabel{4em}}}
	{\titlerule*[0.7em]{.}\contentspage}
	[]

\captionsetup[table]{labelfont={sc,color=activeColor},name=Table}
\captionsetup[figure]{labelfont={sc,color=activeColor},name=Figure}

\renewcommand{\texttt}[1]{{\bfseries\ttfamily #1}}
\usetikzlibrary{shapes}
\newcommand*\circled[1]{
	\tikz[baseline=(char.base)]{
		\node[shape=circle, draw, inner sep=2pt] (char) {#1};
	}
}
\makeatletter
	\newcommand{\showfontsize}{\f@size{} pt}
\makeatother
\newcommand{\cmnt}[1]{\color{red}\uppercase{#1}}
\newcommand{\eq}[1]{Equation~(\ref{#1})}	   % \eq{label}  --> Equation (3)
\newcommand{\scn}[1]{Sec.~\ref{#1}}	   % \scn{label} --> Section 1.2
\newcommand{\fig}[1]{Fig.~\ref{#1}}	   % \fig{label} --> Figure 1.3
\newcommand{\figs}[1]{Figs.~\ref{#1}}	   % \fig{label} --> Figure 1.3
\newcommand{\rf}[1]{~Ref.$\!$\citen{#1}}	   % \rf{label}  --> ~Ref. [11]
\newcommand{\rfs}[1]{~Refs.$\!$\citen{#1}} % \rfs{label} --> ~Refs. [11,15,21-23]
\newcommand{\tab}[1]{Tab.~\ref{#1}}		   % \tab{label} --> Table 2.4
\newcommand{\tabs}[1]{Tabs.~\ref{#1}}	   % \tabs{label},~{label} --> Tables 2.4, 1.3
\newcommand{\refM}[1]{M.\ref{#1}}	   % \rf{label}  --> ~Ref. [11]

\setcounter{totalnumber}{4}
\renewcommand{\topfraction}{.95}
\renewcommand{\bottomfraction}{.95}
\renewcommand{\textfraction}{.05}
\renewcommand{\floatpagefraction}{.66}
\renewcommand{\dbltopfraction}{.66}
\renewcommand{\dblfloatpagefraction}{.66}

\definecolor{dgreen}{rgb}{0.0,0.7,0.0}
\newcommand{\tgreen}[1]{\textcolor{dgreen}{#1}}


\newcommand{\PGR}[1]{{\color{blue} #1}}
\newcommand{\PGRcomm}[1]{{\color{blue}\small\em PGR2all: #1}}
\newcommand{\PGRfoot}[1]{{\color{blue}\footnote{\color{blue} #1}}}
\newcommand{\ES}[1]{{\color{red} #1}}
\newcommand{\EScomm}[1]{{\color{red}\small\em ES2all: #1}}
\newcommand{\ESfoot}[1]{{\color{red}\footnote{\color{red} #1}}}
\newcommand{\MD}[1]{{\color{magenta} #1}}
\newcommand{\MDcomm}[1]{{\color{magenta}\small\em MD2all: #1}}
\newcommand{\MDfoot}[1]{{\color{magenta}\footnote{\color{magenta} #1}}}
\newcommand{\FC}[1]{{\color{olive} #1}}
\newcommand{\FCcomm}[1]{{\color{olive}\small\em FC2all: #1}}
\newcommand{\FCfoot}[1]{{\color{olive}\footnote{\color{olive} #1}}}

\externaldocument[cpc:]{CPC_RTA_v1}

\title{\fontsize{50}{60}\selectfont{Quantum Dissipative Dynamics}\\\vspace{7ex}\fontsize{50}{60}\selectfont{\textsf{User manual}}\vspace{8ex}}
\author{F.M.G.J. Coppens\\P.-G. Reinhard\\P. M. Dinh\\E. Suraud\\M. Vincendon}

\begin{document}

\maketitle
\thispagestyle{empty}

\clearpage
\addtocontents{toc}{\hfill\small{\sffamily Page}\par}
\section*{About this manual}

This is the manual for the `Quantum Dissipative Dynamics' (QDD)
code. It is developed in conjunction with the Computer Physics
Communications (CPC) reference paper. The paper describes the basic
physics dealt with in the QDD code. The manual will mainly address
the technical aspects and add physics details which are not explained
in the paper. It is roughly divided in the following four parts:\\
\begin{itemize}
\item Sections \ref{sec:prerequis} and \ref{sec:install} deal with the
  hard- and software needs and what needs to be done to obtain a
  working copy of QDD.
\item Sections \ref{sec:stat_basics} and \ref{sec:stat_examples}
  explain ground state and stationary calculations with some
  illustrative examples.
\item Section \ref{sec:dyn_basics} and \ref{sec:dyn_examples} explain
  the dynamical aspects of the calculation with again some typical
  example cases.
\item Finally, section \ref{sec:mistakes}  deal
  with common mistakes and pitfalls that are encountered frequently
  and less frequently. 
 \end{itemize}
The last sections deal with more advanced use
of the code and explain QDD's basic code structure.

It is important to note that references to section and equation
numbers in this manual are headed by a label ``M'' while numbers
without initial ``M'' point to section and equation numbers in the
reference paper. 



\clearpage
\begingroup 
\hypersetup{hidelinks} 
\tableofcontents 
\endgroup


% 0be00350db7ccd73db794a7dd9901aac317d782a


\clearpage

\section{Prerequisites}
\label{sec:prerequis}

To successfully obtain and install the QDD software package, one needs:
\begin{itemize}
\item a Fortran90 compiler, e.g. Intel Fortran (\texttt{ifort}) or GNU
  Fortran (\texttt{gfortran}). The compiler should comply with the
  Fortran2008 standard.
\item the discrete Fourier Transform library Fastest Fourier Transform in the West (FFTW)
   or its implementation on Intel's MKL library. 
\item the Linux \texttt{make} program to build the code; other
  \texttt{make} utilities can also be used, but may require
  modification of the makefiles.
\end{itemize}	
The code was tested with the GNU Fortran compiler version 5.4.0
togther with the FFTW3 package version 3.4.4 and the Intel Fortran
compiler version 17.05 and the corresponding MKL version. It
will work then for later versions as well. Earlier versions may also work, but
have not been tested.  If you are working on a larger computing
network, it may happen that Fortran compilers, FFTW3, or MKL are not
immediately available. They are often supplied as a module which has
to be loaded before use. To find out ask your system's manager.

If you do not find the necessary software on your system, you find the
freeware GNU Fortran compiler under
\href{https://gcc.gnu.org/fortran}{https://gcc.gnu.org/fortran} and
FFTW3 package under \href{http://www.fftw.org}{http://www.fftw.org}.
The Intel compiler and MKL can be found at the vendors site
\href{https://software.intel.com}{https://software.intel.com}$\;,$,
both being also freely available.
(The links were checked March 2021 and may change with the years.
Alternatively, one can search the sites by keyword.)


\clearpage

\section{Installation}	
\label{sec:install}

This section is concerned with how to install the code
and to compile it. For the examples treated here, the most basic
settings are chosen so as to minimize the risk of complications. For
the full list of compilation parameters and supported libraries,
please consult \scn{sec:adv_compil}.

		
\subsection{Structure of sub-directories}

After download of the compressed archive \texttt{QDD.zip} with the
project, we assume that it is placed on the project directory with a
name of the user's choice, We refer to it henceforth as
\texttt{\$QDD\_ROOT}. After unpacking, one will find the following
sub-directories:
\begin{description}
  \item[\texttt{\$QDD\_ROOT/bin}:] directory where the \texttt{qdd}
    binary is installed in after compilation
  \item[\texttt{\$QDD\_ROOT/doc}:] contains this user manual
  \item[\texttt{\$QDD\_ROOT/examples}:] contains the examples presented in
    the manual and in the reference paper with input files and typical output files;
    each example is placed in a separate sub-directory with obvious name
  \item[\texttt{\$QDD\_ROOT/src}:] contains 2 sub-directories where
    source code is stored; the directory \texttt{\$QDD\_ROOT/src/qdd}
    contains the QDD code and according make files (see
    \scn{sec:compil}) while the directory
    \texttt{\$QDD\_ROOT/src/auxiliary} contains auxiliary programs for
    post-processing.
\end{description}

\subsection{Compilation}
\label{sec:compil}

		
The first step to produce an executable by compiling and linking.  To
that end, you have to go to directory where the source code is:
\begin{verbatim}
$ cd $QDD_ROOT/src/qdd
\end{verbatim}
Compilation is done by issuing a \texttt{make} in this directory.
However, before doing that one has to adapt the compiler settings to
the given situations and wanted options. To that end, edit the
\texttt{Makefile} in the directory \texttt{\$QDD\_ROOT/src/qdd} and
chose from the following specifications:
\begin{description}
\item[\texttt{COMPILER}:] This sets the compiler of your choice.
  Presently provides are the instructions for invoking
  \texttt{TYPE\_FFT = gfortran} if you use the GNU Fortran and
  \texttt{TYPE\_FFT = ifort} if you use Intel Fortan.
\item[\texttt{OS}:] This specifies the operating system of your
  computer. Presently  available are 
  \texttt{OS = LINUX} for Linux and
  \texttt{OS = MAC} for a Mac operating system.
\item[\texttt{TYPE\_FFT}:] There are two alternatives, set to
  \texttt{TYPE\_FFT = FFTW} if you use a FFTW3 library or to
  \texttt{TYPE\_FFT = MKL} if you have access to the MKL
  implementation of FFTW3.
\item[\texttt{OMP}:] Activates OpenMP parallelization if set to
  \texttt{YES} and compiles sequential code if \texttt{NO}.
\item[\texttt{DYNOMP}:] There are two variants within OpenMP
  parallelization. The choice \texttt{DYNOMP = YES} operates with
  parallelization of the s.p. wave functions while  \texttt{DYNOMP = NO}
  exploits the OpenMP version of the FFTW3  to speed up the
  Fourier transformations.
\item[\texttt{LINK\_STATIC}:] Activates static linking is set to
  \texttt{YES} and dynamic linking for \texttt{NO}.
\item[\texttt{DEBUG}:] Compiles with debugging options if set to
  \texttt{YES} and with optimization for \texttt{NO}.
\item[\texttt{FFTW\_PATH} (optional):] This applies if the linker
  option \texttt{-lfftw3} did not find automatically the right path
  (most installations do that correctly).  Then you have to find the
  path to the FFTW3 library on your system and to insert it here. The
  path should point the top-level directory of your FFTW3
  installation.
\item[\texttt{MKL\_PATH} (optional):] This applies if the linker did
  not find the MKL library (most installations provide the right
  path).  Then you have to find the path to the FFTW3 library on your
  system and to insert it here. The path should point the top-level
  of your MKL installation
\end{description}
The \texttt{Makefile} points to compiler specific settings in
the files \texttt{Makefiles/mfBody.gnu.mk} for GNU Fortran or
\texttt{Makefiles/mfBody.intel.mk} for Intel Fortran. Both files
contain standard compiler options which should  run on most systems.
If that does not work immediately or if you dispose of particular
optomization options, you have to edit the compiler options in the
actually relevant of these both files. 

Now your are ready for issuing a \texttt{make} command. However, you must run a
\texttt{make clean} before if you have edited the \texttt{Makefile} of
one of the included files in the sub-directory \texttt{Makefiles/}.
After the \texttt{make}, you find the executable \texttt{qdd}
in the directory \texttt{\$QDD\_ROOT/bin}.

The \texttt{gfortran} and \texttt{ifort} are supplied as widely
accessible examples.  Of course, you can also compile and link with
other compilers.  In that case, you have go deeper into the
makefiles. A safe way is to generate a new
\texttt{mfBody.<your\_compiler>.mk} from copying one of the two
\texttt{mfBody} files and edit that according to the command structure
of your compiler. Then you need to point to your new
\texttt{Makefiles/mfBody.<your\_compiler>.mk} in the
\texttt{Makefile} and your are ready to go again.

		
\subsection{Launching a calculation}
\label{sec:launch}

After the build process is finished, the executable \texttt{qdd} will
be in the \texttt{\$QDD\_root/bin} directory.
		
Each calculation uses its own set of input files and produces a bunch
of output files. We therefore recommend the user to work in a
dedicated directory for each single calculation.
The code resides at the fixed position
\begin{verbatim}
 $QDD_ROOT/bin/qdd 
\end{verbatim}
and can be invoked from there. The user can, of course, copy it to the
working directory, install a link, or rename it according to one's
preferred working style.
		
There are 2-3 input files required to start a calculation. They are
listed in \tab{tab:input-files}.
%	
\begin{table}[htbp!]
  \caption{Minimum set of input files.}\label{tab:input-files}
  \begin{tabular}{|p{3.2cm}|p{10.8cm}|}
    \hline \texttt{for005} & Top level file containing the calculation
    identifier `\texttt{<name>}'. This can be any string up to 13
    characters, e.g. `\texttt{H2O}' or
    `\texttt{Na8-ionmot}'. \\ \hline \texttt{for005.<name>} & Contains
    the parameters of the calculation using Fortran's
    \textit{namelist}-mechanism. Description of all these input
    parameters is given in subsequent sections and summarized in
    Tables~\ref{tab:input-params-sys-choice},
    \ref{tab:input-params-global-initwf}, 
    \ref{tab:input-params-global-initions}, 
    \ref{tab:input-params-PsP}, 
    \ref{tab:input-params-static},
 %\ref{tab:input-params-static-dynamic},
 \ref{tab:dyn-input-params-general},
 \ref{tab:dyn-input-params-excitation},
 \ref{tab:dyn-input-params-observables}, and \ref{tab:dyn-input-params-rta}.\\ \hline
    \texttt{for005ion.<name>} & Locations and types of the ions in
    case of detailed ionic background, read in if \texttt{nion2=1} in {\tt for005.<name>}.\\ \hline
  \end{tabular}
\end{table}
%
A complete list of the input parameters and their explanation will be
covered in Secs.~\ref{sec:global}, \ref{sec:stat_basics}, and
\ref{sec:dyn_basics}. Examples of input with corresponding output
files can be found in the directory \texttt{\$QDD\_ROOT/examples/} and
will be discussed in Secs.~\ref{sec:stat_examples}
and~\ref{sec:dyn_examples}.
		
Provided that the input files are present in the working directory and
correctly fulfilled, to launch a calculation, execute:
\begin{verbatim}
 $QDD_ROOT/bin/qdd 
\end{verbatim}
or whatever name and place you have given the executable.

\clearpage

\section{Basic input setting system, grid, and initialization parameters}
\label{sec:init_basic}
	
This section explains the input parameters which define the system
(number of electrons, number and sort of ions), the grid (number of
grid points, spacing, options), and the way electronic wave functions
and ionic background is initialized. This is all contained in the
namelist groups \texttt{GLOBAL} and \texttt{PERIO}.  The latter is
confined to the ionic pseudopotentials. The namelist \texttt{GLOBAL}
is richer and will be presented in several portions.

\fbox{\begin{minipage}{0.98\linewidth} {\bf Important note:} All
    tables of input parameters, here and in subsequent sections, give
    default values where they apply. These are distinguished by
    \tgreen{green color} and shown in a separate column, usually
    column 3 except for Tab.~\ref{tab:dyn-input-params-observables}
    where it is column 2.  Parameters which are declared
    \tgreen{$\emptyset$} are undefined and must be given explicitly in
    the input files, else the code terminates right after reading. All
    real parameters in the code are, in fact, double precision
    variables. It is only for reason of better readability that we
    specify default values often in the simpler form as, e.g., 1.0
    instead of 1D0.  Boolean, or logical parameters respectively, can
    be set in an input file either with the syntax {\tt F} (or {\tt
      T}), or with the syntax {\tt .FALSE.} (or {\tt .TRUE.}), while
    in the source code, only the second syntax is possible.  If a
    parameter has a physical dimension, the corresponding unit is
    shown in column 2.
\end{minipage}
}
	
\subsection{Parameters in the namelist GLOBAL}
\label{sec:global}
	
\subsubsection{Defining system and numerical basis}

%\begin{longtable}{|p{2.5cm}|p{1.0cm}|p{10.0cm}|}
\begin{table}[bht]
\caption{General parameters on the physical system and the numerical
  box in the namelist \texttt{GLOBAL} of
  \texttt{for005.<name>}. 
%  The third column shows the default values. An entry
%  $\emptyset$ means that this parameter is undefined and must be given explicitly.
}\label{tab:input-params-sys-choice}
%\endfirsthead
\begin{tabular}{|p{2cm}|p{1.1cm}|p{1.3cm}|p{8.9cm}|}
\hline
\multicolumn{4}{|c|}{\texttt{GLOBAL} namelist}\\
\hline
\multicolumn{4}{|c|}{\textit{\color{activeColor}Setting system}}\\
\hline
\texttt{nelect}&$N_\mathrm{el}$&\tgreen{$\emptyset$}& Number of valence electrons \\
\hline
\texttt{nspdw}&$N_\downarrow$&\tgreen{$\emptyset$}& Number of spin down electrons\\
\hline
\texttt{nion}&$N_\mathrm{ion}$&\tgreen{$\emptyset$}& Number of ions \\
\hline
\texttt{numspin}&&\tgreen{2}& Number of spin components\\
		&&& 1 $\rightarrow$ spin degenerated \\
		&&& 2 $\rightarrow$ full spin treatment\\
\hline
\texttt{temp} & $T$ [Ry]&\tgreen{0.0}&   Electron temperature, see
eq. (\ref{cpc:eq:Fermi0}) and \scn{sec:open-shell}\\
\hline
\multicolumn{4}{|c|}{\textit{\color{activeColor}Numerical grid}}\\
\hline
\texttt{kstate}&&\tgreen{20}& Max. number of possible electron states,
must be $\geq$ \texttt{nelect}\\
\hline
\texttt{kxbox}, &$N_x$&\tgreen{$\emptyset$}& Number of grid points in $x$ direction, see \scn{cpc:sec:griddef}\\
\texttt{kybox}, &$N_y$&\tgreen{$\emptyset$}& Number of grid points in $y$ direction, see \scn{cpc:sec:griddef}\\
\texttt{kzbox} &$N_z$&\tgreen{$\emptyset$}& Number of grid points in $z$ direction, see \scn{cpc:sec:griddef}\\
\hline
\texttt{dx}&$\delta x$ [a$_0$]&\tgreen{$\emptyset$}& Grid spacing in $x$ direction, see \scn{cpc:sec:griddef}\\
\texttt{dy}&$\delta y$ [a$_0$]&\tgreen{\texttt{dy=dx}}& Grid spacing in $y$ direction, see \scn{cpc:sec:griddef}\\
\texttt{dz}&$\delta z$ [a$_0$]&\tgreen{\texttt{dz=dx}}& Grid spacing in $z$ direction see \scn{cpc:sec:griddef} \\
\hline
\texttt{tcoulfalr} &&\tgreen{\tt F}&   Switch to FALR Coulomb solver, else exact solver\\
\hline
\texttt{numthr} &&\tgreen{0}&   Max. number of OpenMP threads to use\\
  &&& $=0$ $\rightarrow$ get \texttt{numthr} from system\\
  &&& $<0$ $\rightarrow$ fixed setting (must stay below system limit)  \\
\hline
\end{tabular}
\end{table}
Table \ref{tab:input-params-sys-choice} collects the basic parameters
of a calculation, the system in terms of numbers of electrons and ions
as well as the numerical grid. Most of the entries are self
explaining. A few of them need some more explanation.

The parameter \texttt{numspin} sets the number of spin components in
the calculations. The standard is to deal with both components, spin
up and spin down, explicitly (\texttt{numspin=2}).  However, there are
many situations where we know in advance that spin degeneracy is
maintained throughout the whole process. In that case, one can save
storage space and computing time when handling only one spin component
(\texttt{numspin=1}). A typical case is the large C$_{60}$ system, see
the examples in the directory \texttt{examples}. In case of separate
spin-up and spin-down (\texttt{numspin=2}), the parameter
\texttt{nspdw} has to be specified which fixes the number of spin down
electrons. Proper choice allows one to consider spin polarized systems or
systems in which there is an explicit violation of spin degeneracy,
such as for example the carbon atom.

The parameter \texttt{kstate} is used in defining the dimensions of
the wave function arrays. It sets the upper limits for the number of
states which can be covered in a calculation. The actual number of
states is determined at the time of wave function initialization, see
Sec. \ref{sec:initwf}. It is denoted \texttt{nstate} in the code
and corresponds to the quantity $\Omega$ in the paper (see
eq. (\ref{cpc:eq:gen_nbrs})).  Note that it is not a direct input of
the code. It results from the initialization itself.  A message will
be printed if the actual number exceeds \texttt{kstate} and the code
terminates prematurely.

The parameter \texttt{numthr} is only applicable if the code was
compiled with OpenMP activated. It allows one to specify the number of
threads explicitly. Default is \texttt{numthr=0} which sets the
number of threads to the value provided by the actual computer
system.

%\paragraph{Coulomb solvers}

The Boolean parameter
\texttt{tcoulfalr} switches the Coulomb
solver. This requires some explanation. 
The Coulomb potential $U_\mathrm{C}$ to a given charge density
$\varrho_\mathrm{C}$ is determined by solving the Poisson equation
\begin{equation}
  \Delta U_\mathrm{C} + 4\pi e^2\varrho_\mathrm{C} = 0
  \;.
\end{equation}
We do that in Fourier space where the Laplacian operator $\Delta$
amounts to simple multiplication. A problem is the long-range nature,
$\propto r^{-1}$, of the Coulomb potential which means that sizable
values of $U_\mathrm{C}$ remain at the bounds of typical numerical
boxes.  There are two options to deal with the long-range part.
%
The first scheme, switched by {\tt tcoulfalr=.FALSE.}, manages to
reproduce the exact Coulomb field at the bounds. To that end, it
doubles the 3D grid in each spatial direction amounting to an eight
times larger grid altogether, constructs a numerical representative of
the Laplacian from the exact Coulomb Green's function on the grid, and
solves the Poisson equation in Fourier space of the eightfold grid,
for details see \cite{Eas79,Mar14aR}.
%
The second scheme, switched by {\tt tcoulfalr=.TRUE.}, approximates
the $U_\mathrm{C}$ at the bounds by a multipole expansion going up the
hexadecapole order, the residual error thus being of order $r^{-5}$.
It separates the given density into a short-range part which produces
no multipole fields up to hexadecapole and a long-range part covering
all remaining contributions, particularly the long-range part. The
short-range part is solved in the Fourier space of the regular
grid. The long-range part uses model densities for which exact
solutions are known and which are adjusted to reproduce all multipole
moments up to hexadecapole.  The final Coulomb field is then obtained
from summing the separately computed short-range and long-range parts.
The method is coined Fourier Analysis with Long Range forces (FALR),
for details see \cite{Lau94}.

The grid spacings could be set differently. But it is highly
recommended to use the same spacing in all three directions. This is
the default. One should only specify \texttt{dx}. The \texttt{dy} and
\texttt{dz} will then follow automatically. Specifying all three
parameters explicitly, allows one to override the default.
	
\subsubsection{Initialization of the electronic wave functions}
\label{sec:initwf}

%\begin{longtable}{|p{2.5cm}|p{1.0cm}|p{10.0cm}|}
\begin{table}
\caption{Parameters for electronic properties in namelist
  \texttt{GLOBAL} of
  \texttt{for005.<name>}\label{tab:input-params-global-initwf}}
%\endfirsthead
\begin{tabular}{|p{2.3cm}|p{1.5cm}|p{0.45cm}|p{9.0cm}|}
\hline
\multicolumn{4}{|c|}{\texttt{GLOBAL} namelist}\\
\hline
\multicolumn{4}{|c|}{\textit{\color{activeColor}Initialization of
    electron wave functions}} \\
\hline
\texttt{osfac} &$\eta_\mathrm{width}$&\tgreen{1.0}& Scaling width of the
oscillator functions, eq. (\ref{eq:HOwf})   \\       
\hline
\texttt{b2occ}&$\beta_\mathrm{init}$&\tgreen{0.0}& Deformation for initial harmonic oscillator wave functions \\
\hline
\texttt{gamocc}&$\gamma_\mathrm{init}$&\tgreen{0.0}& Triaxiality for initial harmonic oscillator wave functions \\
\hline
\texttt{deocc}&$\Delta\varepsilon_\mathrm{occ}$&\tgreen{0.0}&
   Size of additional shell of initial states above Fermi energy \\
\hline
\texttt{shiftWFx}&${r}_{\mathrm{shift},x}$[a$_0$]&\tgreen{0.0}&
     Shift of initialized wave functions in $x$-direction \\
\texttt{shiftWFy}&${r}_{\mathrm{shift},y}$[a$_0$]&\tgreen{0.0}&
     Shift of initialized wave functions in $y$-direction \\
\texttt{shiftWFz}&${r}_{\mathrm{shift},z}$[a$_0$]&\tgreen{0.0}&
     Shift of initialized wave functions in $z$-direction \\
\hline
\texttt{ispinsep}&&\tgreen{0}& Initialisation of wave functions with spin asymmetry\\
\hline
\texttt{init\_ao}&&\tgreen{F}& Initialize wave functions with atomic orbitals \\
\hline
\multicolumn{2}{|l|}{\texttt{tshiftCMtoorigin}}& \tgreen{F} & Shift center of mass of ions to origin of numerical box\\
\hline
  \multicolumn{4}{|c|}{\textit{\color{activeColor}Choice of electronic
      functional and pseudopotential}}\\
  \hline
  \texttt{idenfunc} &&\tgreen{1}& Choice of density functional for LDA\\
    &&& 1  $\rightarrow$ Perdew \& Wang 1992 \cite{Per92} \\
    &&& 2 $\rightarrow$ Gunnarson \& Lundquist  \cite{Gun76}\\
    &&& 3 $\rightarrow$ only exchange in LDA \\
  \hline
  \texttt{ifsicp}&&\tgreen{2}& Type of self-interaction correction (SIC, see
      \scn{cpc:sec:SIC})\\
    &&& 0 $\rightarrow$ pure LDA\\
    &&& 2 $\rightarrow$ ADSIC \\
    &&& 3 $\rightarrow$ SIC-Slater\\
    &&& 4 $\rightarrow$  SIC-KLI\\
    &&& 5 $\rightarrow$ exact exchange\\
  \hline
  \texttt{ipsptyp}&&\tgreen{$\emptyset$}& Type of pseudopotentials: \\
    &&& 0 $\rightarrow$  soft local (\texttt{errf})\\
    &&& 1 $\rightarrow$ full Goedecker\\
    &&& 2 $\rightarrow$ local Goedecker\\
\hline
\multicolumn{4}{|c|}{\textit{\color{activeColor}Setting for observables}} \\
\hline
  \texttt{tmoms\_rel\_cm}&&\tgreen{F}& Origin for calculating multipole momenta
  of electron density \\
  &&& \texttt{F} $\rightarrow$ relative to numerical box origin \\
  &&& \texttt{T} $\rightarrow$ relative to center of mass of the system\\
\hline
\end{tabular}
\end{table}
%\end{longtable}

Table \ref{tab:input-params-global-initwf} collects the parameters for
the initialization of the electronic wave functions, choice of
electronic energy functional, and reference point for global
observables.  There are two strategies for wave function
initialization.

The option \texttt{init\_ao=.FALSE.} switches to harmonic oscillator
states. That initialization is inspired by the Clemenger-Nilsson model
for the wave functions of metal clusters \cite{Cle85}, but widely used
also elsewhere \cite{Mar10aB}. The first step is to determine an
appropriate sequence of initial states. To this end, we use the
closed formula for energies of the states of triaxially deformed
harmonic oscillator
\begin{subequations}
\label{eq:initho}
\begin{equation}
  \varepsilon_{n_xn_yn_z}
  =
  n_x\hbar\omega_x+  n_y\hbar\omega_y+  n_z\hbar\omega_z+\frac{3}{2}
\end{equation}
with the oscillator parameters defined as
\begin{eqnarray}
\\
  \hbar\omega_x
  &=&
  \hbar\omega_0
  \frac{1}{1-\sqrt{\frac{5}{16\pi}}\beta_\mathrm{init}
       (\cos(\gamma_\mathrm{init})-\sin(\gamma_\mathrm{init}))}
  \;,
\\
  \hbar\omega_y
  &=&
  \hbar\omega_0
  \frac{1}{1-\sqrt{\frac{5}{16\pi}}\beta_\mathrm{init}
       (\cos(\gamma_\mathrm{init})+\sin(\gamma_\mathrm{init}))}
  \;,
\\
  \hbar\omega_z
  &=&
  \hbar\omega_0
  \frac{1}{1+2\sqrt{\frac{5}{16\pi}}\beta_\mathrm{init}\cos(\gamma_\mathrm{init})}
  \;,
\\
  \hbar\omega_0
  &=&
  \frac{1}{4}(N_\mathrm{e})^{-1/3}\frac{\hbar^2}{2m_\mathrm{el}a_0^2}
  \;.
\end{eqnarray}
\end{subequations}
Reference for filling the state is the Fermi energy which is estimated
for electrons with spin $\sigma$ in the oscillator model as (in units
of Ry)
\begin{subequations}
\begin{eqnarray}
  \epsilon_{\mathrm{F},\sigma}
  &=&
  \left(\frac{6N_\sigma}{1-(6N_\sigma)^{-2/3}}\right)^{1/3}-\frac{3}{2}
  \;.
\end{eqnarray}
We want to be flexible and augment that by an additional energy band
$\Delta\varepsilon_\mathrm{occ}$ which yields a cutoff energy and
subsequently cutoff criterion
\begin{eqnarray}
  \epsilon_\mathrm{cutoff}
  &=&
  \epsilon_{\mathrm{F},\sigma}+\Delta\varepsilon_\mathrm{occ}
  \;,
\\
  \varepsilon_{n_xn_yn_z}
  &<&
  \epsilon_\mathrm{cutoff}
  \;.
\label{eq:cutoffinitwf}
\end{eqnarray}
\end{subequations}
The states are selected in order of increasing energy until condition
Eq. (\ref{eq:cutoffinitwf}) is reached. That is done for spin-up and
spin-down separately. Finally, it is checked whether the actual number
of states stays below or equal the parameter \texttt{kstate}, see
table \ref{tab:input-params-sys-choice}. We must emphasize that the
initial deformations $\beta_\mathrm{init}$ and $\gamma_\mathrm{init}$
are most crucial parameters. They determine the sequence of initial
states in terms of their numbers of nodes in $x$-, $y$-, and
$z$-directions. A proper choice is important to avoid side-trapping in
isomeric electron states. For an example see Sec. \ref{sec:initelec}.


Having identified the wanted states, we initialize the corresponding
wave function by the well known oscillator wave functions
\begin{subequations}
\label{eq:HOwf}
\begin{equation}
  \psi_{n_xn_yn_z}(\mathbf{r})
  =
  \psi_{n_x}(x)\psi_{n_y}(y)\psi_{n_z}(z)
\end{equation}
where
\begin{eqnarray}
  \psi_{n_x}(x)
  &\propto&
  \exp\left(-\frac{x^2}{2\delta_x^2}\right)H_{n_x}\left(\frac{x}{\delta_x}\right)
  \;,
\\
  \delta_x
  &=&
  \eta_\mathrm{width}\sqrt{\frac{2}{\hbar\omega_x}\,\frac{\hbar^2}{2m_\mathrm{e}}}
\end{eqnarray}
\end{subequations}
and analogously for $\psi_{n_y}(y)$ and $\psi_{n_z}(z)$.  The input
allows one to tune the width of the oscillator states by the factor
$\eta_\mathrm{width}\equiv$\texttt{osfac} in cases where the initial
guess from the oscillator model was felt to be inefficient.

The option \texttt{init\_ao=.TRUE.} switches to a localized
initialization which associates electron states with atomic orbitals
at each ion. As we do not have a simple energy estimator, this
strategy works presently only for cases where we want to have only
fully occupied states (all states associated with occupation number
$w_\alpha=1$).  For the following explanation, we introduce two
quantities related to the ionic pseudopotentials, $Z_I^{\rm (PsP)}$ the
charge of the ionic core for the given pseudopotential and
$Z_\mathrm{eff}=\sum_IZ_I^{\rm (PsP)}$ the total charge of the ionic
cores.  The scheme then proceeds as:
\begin{enumerate}
  \item\label{it:fill} 
    We go through the ions in the order as they are given in
    \texttt{for005ion.<name>} and fill each ion with  $Z_I^{\rm (PsP)}$
    electrons according to the prescription explained in the next step.
  \item For each ion, we start occupying 
     with spin {\tt ipol(ion)} as given as last entry of a line
    in \texttt{for005ion.<name>} and switch
    up$\leftrightarrow$down for each next electron. The states are
    filled along the levels of the spherical harmonic oscillator in
    the order as prescribed in column 5 of the file
    \texttt{for005ion.<name>}, see paragraph \ref{sec:for005ion}.
    We choose for the electronic states the ones given in Eq. \eqref{eq:HOwf} without
    deformation and where the oscillator width is specified 
    in column 6 of the file
    \texttt{for005ion.<name>}. % \ref{sec:for005ion}.
  \item The above process terminates if all \texttt{nelect} electrons
    are distributed (case $N_\mathrm{el}\leq Z_\mathrm{eff}$) or if all
    ions are neutralized (case $N_\mathrm{el}\geq Z_\mathrm{eff}$).
    For anions, we have  $N_\mathrm{el}>Z_\mathrm{eff}$ and
    there remain  $N_\mathrm{el}-Z_\mathrm{eff}$ electrons not yet
    defined. These remaining electrons are distributed over the
    ions  one-by-one in the
    order they are given in the file \texttt{for005ion.<name>}.  This
    defines the number of electron states per ion, internally stored
    in the array \texttt{nmxst(:)}.
\end{enumerate}
After all, we see that the atomic orbital initialization is
straightforward for neutral systems with minimal net spin, but can
become problematic for anions and cations because we do not know ahead
of time where is the best place to put the extra
electron or to leave the hole. Moreover, this initialization is well suited only to
  cases where the final electron states are all localized. Thus we
  prefer \texttt{init\_ao=.FALSE.} in most cases. For a detailed
  discussion of initializations see Sec. \ref{sec:initelec}.


\subsection{Choice of electronic functional}

Table \ref{tab:input-params-global-initwf} shows a further global
setting, namely the selection of the electronic energy-density
functional with \texttt{idenfunc}, the treatment of the SIC
determined by parameter \texttt{ifsicp}, and the 
choice of pseudopotential with \texttt{ipsptyp}. The details of SIC are
explained in Sec. \ref{cpc:sec:SIC} of the reference paper. We emphasize here
once again that SIC-Slater and KLI are not suited for dynamical
simulations over long periods \cite{Mun05}. Exact exchange works in
all regimes, however requiring that all active electron states in the
calculations are fully occupied ($w_\alpha=1$). The best compromise in
most situations is ADSIC which provides appropriate single particle (s.p.) energies in
surprisingly many situations \cite{Klu13}.
The various choices of pseudopotentials are explained in 
\scn{sec:explions}.


The last parameter in table \ref{tab:input-params-global-initwf} is
\texttt{tmoms\_rel\_cm}. It determines the origin from which the
global geometry parameters of the electronic density, radius and other
moments, are computed. Default is \texttt{tmoms\_rel\_cm=.FALSE.}
taking the center of the box as a reference.
	
\subsubsection{Initialization of the ionic background}
\label{sec:ionbackinit}

Information on the ionic background is given at three places: the file
\texttt{for005ion.<name>} provides the ions specifying their
positions, type of chemical element, and instructions for electrons attached
with the ion (only relevant for option \texttt{init\_ao=.TRUE.}),
options for geometrical transformations of the ions in namelist
\texttt{GLOBAL}, and the ionic pseudopotentials in namelist
\texttt{PERIO}. We will work them up now in that order.

\paragraph{The anatomy of the ionic input file  \texttt{for005ion.<name>}\\}
\label{sec:for005ion}
				
The file \texttt{for005ion.<name>}, containing the basic information
about ions, is read provided we have not chosen the option for jellium
background, namely \texttt{nion2=0}.  Each line in
\texttt{for005ion.<name>} corresponds to each ion and is composed of
several fields as follows:
\begin{align*}
  \underbrace{\tikz[baseline]{\node[fill=blue!20,anchor=base]{$x_I\:\:\: y_I\:\:\: z_I$};}}_{\circled{1}}\quad
  \underbrace{\tikz[baseline]{\node[fill=red!20,anchor=base]{$Z_I$};}}_{\circled{2}}\quad
  \underbrace{\tikz[baseline]{\node[fill=green!20,anchor=base]{\texttt{xyz}};}}_{\circled{3}}\quad
  \underbrace{\tikz[baseline]{\node[fill=yellow!20,anchor=base]{$r_I$};}}_{\circled{4}}\quad
  \underbrace{\tikz[baseline]{\node[fill=magenta!20,anchor=base]{$\mathcal{S}_I$};}}_{\circled{5}}
\end{align*}
\begin{itemize}
\item[\circled{1}] the $x,y,z$ coordinates of ion $I$
\item[\circled{2}] the atomic number $Z_I$ of the chemical element in the periodic table of ion $I$
\item[\circled{3}] defines the ordering in which the harmonic
  oscillator states are filled in localized initialization in case of
  \texttt{init\_ao=.TRUE.}
\item[\circled{4}] the radius of the initial Gaussian around ion $I$
  in case of \texttt{init\_ao=.TRUE.}
\item[\circled{5}] the first type of spin, $\uparrow\equiv{\tt 1}$
 or $\downarrow\equiv$\,{\tt -1}, of ion $I$
  (needed only if \texttt{init\_ao=.TRUE.}), see Sec.
    \ref{sec:initwf}
\end{itemize}
		
We give below as an example the content of a \texttt{for005ion.H2O}
file describing a H$_2$O molecule:
\\[4pt]
\tikz[baseline]{\node[fill=blue!20,anchor=base]
			{\texttt{~0.80400~~~0.00000~~0.00000}};
		}
\tikz[baseline]{\node[fill=red!20,anchor=base]
                        {\texttt{8}};
		}
\tikz[baseline]{\node[fill=green!20,anchor=base]
			{\texttt{zxy}};
		}
\tikz[baseline]{\node[fill=yellow!20,anchor=base]
			{\texttt{1.0}};
		}
\tikz[baseline]{\node[fill=magenta!20,anchor=base]
			{\texttt{1}};
		}

\tikz[baseline]{\node[fill=blue!20,anchor=base]
			{\texttt{-0.41628~~~1.45009~~0.00000}};
		}
\tikz[baseline]{\node[fill=red!20,anchor=base]
			{\texttt{1}};
		}
\tikz[baseline]{\node[fill=green!20,anchor=base]
			{\texttt{zxy}};
		}
\tikz[baseline]{\node[fill=yellow!20,anchor=base]
			{\texttt{1.0}};
		}
\tikz[baseline]{\node[fill=magenta!20,anchor=base]
			{\texttt{1}};
		}
		
\tikz[baseline]{\node[fill=blue!20,anchor=base]
  {\texttt{-0.41628~~-1.45009~~0.00000}}; }
\tikz[baseline]{\node[fill=red!20,anchor=base] {\texttt{1}}; }
\tikz[baseline]{\node[fill=green!20,anchor=base] {\texttt{zxy}}; }
\tikz[baseline]{\node[fill=yellow!20,anchor=base] {\texttt{1.0}}; }
\tikz[baseline]{\node[fill=magenta!20,anchor=base] {\texttt{1}}; }
\\[4pt] We remind that $Z_I$ is the element
number. The number of active (valence) electrons which we treat
explicitly for a given atom is usually smaller. For instance, in the
case of the O atom, we have only 6 active electrons while the element
number is $Z_{\rm O}=8$.

As indicated above, the information from
  \texttt{for005ion.<name>} is used depending on the chosen way of
  initialization.  Only the first four columns ($x$, $y$, $z$ values
  and element number $Z_I$) are used for \texttt{init\_ao=.FALSE.}, the
  initialization by global oscillator wavefunctions as explained in
  section \ref{sec:initwf}. The further columns are used for
  \texttt{init\_ao=.TRUE.}, triggering localized initialization of atomic
  orbitals. We detail the filling of the local oscillator shells on
that example using the notation $(n_x,n_y,n_z,\sigma)$ for an
oscillator state with spin $\sigma$: The first line stands for the O
ion which gathers 6 valence electrons whose states are filled in the
order $(0,0,0,\uparrow)$, $(0,0,0,\downarrow)$, $(0,0,1,\uparrow)$,
$(0,0,1,\downarrow)$, $(1,0,0,\uparrow)$, $(1,0,0,\downarrow)$.
%, $(0,0,1,\uparrow)$, $(0,0,1,\downarrow)$. 
The second line stands for one H atom occupied with an electron in
state $(0,0,0,\uparrow)$ and the third line associates
$(0,0,0,\downarrow)$ to the other H atom. The spin label (last entry)
is unimportant here because the remaining spin is chosen to guarantee
the wanted number of spins (parameter \texttt{nspdw}).  If one uses
the option \texttt{init\_ao=.TRUE.}, the last three columns are read.
A word is in order here concerning the column that sets the ordering
in which the harmonic oscillator states are filled. Naively, one would
fill in alphabetical order $x$, $y$ and $z$. In the case of the H$_2$O
molecule in the actual ionc configuration, there is a reason to
initialize the modes at the O site in the order $z$ excitation first
and the $x$ excitation. It is done to avoid time consuming
rearrangements of the node structure of the wavefunctions, as
exemplified in section \ref{sec:initelec}. The positions of the two
$H$ atoms extends most in $y$ direction which means that an initial
expansion in $y$ direction is automatically achieved by the two
wavefunctions localized at the $H$ sites whereas we still need
wavefunctions extending in $z$- and $x$-direction to cover all three
directions. This is exactly achieved by the chosen order \texttt{zxy}
in \texttt{for005ion.H2O}.

\paragraph{Applying an initial global transformation on the ionic configuration\\}
\label{eq:iontransform}

The ionic background is given in {\tt for005ion.<name>} as described
above. One often wants to give the molecule a different position
and/or orientation. This can be done by changing the entries in
\texttt{for005ion.<name>} ``by hand''. But that is painful. The code
offers the more elegant option to perform global operations on ions by
a few input parameters as explained in table
\ref{tab:input-params-global-initions}. 

\begin{table}[htbp]
\caption{Parameters for initial ionic transformations in namelist \texttt{GLOBAL} of \texttt{for005.<name>}}\label{tab:input-params-global-initions}
\begin{tabular}{|p{2.3cm}|p{2.25cm}|p{0.85cm}|p{8.0cm}|}
\hline
\multicolumn{4}{|c|}{\textit{\color{activeColor}More on ionic background}}\\
\hline
\texttt{rotionx}&${\omega}_{\mathrm{rot},x}$ [$^\circ$]&\tgreen{0.0}&\\
\texttt{rotiony}&${\omega}_{\mathrm{rot},y}$ [$^\circ$]&\tgreen{0.0}&
                   Rotate ionic background, see Eq. (\ref{eq:rotateions})\\
\texttt{rotionz}&${\omega}_{\mathrm{rot},z}$ [$^\circ$]&\tgreen{0.0}& \\
\hline
\texttt{scaleionx} &$\eta_{\mathrm{scale},x}$&\tgreen{1.0}& \\
\texttt{scaleiony} &$\eta_{\mathrm{scale},y}$&\tgreen{1.0}&
   Scale ionic configuration, Eq. (\ref{eq:scaleions})\\
\texttt{scaleionz} &$\eta_{\mathrm{scale},z}$&\tgreen{1.0}& \\
\hline
\texttt{scaleion} &$\eta_\mathrm{scale}$&\tgreen{1.0}&Same scaling in each
direction, equivalent to $\eta_{\mathrm{scale},x}=\eta_{\mathrm{scale},y}=\eta_{\mathrm{scale},z}=\eta_\mathrm{scale}$\\
\hline
\texttt{shiftionx} &${R}_{\mathrm{shift},x}$[a$_0$]&\tgreen{0.0}& \\
\texttt{shiftiony} &${R}_{\mathrm{shift},y}$[a$_0$]&\tgreen{0.0}&
     Shift ionic configuration, Eq. (\ref{eq:shiftions})\\
\texttt{shiftionz} &${R}_{\mathrm{shift},z}$[a$_0$]&\tgreen{0.0}& \\
\hline
\hline
\texttt{endcon} &&\tgreen{1D-5}& Jellium initialization parameter, see \scn{sec:jelliter} \\
\hline
\texttt{itback} &&\tgreen{200}& Jellium initialization parameter, see \scn{sec:jelliter} \\  
\hline\hline
\texttt{dpolx}&$E_{0,x}^\mathrm{(stat)}$[Ry/a$_0$]&\tgreen{0.0}& Additional static $x$-dipole, see \scn{cpc:sec:comppol}\\
\hline
\texttt{dpoly}&$E_{0,y}^\mathrm{(stat)}$[Ry/a$_0$]&\tgreen{0.0}& Additional static $y$-dipole, see \scn{cpc:sec:comppol}\\
\hline
\texttt{dpolz}&$E_{0,z}^\mathrm{(stat)}$[Ry/a$_0$]&\tgreen{0.0}& Additional static $z$-dipole, see \scn{cpc:sec:comppol}\\
\hline
\end{tabular}
\end{table}
%\end{longtable}

The simplest one is a shift
\begin{subequations}
\begin{equation}
  \mathbf{R}_I\longrightarrow\mathbf{R}_I+\mathbf{R}_\mathrm{shift}
  \;.
\label{eq:shiftions}
\end{equation}
More involved is rotation of the ionic background. The rotation is
performed about the center of mass $\mathbf{R}_\mathrm{cm}$ of the molecule
which renders rotation to be a three-step process
\begin{equation}
  \mathbf{R}_I\rightarrow\mathbf{R}_I-\mathbf{R}_\mathrm{cm}
  \;,\;
  \mathbf{R}_I\rightarrow\hat{D}(\bm{\omega}_\mathrm{rot})\mathbf{R}_I
  \;,\;
  \mathbf{R}_I\rightarrow\mathbf{R}_I+\mathbf{R}_\mathrm{cm}
  \;,
\label{eq:rotateions}
\end{equation}
where $\hat{D}(\bm{\omega}_\mathrm{rot})$ is the 3$\times$3 matrix of
rotation by angle $|\bm{\omega}_\mathrm{rot}|$ about the axis
$\bm{\omega}_\mathrm{rot}/|\bm{\omega}_\mathrm{rot}|$. 
Finally, there is the option to scale the ionic coordinates about the
ionic center in each direction  separately as
\begin{equation}
  {R}_{I,i}
  \longrightarrow
  \eta_{\mathrm{scale},i}\left({R}_{I,i}-{R}_{\mathrm{cm},i}\right)
  +{R}_{\mathrm{cm},i}
  \;,\;
  i\in\{x,y,z\}
  \;.
\label{eq:scaleions}
\end{equation}
\end{subequations}

These ionic transformations can be used for different purposes.  One
option is to perform several ground state computations from the same
input file \texttt{for005ion.<name>}, but with different spatial
orientations or scaling.  One can also use scaling of the background
for instantaneous excitation of electronic breathing modes.  To that
end, one uses a scaled configuration for the static calculation, saves
that to file, and restarts dynamics with the standard (unscaled) ionic
ground-state configurations, reading the scaled electron cloud from
\texttt{RSAVE.<name>} (with invoking \texttt{tstat=.TRUE.}, see
table~\ref{tab:static-output-files}).  One could use the same strategy
for rotational excitation. But the parameter \texttt{irotate}, see
table \ref{tab:dyn-input-params-instantaneous}, is the more convenient
tool for that purpose.


\paragraph{Static dipole field and computation of polarizability\\}
\label{sec:static-dipole}

The parameters \texttt{dpolx}, \texttt{dpoly} and \texttt{dpolz} add a
static electric dipole field to the Kohn-Sham potential.  As explained
in \scn{cpc:sec:comppol} of the reference paper, this is used to
extract the static polarizability of the system under study.  More
details on the corresponding output can be found in
\scn{sec:results-pstat}



\subsubsection{Initialization of the jellium background}
\label{sec:jelliter}

It is a long standing experience that the electronic structure of
metals can be well approximated through replacing the ionic background
by a smooth positive background charge \cite{GV05}. Thus the jellium
model has been widely used in the physics of metal clusters
\cite{Bra93} and  QDD contains this as an option by choosing
\texttt{nion2=0} in namelist \texttt{PERIO} where also the jellium
parameters can be found. Nonetheless, there remains the
initialization.  The jellium density (\ref{cpc:eq:rhojell}) is not
exactly fulfilling the condition (\ref{cpc:eq:R0cond}) due to grid
representation, soft surface width, and deformation. We have to
renormalize $\varrho_{\mathrm{jel},0}$ slightly to match condition
(\ref{cpc:eq:R0cond}). This is done by iteration of the jellium radius
$R_\mathrm{jel}$. This is what the parameters \texttt{itback} and
\texttt{endcon} in namelist \texttt{GLOBAL}, see table \ref{tab:input-params-global-initions}, are for.


\subsubsection{External ionic potential} 

The final option \texttt{nion2=2} allows one to initialize the ionic
background as an externally given potential. This is achieved by
copying the content of the input file \texttt{potion.dat} onto the
array \texttt{potion} (that contains the ionic potential). This can
correspond for instance to a Woods-Saxon or a Clemenger-Nilsson
potential, or whatever external potential, provided that
\texttt{potion.dat} contains its discretization on the numerical grid.
This is clearly an advanced option because properly mapping a
potential onto the grid can become cumbersome.

\subsection{PERIO namelist}
\label{sec:perio}
		
%\begin{longtable}{|p{2.5cm}|p{1.0cm}|p{10.0cm}|}
\begin{table}
  \caption{Parameters in namelist \texttt{PERIO} of
    \texttt{for005.<name>}, either for a jellium background or with
    explicit ions described by a Goedecker-like pseudopotential. The
    comment {\em some preset} indicates that the code contains a list
    of default values for Goedecker and Gaussian pseudopotentials for
    a couple elements, check subroutine \texttt{iperio} in
    \texttt{init.F90} which ones are preset, see paragraph \ref{sec:explions}.
  \label{tab:input-params-PsP}}
%  \endfirsthead
\begin{tabular}{|p{2.1cm}|p{2.1cm}|p{0.55cm}|p{8.55cm}|}
  \hline \multicolumn{4}{|c|}{\texttt{PERIO} namelist}\\ 
  \hline
  \texttt{nion2} &&\tgreen{1}& Selects type of ionic background \\ 
  &&& 0  $\rightarrow$ jellium background \\
  &&& 1 $\rightarrow$ background  from ionic pseudo-potentials \\
  &&& 2 $\rightarrow$ background read in from \texttt{potion.dat}\\ 
  \hline \hline
  \multicolumn{4}{|c|}{\textit{\color{activeColor}Jellium background
      parameters, Eq. (\ref{cpc:eq:softJ})}}\\ 
  \hline 
  \texttt{radjel}&$r_s$ [a$_0$]&\tgreen{4.0}& Wigner-Seitz radius \\ 
  \hline
  \texttt{surjel}&$\sigma_\mathrm{jel}$ [a$_0$]&\tgreen{1.0}& Surface thickness \\ 
  \hline
  \texttt{bbeta}&$\beta_\mathrm{jel}$&\tgreen{0.0}& Quadrupole deformation \\ 
  \hline
  \texttt{gamma}&$\gamma_\mathrm{jel}$&\tgreen{0.0}& Triaxiality \\ 
  \hline
  \texttt{beta4}&$\alpha_{40,\mathrm{jel}}$&\tgreen{0.0}& Hexadecapole deformation\\ 
  \hline
  \multicolumn{4}{|c|}{\textit{\color{activeColor}Choice of
      pseudopotential, see \scn{cpc:sec:practPsP}}}\\ 
  \hline
  \multicolumn{4}{|c|}{\textit{\color{activeColor}Parameters of the
      Goedecker-like pseudopotentials Eq. (\ref{cpc:eq:Goed}) (many preset)}}\\
   \hline
  \texttt{amu(Z)}&$M_\mathrm{ion}$&& Atomic mass of chemical element
    \texttt{Z}\\
  \hline
  \texttt{ch(Z)}& $Z$ && Atomic number of chemical element \texttt{Z}\\
  \hline
  \texttt{cc<k>(Z)}&$C_k$ [Ha]&&Local expansion coefficients $C_k$ in powers of
     $\left(r/r_\mathrm{loc}\right)^{2k-2}$ for chemical element
    \texttt{Z}, where (\texttt{k}, $k)\in\{1,2,3,4\}$\\
  \hline
  \texttt{crloc(Z)}& $r_\mathrm{loc}$ [a$_0$] &&Local radius $r_\mathrm{loc}$
     of chemical element \texttt{Z}, in $a_0$\\
  \hline
  \texttt{r0g(Z)}& $r_s$ [a$_0$]&& Non-local radius $r_s$ for $l=0$ of element
      \texttt{Z}\\
  \hline 
  \texttt{r1g(Z)}& $r_p$ [a$_0$]&& Non-local radius $r_p$ for $l=1$ of
    element \texttt{Z}\\
  \hline
  \texttt{h0\_11g(Z)}& $h^s_{11}$ [Ha]&&Non-local radial projector
    coefficient\\
  \hline
  \texttt{h1\_11g(Z)}& $h^p_{11}$ [Ha]&& Non-local radial projector
    coefficient\\
  \hline 
  \texttt{h0\_22g(Z)}& $h^s_{22}$ [Ha]&&Non-local radial projector
    coefficient\\ 
  \hline
  \texttt{radiong(Z)}& [a$_0$]&& radius  of the sphere within which the
    pseudopotential is evaluated\\
  \hline
  \multicolumn{4}{|c|}{\textit{\color{activeColor}Parameters of the
      soft local pseudopotentials Eq. (\ref{cpc:eq:softlocal}) (some preset)}}\\
  \hline 
  \texttt{dr1}& $\sigma_1\sqrt{2\mathrm{ln}2}$ [a$_0$]&& First Gaussian width parameter\\ 
  \hline 
  \texttt{dr1}& $\sigma_2\sqrt{2\mathrm{ln}2}$ [a$_0$]&& Second Gaussian width parameter\\ 
  \hline 
  \texttt{prho1}& $\frac{c_1}{(\sigma_2\sqrt{2\mathrm{ln}2})^3}$&&
  First strength parameter\\ 
  \hline 
  \texttt{prho2}& $\frac{c_2}{(\sigma_2\sqrt{2\mathrm{ln}2})^3}$&&Second strength parameter\\ 
%  \hline
%  \texttt{tinterpol} &&\tgreen{.F.}& Switch to interpolation of pseudopotential on the grid \\
  \hline
\end{tabular}
\end{table}
%\end{longtable}

The parameters of the interaction of the ions with electrons are
provided in the namelist \texttt{PERIO}. They are collected here in
table \ref{tab:input-params-PsP}.  As already discussed in the
previous sections, there are three ways to deal with the ionic
background. The choice is governed by the parameter \texttt{nion2}.


\paragraph{Jellium background}
%
Setting \texttt{nion2=0} switches to soft jellium background
whose parameters in table \ref{tab:input-params-PsP} can be easily
related to Eq. (\ref{cpc:eq:rhojell}) in the reference paper.
An example for jellium input is:
\begin{Verbatim}[frame=single,label=Example of Jellium parameters for Na$_8$,fontsize=\footnotesize]
&PERIO
	nion2=0,
	radjel=3.8, bbeta=0.0, gamma=0, surjel=0.9,
	beta4=0.0,
&END
\end{Verbatim}
The full example input file together with output files can be found
in:
\begin{verbatim}
$QDD_ROOT/examples/ground-state/Na2-jel-egs/
\end{verbatim}


\newpage
\paragraph{Explicit ions}
\label{sec:explions}
%
Detailed ionic pseudopotentials (PsP), triggered by \texttt{nion2>0},
require many more parameters.  Again, the respective parameters in
table \ref{tab:input-params-PsP} can easily be related to Sec.
\ref{cpc:sec:practPsP} of the reference paper.

		
The case \texttt{ipsptyp=0} uses soft local PsP, see
Eq. (\ref{cpc:eq:softlocal}), whose parameters are explained in the last
four rows of table \ref{tab:input-params-PsP}. They need rarely to be
entered explicitly as those elements for which we had developed such
pseudopotentials, namely H, Na, Mg, Ar, K, and Ce, are already
implemented with appropriate default values \cite{Kue99}.

		
The case \texttt{ipsptyp=1} uses Goedecker-like PsP. The corresponding
parameters are listed in \tab{tab:input-params-PsP}. Default values
for some chemical elements, taken from \cite{PhysRevB.54.1703}, are
already implemented, namely H, He, B, C, N, O, F, Ne, Na, Mg, Al, Si,
P, S, Ar, Ca, Cu, and Ag. One can also use one's own set of PsP parameters,
see e.g. explicit examples in the directory
\texttt{\$QDD\_ROOT/examples/}. One of them is shown here:
\vskip 0.5cm
\begin{Verbatim}[frame=single,label=Example of pseudopotential parameters for H$_2$O,fontsize=\footnotesize]
&PERIO
	nion2=1,
	ipsptyp=1,
	cc1(1)   	=  0.28897655,
	cc2(1)   	= -0.08386485,
	crloc(1) 	=  0.5,
	r0g(1)    	=  0.5,
	r1g(1)     =  0.5,
	radiong(1) =  2.2,
	h0_11g(1)	= -0.49941755, 
	h1_11g(1) 	=  0.0,
	
	cc1(8)   	=  0.7345366,
	cc2(8)   	= -2.050880,
	crloc(8) 	=  0.5,
	r0g(8)		=  0.5,
	r1g(8)		=  0.5,
	radiong(8)	=  2.2,
	h0_11g(8)	=  1.750506,
	h1_11g(8) 	=  0.0,
&END
\end{Verbatim}

The case \texttt{ipsptyp=2} uses only the local part of Goedecker-like
pseudopotentials. Currently, defaults for H, Na and Ar are implemented,
and again, one can enter one's own values through the namelist \texttt{PERIO}.

Finally, there are two more parameters of more technical nature.  One is
\texttt{radiong(Z)}. Mind that pseudopotentials are of short range. It
does not make much sense to evaluate them over the whole numerical
box. We confine the evaluation to a sphere of radius
\texttt{radiong(Z)} around the concerned ion.  A good choice is five
times the largest radius in the corresponding pseudopotential.  


\clearpage
	
\section{I/O structure of a ground state calculation}
\label{sec:stat_basics}

After having defined the system, grid, and initial state we continue
with a static or dynamic calculation, or one after the other. This
requires further specific input parameters. This section discusses the
input parameters for a static calculation to obtain the ground state
of the studied system.



\subsection{Input parameters for a static calculation}
	

%\begin{longtable}{|p{3.5cm}|p{11.2cm}|}
\begin{table}[bht]
  \caption{Numerical and physical parameters for a static calculation
    in namelist \texttt{STATIC} of \texttt{for005.<name>}
\label{tab:input-params-static}}
%  \endfirsthead
\begin{tabular}{|p{2.4cm}|p{1.1cm}|p{1.0cm}|p{9.0cm}|}
\hline
\multicolumn{4}{|c|}{namelist \texttt{DYNAMIC}}\\
\hline
\multicolumn{4}{|c|}{\textit{\color{activeColor}Convergence parameters
    for a static calculation}}\\
\hline
\texttt{epswf}&$\delta_\mathrm{damp}$&\tgreen{0.2}&Step size for KS static solution, see Eq. (\ref{cpc:eq:dampstep}) \\
\hline
\texttt{e0dmp}& $E_{0,\mathrm{damp}}$&\tgreen{2.0}& Damping parameter
for KS static solution,\\
 &\multicolumn{1}{|r|}{[Ry]} && see Eq. (\ref{cpc:eq:dampstep}) \\
\hline
\texttt{epsoro}&&\tgreen{1D-8}& Termination criterion for static iterations \\
\hline
\texttt{occmix} &$\eta_\mathrm{occ}$&\tgreen{0.5}&Mixing of occupation numbers in static iterations, see Eq.~(\ref{cpc:eq:occmix}) \\
  \hline
  \texttt{variance\_gain} &$\eta_{var,max}$ &\tgreen{1/3}& Required
  relative gain in
  s.p. variance to activate diagonalization (see \texttt{ifhamdiag}) 
  \\
  \hline
  \texttt{ismax}&&\tgreen{1000}& Maximum number of static iterations \\
  \hline
  \texttt{idyniter}&&\tgreen{0}& Switch to s.p. energy as \texttt{e0dmp} for \texttt{iter}$>$\texttt{idyniter} \\
  \hline
  \texttt{ifhamdiag} &&\tgreen{0}& Frequency of diagonalization of mean-field Hamiltonian \\
  \hline
  \multicolumn{4}{|c|}{\textit{\color{activeColor} Output parameters for
      static iterations}} \\
  \hline
  \texttt{tstat}&&\tgreen{F}& To read static wave functions from \texttt{RSAVE.<name>}\\
		&&& F $\rightarrow$ starts a static calculation from scratch \\
		&&& T $\rightarrow$ continues iterations from \texttt{RSAVE.<name>}\\
  \hline
  \texttt{isaves}&&\tgreen{0}& Frequency to save static state in \texttt{RSAVE.<name>} \\
  \hline
  \texttt{istinf}&&\tgreen{10}& Frequency for printing information during static calculation \\
  \hline			
  \texttt{ifspemoms}&&\tgreen{F}& Switch to compute and write s.p.
  spatial moments to final \texttt{pstat.<name>} \\
  \hline
  \texttt{iftransme}&&\tgreen{F}& Switch to compute and write dipole transition
  matrix elements to file \texttt{mte\_xyz} \\
  \hline
  \texttt{tplotorbitals} &&\tgreen{F}& Switch to plot the orbitals at the end of the statics  \\
  \hline
  \texttt{iflocaliz}&&\tgreen{F}& Switch to compute electron localization
  function and 
     to write it to \texttt{pelfstat*.<name>} \\
  \hline
\end{tabular}
\end{table}
%\end{longtable}

Table \ref{tab:input-params-static} summarizes the parameters
governing the solution  of the static Kohn-Sham (KS) equations as described
in \scn{cpc:sec:numstaticel} of the reference paper.

\paragraph{Static iteration parameters\\}

The first block sets numerical parameters for the static iteration
step (\ref{cpc:eq:dampstep}).  The choice of the damping energy
$E_\mathrm{damp}$ has some impact on the speed of convergence. It
should be of order of the depth of the mean-field potential and stay
$\leq|\varepsilon_1|$, the binding energy of the lowest
s.p. state. Standard is to run it with an appropriately chosen fixed
value of $E_\mathrm{damp}$, for inspiration see the test cases in the
subdirectory {\tt examples} of this package. However, the code offers
the option of a dynamic regulation by setting
\texttt{idyniter>0}. This activates the automatic setting
$E_\mathrm{damp}=|\varepsilon_1|$ after \texttt{idyniter} initial
iterations with the initially given fixed $E_\mathrm{damp}$. This
option provides the optimal choice of $E_\mathrm{damp}$. However, it
requires observation as it may over-regulate in rare cases thus
leading to delays or unstable iterations. This is why we always start
for a few iterations with fixed $E_\mathrm{damp}$.

The overall step size $\delta_\mathrm{damp}$ can be of order 0.5 if an
appropriate $E_\mathrm{damp}$ is chosen. Slightly lower values help to
avoid unstable regions at the price of requiring a few more
iterations. A straightforward gradient step is switched by
$E_\mathrm{damp}=0$ in which case one should choose
$\delta_\mathrm{damp}<E_\mathrm{max}^{-1}$ with $E_\mathrm{max}$
being the maximal representable energy on the grid
\cite{Rei82a,Blu92}.


% ifhamdiag
A further option to speed up static iterations is to diagonalize the
matrix $\langle\varphi_\alpha|\hat{h}|\varphi_\beta\rangle$ of the
mean-field Hamiltonian within the space of given s.p. states. This is
particularly efficient if $\Omega>N_\mathrm{el}$ (Eq. (\ref{cpc:eq:gen_nbrs})), i.e. the space of
s.p. states is larger than the actual number of electrons as, e.g., in
calculations at finite temperature or in RTA. This option is set by
the input parameter \texttt{ifhamdiag}. It determines the frequency of
explicit diagonalization steps, i.e., diagonalization of
$\langle\varphi_\alpha|\hat{h}|\varphi_\beta\rangle)$ is invoked if
\texttt{MOD(iter,ifhamdiag)=0}. It suffices to use values of order 10.
Diagonalization has a further control parameter
\texttt{variance\_gain}$=\eta_{var,max}$. Before running
diagonalization, it is checked whether the gain factor in the variance
of s.p. energies came out below $\eta_{var,max}$. If not,
diagonalization is over-ridden.


Important control parameters are \texttt{isaves} and \texttt{tstat}
They regulate saving and recycling a full static electron
configuration (all s.p. wave functions together with their occupation
weights and grid parameters). If \texttt{isaves}$>0$ the state is
always saved at the end of iteration and additionally every
\texttt{isaves} times in between. The \texttt{tstat} triggers reading
of the saved static configurations. This is useful, e.g., to continue
a static iteration if one wants to improve the solution. It is mostly
used to start a dynamical run from the static configuration without
the need to recompute the ground state from scratch.
For more details on the use of \texttt{tstat}, go to
\scn{sec:basicDYN}
%\scn{sec:resuming}. 
%\MDfoot{pb in label of section}

\paragraph{Observables during and after a static calculation\\}
\label{sec:statobs}

The final values of the most relevant observables (energy,
s.p. energies, radius, \ldots)  are always printed on the file
\texttt{pstat.<name>}. The parameter \texttt{ifspemoms} triggers
adding more detailed information as, e.g., the spatial moments of each
s.p. to \texttt{pstat.<name>}. With the parameter \texttt{iftransme} one
can order computation of all one-particle-one-hole ($1ph$)
configurations and their dipole transition elements which are printed
finally on a separate file \texttt{mte\_xyz}. This makes sense only if
a sufficient amount of unoccupied s.p. states was computed and then it
helps to get a first overview of the structure of dipole
excitation. Thus far, the switches determine final output.  One can
also trigger output of during calculation with the frequency parameter
\texttt{istinf}. This activates printing of key observables on
\texttt{infosp.<name>} and more details on
\texttt{out\_details.<name>}.

The switch \texttt{tplotorbitals} triggers output of the full final
s.p. wave functions to file \texttt{pOrbitals.<name>}. This is, of
course, space eating and can be used only for small systems. On the
other hand, it can be very instructive to visualize the detailed
wave functions once in a while, particularly when trying to deal with a
new molecule.

A very special and interesting observable is the electron localization
function which serves as an indicator of region where one particular
electron state dominates the local density. It was first proposed in
\cite{Bec90}, extended to dynamical simulations in \cite{Bur05} and
also to nuclear TDHF calculations in \cite{Rei11d}. For electrons, the
localization function is defined as
\begin{equation}
\label{eq:local}
  {\cal C}_{\sigma}(\mathbf{r})
  =
  \left[
  1
  +
  \left(
  \frac{\tau_{\sigma}\varrho_{\sigma}-\frac{1}{4}[\nabla\varrho_{\sigma}]^2-{\bf j}_{\sigma}^2}
       {\varrho_{\sigma}\tau_{\sigma}^\mathrm{TF}}
  \right)^2
  \right]^{-1}
%\\
  \;,\;
  \tau_{\sigma}^\mathrm{TF}
  =
  \frac{3}{5}\left(6\pi^2\right)^{2/3}\varrho_{\sigma}^{5/3}
  \;,
\end{equation}
where $\tau_{\sigma}^\mathrm{TF}$ is the Thomas-Fermi approximation to
the kinetic energy density, $\varrho_\sigma$ the local density of
spin $\sigma$ electrons as defined in Eq. (\ref{cpc:eq:locdens}),
$\tau_\sigma$ the corresponding kinetic energy density, and ${\bf
  j}_{\sigma}$ the current density. Note that the current density
${\bf j}_{\sigma}$ vanishes in the static case. For further details
see \cite {Bec90,Bur05,Rei11d}.


\subsection{Output files of a static calculation}

During a calculation, output files are generated and stored in the same directory as where the \texttt{qdd} binary is executed. For a ground state calculation, the output files are listed in \tab{tab:static-output-files}. The physical information stored in these output files will be discussed in an example in \scn{sec:stat_examples}.

%\begin{longtable}{|p{3.7cm}|p{11cm}|}
\begin{table}[htbp]
  \caption{Output files generated during a static
    calculation\label{tab:static-output-files}}
 \begin{tabular}{|p{3.7cm}|p{10.5cm}|}
%\endfirsthead
  \hline
  \texttt{poptions.<name>}& Overview of basic parameters
  and chosen options on solvers, compiler options, etc.\\
  \hline
  \texttt{out\_detail.<name>}& Protocol file that contains
  similar detailed information as screen output\\
  \hline
  \texttt{sconver.<name>} & Contains protocol information on static convergence\\
  \hline
  \texttt{sspenergies.<name>} & Contains protocol information on s.p. energies\\
  \hline
  \texttt{sspoccup.<name>} & Contains protocol information on s.p. occupation numbers\\
  \hline
  \texttt{sspvariances.<name>} & Contains protocol information on s.p. variances\\
  \hline
  \texttt{infosp.<name>}& Energy and variances at given iteration steps determined by the variable \texttt{istinfo} in the \texttt{DYNAMIC} namelist\\
  \hline
  \texttt{pstat.<name>}& Contains the final information about the s.p. energies, spins, variances, occupation numbers, monopole-, dipole- and quadrupole moments, etc.\\
  \hline
  \texttt{RSAVE.<name>} & Saves the full electronic and ionic
  configuration of a static calculation\\
  \hline
\end{tabular}
\end{table}
%\end{longtable}

		
\clearpage

\section{Example of a static calculation}	
\label{sec:stat_examples}

We present here a simple example, that is the ground state of the
sodium dimer.  First we will go through the output files and explain
their content and use. As more elaborate examples, we shows how to
extract from these calculations ionization potential (IP) and
HOMO–LUMO gap of Na$_2$. Finally, a series of static calculations can
also be used to find the most stable ionic structure for dimer
molecules by manually varying the distance between the ions,
i.e. mapping the Born-Oppenheimer surface.
			
\subsection{Input and output files for Na$_2$}

The input files for the ground state of the sodium dimer can be found in:
\begin{verbatim}
$QDD_ROOT/examples/user_manual/ground-state/Na2/ground-state
\end{verbatim}
As discussed in \scn{sec:launch}, there are 3 input files, namely
\texttt{for005}, \texttt{for005.Na2} and \texttt{for005ion.Na2}. One
can either copy these files to any desired location or run the
\texttt{qdd} executable directly inside this directory, by
executing:
\begin{verbatim}
$ cd "your working directory"
$ QDD_ROOT/bin/qdd > terminal.out 2> messages.out
\end{verbatim}
		
		
\bigskip
		
\begin{table}[htbp]
\caption{\label{tab:output-example}Typical file structure after a
  static calculation.}
\begin{Verbatim}[frame=single,label=Example directory listing]
-rw-rw-r-- 1 mpt218 mpt218      10 Aug 17 21:37 for005
-rw-rw-r-- 1 mpt218 mpt218     102 Aug 17 21:37 for005ion.Na2-egs
-rw-rw-r-- 1 mpt218 mpt218     457 Sep 15 08:28 for005.Na2-egs
-rw-rw-r-- 1 mpt218 mpt218   32985 Sep 15 08:37 out_detail.0.Na2-egs
-rw-rw-r-- 1 mpt218 mpt218    4590 Sep 15 08:37 infosp.Na2-egs
-rw-rw-r-- 1 mpt218 mpt218     153 Sep 15 08:37 messages.out
-rw-rw-r-- 1 mpt218 mpt218    1241 Sep 15 08:36 poptions.Na2-egs
-rw-rw-r-- 1 mpt218 mpt218    2098 Sep 15 08:37 pstat.Na2-egs
-rw-rw-r-- 1 mpt218 mpt218 4194848 Sep 15 08:37 RSAVE.Na2-egs
-rw-rw-r-- 1 mpt218 mpt218    2372 Sep 15 08:37 sconver.Na2-egs
-rw-rw-r-- 1 mpt218 mpt218     617 Sep 15 08:37 sspenergies.Na2-egs
-rw-rw-r-- 1 mpt218 mpt218     623 Sep 15 08:37 sspoccup.Na2-egs
-rw-rw-r-- 1 mpt218 mpt218     762 Sep 15 08:37 sspvariances.Na2-egs
-rw-rw-r-- 1 mpt218 mpt218   92993 Sep 15 08:37 terminal.out
\end{Verbatim}
\end{table}
This will save the terminal output on the screen to the
\texttt{terminal.out} file, and any error messages that might be
generated during the calculation to the \texttt{messages.out}
file. %Depending on the speed of your machine, after a few minutes, 
When the calculation is over (after a few minutes),
you should have a set of  files as listed in table \ref{tab:output-example}.  The
largest file is \texttt{RSAVE.<name>} that saves the static
configuration (wave functions, occupations, ionic coordinates, basic
parameters) provided \texttt{isaves>0} was set.  For further use of
this file, see \scn{sec:basicDYN}.
%\scn{sec:resuming}.  
The files starting by \texttt{s...} and the file
\texttt{infosp.Na-egs} show evolution of observables along  static
iterations with a frequency as set by \texttt{istinfo}. These are of
interest for those wanting to scrutinize numerical
performance. Observables related to the final solution will be
discussed in the following three subsections.

\subsection{Observables in the pstat.<name> output file}
\label{sec:results-pstat}

\begin{table}
\caption{\label{tab:pstat}An example for a \texttt{pstat.Na2-egs}
  output file from a static calculation. All energies in units of Ry and
  all lengths in units of a$_0$.}  
\begingroup
\renewcommand{\arraystretch}{0.85}
\renewcommand{\baselinestretch}{0.85}
\begin{Verbatim}[frame=single,fontsize=\footnotesize]
final protocol of static for IFSICP=  2
level:  1 spin,occup,ekin,esp,variance =  1  1.00000  0.11487 -0.37147  5.0973E-10
level:  2 spin,occup,ekin,esp,variance =  1  0.00000  0.15073 -0.26170  6.7700E-06
level:  3 spin,occup,ekin,esp,variance =  1  0.00000  0.14652 -0.21593  1.1029E-05
level:  4 spin,occup,ekin,esp,variance =  1  0.00000  0.14652 -0.21593  1.1838E-05
level:  5 spin,occup,ekin,esp,variance = -1  1.00000  0.11487 -0.37147  5.0973E-10
level:  6 spin,occup,ekin,esp,variance = -1  0.00000  0.15073 -0.26170  6.7700E-06
level:  7 spin,occup,ekin,esp,variance = -1  0.00000  0.14652 -0.21593  1.1029E-05
level:  8 spin,occup,ekin,esp,variance = -1  0.00000  0.14652 -0.21593  1.1838E-05
binding energy =   -0.8091395
total variance =  5.0973E-10
sp pot, sp kin, rearr, nonlocal=   -0.97266    0.22973   -0.02291    0.00000
e_coul: i-i , e-i , e-e , total=    0.35255   -1.76436    0.85760   -0.55422
mon.:   2.00
dip.IN :    0.00000    0.00000    0.00000
dip.OUT :    0.00000    0.00000    0.00000
quadrupole moments:
xx,yy,zz:     9.0197     6.2984     6.2984
xy,zx,zy:    -0.0000    -0.0000     0.0000
spindip.:     0.0000     0.0000     0.0000
rms radius:     4.6494, corresponding average density,k_F:  3.67991E-03  0.41278
protocol of s.p. moments:
 state energy x y z variance xx yy zz xy xz yz
   1 -0.371  2.84 -0.00 -0.00  5.45     -0.0   -0.0   -0.0
   2 -0.262  2.84 -0.00 -0.00  7.24      0.0    0.0    0.0
   3 -0.216  2.84  0.00 -0.00  7.40      0.0   -0.0   -0.0
   4 -0.216  2.84 -0.00  0.00  7.40     -0.0    0.0    0.0
   5 -0.371  2.84 -0.00 -0.00  5.45     -0.0   -0.0   -0.0
   6 -0.262  2.84 -0.00 -0.00  7.24      0.0    0.0    0.0
   7 -0.216  2.84  0.00 -0.00  7.40      0.0   -0.0   -0.0
   8 -0.216  2.84 -0.00  0.00  7.40     -0.0    0.0    0.0
  average:   2.84  0.00  0.00  6.92
\end{Verbatim}
\endgroup
\vspace*{4pt}
\begingroup
\renewcommand{\arraystretch}{0.97}
\begin{center}
\begin{tabular}{ll}
\hline
\multicolumn{2}{c}{Energetics}
\\
\hline
\texttt{binding energy}: & binding energy of the system
\\
\texttt{total variance}: & average variance of s.p. energies
\\
\texttt{sp pot}: & sum over all occupied s.p. potential
energies
\\
\texttt{sp kin}: & sum over all occupied s.p. kinetic
energies
\\
\texttt{rearr}: & rearrangement energy
\\
\texttt{nonlocal}: & non-local part of the potential energy
\\
\texttt{i-i}: & ion-ion Coulomb energy
\\
\texttt{e-i}: & electron-ion Coulomb energy
\\
\texttt{e-e}: & electron-electron Coulomb energy
\\
\texttt{total}: & total Coulomb energy (\texttt{i-i} + \texttt{e-i} +
                  \texttt{e-e})
\\
\hline
\multicolumn{2}{c}{Electronic s.p. levels} 
\\
\hline
\texttt{spin}: &  spin of the s.p. level
\\
\texttt{occup}: & occupation number
\\
\texttt{ekin}: & s.p. kinetic energy
\\
\texttt{esp}: & s.p. energy
\\
\texttt{variance} : & variance on the s.p. energy
\\
\hline
\multicolumn{2}{c}{Multipole moments}
\\
\hline
\texttt{mon}: & monopole moment (should equal $M_\mathrm{el}$)
\\
\texttt{dip.in/out}: & dipole moments in $x$, $y$,
and $z$ directions
\\
\texttt{xx,yy,zz}: & quadrupole moments in $xx$, $yy$,
and $zz$ directions
\\
\texttt{xy,zx,zy}: & quadrupole moments in
$xy$, $zx$, and $zy$ directions
\\
\texttt{spindip}: & spin-dipole moments in $x$, $y$,
and $z$ directions
\\
\texttt{rms radius}: & root mean square radius of the electron density
\\
\texttt{average density}: & average electron density
\\
\texttt{k\_F}: & Fermi momentum
\\
\hline
\end{tabular}
\end{center}
\endgroup
\end{table}

The most relevant observables are located in the \texttt{pstat.<name>}
file which is shown in the upper part of table
\ref{tab:pstat}. The lower part of the table explains the
entries. Most entries are self-explaining. Some of them need a few more
words. The \texttt{total variance} is the average s.p. variance
weighted by the occupation numbers, i.e.  
%
$\sqrt{\sum_\alpha  w_\alpha\Delta^2\varepsilon_\alpha/N_\mathrm{el}}$ 
%
where $\Delta^2\varepsilon_\alpha$ is the squared energy variance of
s.p. state $\alpha$.  This quantity is an important counter-check of
convergence of the static iterations. If it turns out to be larger
than wanted, one can restart another round of static iterations from
\texttt{RSAVE.<name>} if it was saved.

The entry \texttt{dip.in} protocols the possible external dipole
field, see entries \texttt{dpolx}, \texttt{dpoly} and \texttt{dpolz} in the namelist {\tt GLOBAL},
see table \ref{tab:input-params-global-initions}, while \texttt{dip.out} covers
the dipole moments emerging from the static calculation.  The average
density and Fermi momentum are deduced from the electronic
r.m.s. radius as
%
$\varrho_\mathrm{average}=3N_\mathrm{el}/(4\pi\sqrt{5/3}r_\mathrm{rms}^3)$
% 
and 
%
$k_\mathrm{F}=(9\pi/4)^{1/3}/r_\mathrm{rms}$.
%
These quantities are motivated from taking the electron cloud as an
homogeneous sphere \cite{Mar10aB} and are useful to determine the
electron-electron cross section for the RTA part.

The file \texttt{pstat.<name>} is cumulative. This means if static
calculations are sequentially launched in the same directory with the
same qualifier \texttt{<name>}, the outputs are appended to
\texttt{pstat.<name>}.
			
\subsection{Ionization potential (IP)} 

There are two ways to read off the IP. The first, and simpler, method
is to take it from \texttt{pstat.<name>}. The IP is the (negative)
s.p. energy of the least bound occupied level known as the Highest
Occupied Molecular Orbital (HOMO). These are here degenerated, namely
levels 1 and 5 yielding an IP of 0.37 Ry. However, one has to be
cautious with that simple approach.  The HOMO energy suffers from the
self-interaction error when working with mere LDA (\texttt{ifsicp=0}).
The strategy is applicable only in combination with a SIC.  In the
\texttt{pstat} discussed above, the first line indicates
\texttt{ifsicp=2}, corresponding to ADSIC, (see
\tab{tab:input-params-global-initwf}). This means that the (absolute
value of the) HOMO energy can be assumed to be close to the IP.

The alternative is to compute the IP as it is defined, namely as the
energy difference between the binding energy (BE) of the given molecule and the one of the once ionized
molecule without changing the ionic configuration (often coined
vertical ionization). We have already the binding energy of Na$_2$
from the static calculation. We need now to compute also the ground
state of Na$_2^+$.  There is an example calculation for the ground
state of Na$^+_2$ in the directory
\texttt{\$QDD\_ROOT/examples/Na2p/ground-state}. It
uses exactly the same input file as that for Na$_2$, but with one less
electron (\texttt{nelect=1}). Once completed, the IP of Na$_2$ is
given as:
\begin{equation*}
\mathrm{IP(Na_2)} = \mathrm{BE(Na_2)} - \mathrm{BE(Na^+_2)} 
= [-0.81~\mathrm{Ry}] - [-0.44~\mathrm{Ry}] 
= 0.37~\mathrm{Ry} \quad.
\end{equation*}
This value is identical to the one of  the IP obtained from the HOMO energy
above which can be taken as a practical illustration that Koopman's
theorem is often well fulfilled for ADSIC \cite{Klu13}. Koopman's
theorem is violated by LDA. But the direct evaluation of the IP from
difference of binding energies still delivers a pertinent result.

			
\subsection{HOMO–LUMO gap} 

This quantity is defined as the difference between the energy of the
Lowest Unoccupied Molecular Orbital (LUMO) and the HOMO.  In the
example here this is the difference between levels 2 and 1, or 6 and 5
respectively.  To get the energy value of the LUMO, one needs to
calculate more electronic states than there are valence electrons (see
\texttt{nelect} parameter in \tab{tab:input-params-sys-choice}).  This
can be controlled by the parameter \texttt{deocc} (see
\tab{tab:input-params-global-initwf}) together with taking care that
the initial deformation (parameters \texttt{b2occ} and
\texttt{gamocc}, see \tab{tab:input-params-global-initwf}) is somehow
close to the distribution of the ionic background.  The fine tuning of
these parameters calls for some experience. We give the following
strategy as a rough guide:
\begin{enumerate}
\item\label{it:setkstate} Set \texttt{kstate} to a value that is at
  least double the number of electrons.
\item Increase \texttt{deocc} until the static calculation uses 
about twice as much states as occupied ones. 
\item It may happen that the calculation fails to start and exits with
  the message: \texttt{deocc or part.number too large for given
    ksttot}. In that case, enhance  \texttt{kstate} and
continue with step \ref{it:setkstate}.
\end{enumerate}
In the present case, the HOMO-LUMO gap  of Na$_2$ reads:
\begin{equation*}
\mathrm{gap(Na_2)} = \mathrm{LUMO(Na_2)} - \mathrm{HOMO(Na_2)} 
= [-0.26170~\mathrm{Ry}] - [-0.37147~\mathrm{Ry}] 
= 0.10977~\mathrm{Ry} \quad.
\end{equation*}
			
\subsection{Finding the ionic ground state of Na$_\mathsf{2}$}
\label{sec:Na2}

In the previous example, the electronic density is relaxed around a
frozen ionic configuration of Na$_2$. This is not necessarily the
ionic ground state configuration which is defined as the configuration
with the absolute minimum of total energy. Asserting that one has
definitively found the absolute minimum is in general impossible,
except for few overseeable cases. There are several strategies to find
a minimum in the ionic landscape \cite{Pre92}. Some of them can also
be pursued with the QDD code (see parameter \texttt{icooltyp} in
\scn{sec:dyn_examples}).  The Na$_2$ dimer belongs to the overseeable
cases and here we have the chance to find the minimum in illustrative
manner.  We merely have to perform a series of calculations with
varied distance between the ions, and draw the emerging binding
energies (BE) as function of distance. One can find the input file to
such a series in the directory
\texttt{\$QDD\_ROOT/examples/Na2/ionic-BO-surfaces/},
sorted in subdirectories called \texttt{gs$\pm$<value>}.

Figure~\ref{fig:pes} shows the result, which is often called the
potential energy surface or Born-Oppenheimer surface (even if here,
this is rather a curve than a surface).  
\begin{figure}[bh]
  \centering
  \includegraphics[width=0.6\linewidth]{na2_pes.pdf}
  \caption{Potential energy surface of Na$_2$, i.e. binding energy as
    a function of the Na-Na distance. The solid curve is a fit with
    the Morse function $f(x)=a\left\{1-
    \exp\left[-b(x-c)\right]\right\}^2+d$.}
  \label{fig:pes}
\end{figure}
Close to the equilibrium
ionic configuration (i.e. the minimum), the plot of binding energy
versus ionic distance comes close to a parabola from which we may
deduce also the molecular vibration frequency.
			


\clearpage

\section{Basic I/O structure of dynamic calculations}
\label{sec:dyn_basics}

A dynamical simulation, of course, requires that the basic parameters
for system and gridding (namelists \texttt{GLOBAL} and \texttt{PERIO})
are set and that a reasonable starting point is defined from a static
calculation.  All parameters specific for running a dynamic simulation
are then contained in namelist \texttt{DYNAMIC} in file
\texttt{for005.<name>}. This namelist is long. We present it in
smaller chunks collected according to topic. Table
\ref{tab:dyn-input-params-general} describes parameters for dynamical
propagation, tables \ref{tab:dyn-input-params-instantaneous} and
\ref{tab:dyn-input-params-excitation} describe the choice of
excitation mechanism, and tables
\ref{tab:dyn-input-params-observables} with
\ref{tab:dyn-input-params-observables2} indicate how to activate the
various observables.

Mind, as said above, that a dynamical calculation should start from a
(converged) stable static configuration.  This can be achieved in two
ways: One strategy is to perform the static solution up to
convergence, save the configuration by setting \texttt{isaves>0}, and
starting dynamics from this static configuration by setting
\texttt{tstat=.TRUE.}, for details see \scn{sec:basicDYN}.
%\scn{sec:resuming}. 
The other
strategy is to combine static and dynamic calculations in one run
simply be setting a sufficient number of static iterations
\texttt{ismax} together with the desired number of dynamics steps
\texttt{itmax}. The code will automatically run through.  Finally,
there is a third option to continue a dynamical calculation by
restarting from a previously saved dynamical configuration, see
parameter \texttt{irest} and  \scn{sec:basicDYN}. %\scn{sec:resuming}.

This section is concerned with input parameters and output files.
Examples are presented in \scn{sec:dyn_examples}. One can find there
typical values for some of the parameters used in the calculation of 
dynamics.
			

\subsection{Basic input parameters in the \texttt{DYNAMIC} namelist}
\label{sec:basicDYN}

%			\begin{table}[htbp]
%\begin{longtable}{|p{3.5cm}|p{11.2cm}|}
\begin{table}
\caption{Basic dynamical parameters in the \texttt{DYNAMIC} namelist in \texttt{for005.<name>}}\label{tab:dyn-input-params-general}
\begin{tabular}{|p{2.3cm}|p{1cm}|p{0.8cm}|p{9.3cm}|}
%\endfirsthead
%				\begin{tabular}{|p{3.5cm}|p{11.2cm}|}
\hline
\multicolumn{4}{|c|}{Basic dynamic settings in  namelist\texttt{DYNAMIC}}\\
\hline
\multicolumn{4}{|c|}{\textit{\color{activeColor}Storing/retrieving full states of the system}} \\
\hline
\texttt{tstat}&&\tgreen{F}& Switch to read static wave functions from \texttt{RSAVE.<name>} \\
\hline
\texttt{irest}&&\tgreen{0}& Switch to restart dynamics  from \texttt{SAVE.<name>} \\
\hline
\texttt{isaved}&&\tgreen{0}& Frequency of saving actual state to \texttt{SAVE.<name>} \\
\hline
\multicolumn{4}{|c|}{\textit{\color{activeColor}Electronic stepping, see \scn{cpc:sec:numdynel}}} \\
\hline
\texttt{dt1}& $\delta t$ [1/Ry]&\tgreen{$\emptyset$}& Time step for propagating electronic wave
functions
 (1 $\hbar$/Ry $=$ 0.0484 fs) \\
\hline
\texttt{itmax}&&\tgreen{1000}& Number of time steps for electronic propagation  \\
\hline
\texttt{ifexpevol} &&\tgreen{F}& Exponential evolution instead of TV splitting \\
\hline
\multicolumn{4}{|c|}{\textit{\color{activeColor}Ionic stepping, see \scn{cpc:sec:numdynion}}} \\
\hline
\texttt{ionmdtyp}&&\tgreen{0}& Ionic propagation scheme\\
  &&& 0 $\rightarrow$ no ionic propagation \\
  &&& 1 $\rightarrow$ leap-frog\\
  &&& 2 $\rightarrow$ velocity Verlet\\
\hline
\texttt{modionstep}&$\delta t_I/\delta t$&\tgreen{1}& Number of electronic steps per ionic step  \\
\hline
\texttt{icooltyp}&&\tgreen{0}& Type of ionic cooling \\
  &&& 0 $\rightarrow$ none  \\
  &&& 1 $\rightarrow$ pseudo-dynamics\\
  &&& 2 $\rightarrow$ steepest descent\\
  &&& 3 $\rightarrow$ Monte Carlo\\
\hline
\texttt{ifredmas}&&\tgreen{F}& Switch to use reduced mass for ions in dynamics (pseudo-dynamics \\
\hline
\texttt{tfixcmion} &&\tgreen{F}& Fix ionic c.m. during ionic motion  \\
\hline
\texttt{tfreezekspot} &&\tgreen{F}& Switch to freeze the initial KS potential
during all dynamics \\
\hline
\hline
\multicolumn{4}{|c|}{\textit{\color{activeColor}Absorbing boundary conditions, see \scn{cpc:sec:abso}}} \\
\hline
\texttt{ispherabso}&&\tgreen{1}& Choice of shape of mask function in absorbing bounds\\
  &&& 0 $\rightarrow$ 3D box\\
  &&& 1 $\rightarrow$ spherical  \\
  &&& 2 $\rightarrow$ ellipsoidal\\
\hline
\texttt{nabsorb}&$N_\mathcal{M}$&\tgreen{0}& Number of grid points in the absorbing zone at
boundary  \\
  &&& 0 $\rightarrow$ switches off the absorbing boundary conditions\\
\hline
\texttt{powabso}&$\gamma_\mathcal{M}$&\tgreen{1/16}& Power  in the mask function in the case of absorbing boundary conditions \\
  \hline
  \texttt{iangabso} &&\tgreen{0}& Choice of origin for initialization of absorbing bounds \\ 
   &&& 0 $\rightarrow$ center of numerical box \\
   &&& 1 $\rightarrow$ electronic center of mass\\
   &&& 2 $\rightarrow$ ionic center of mass\\
\hline
\multicolumn{4}{|c|}{\textit{\color{activeColor}Activating RTA}} \\
  \hline
  \texttt{jrtaint} &$\Delta t/\delta t$&\tgreen{0}&  Frequency for calling RTA, see 
section \ref{cpc:sec:RTA} \\
\hline
\end{tabular}
\end{table}
%\end{longtable}
%
Table~\ref{tab:dyn-input-params-general} lists basic settings for the
computation of a dynamics. First come the parameters for
start/restart. The parameter \texttt{tstat} is already known from the
description of static calculations, see \scn{sec:stat_basics}. It is
also relevant for the initialization of a dynamic run as setting
\texttt{tstat=.TRUE.} together with \texttt{ismax=0} allows one to
start directly from a previously saved static configuration.  A
setting of the parameter \texttt{irest>0} continues a dynamical
simulation from a configuration previously saved on
\texttt{SAVE.<name>}. Note that the time index starts on the latest
value before saving which means that \texttt{itmax} has to be larger
than the time step saved in order to render continuation active. Note
also that computations performed during the former run {\em after} the
last save are thus when restarting. It is thus advantageous to choose,
whenever possible, a value of maximum time \texttt{itmax} as a integer
multiple of \texttt{isaved} to avoid such loss.  Crucial for reading
static or dynamic configurations is that these had been saved before
which is achieved with setting \texttt{isaved>0} (or \texttt{isaves>0}
in static runs). That parameter should not be too small because
writing on a full configuration on disk is time consuming. Ideally,
saving at the end of a computation is the best. However, occasional
intermediate saving is recommended in situation where one is not sure
how a simulation performs on long term. 

At this point, it may be useful to summarize briefly the save and
restart strategies. Static configurations are saved on
\texttt{RSAVE.<name>} during static calculations if \texttt{isaves>0}
is specified. The saved static configurations are read by invoking
\texttt{tstat=T} in the static input. This holds also if one wants to
use this configuration as basis for a dynamical calculation. In that
case, one has to override further static steps by setting
\texttt{ismax=0}. Dynamic configurations are saved on
\texttt{SAVE.<name>} if \texttt{isaved>0} is specified. To restart a
dynamical calculation directly from the saved dynamical configuration,
set \texttt{irest>0}. In all cases, it is important to use the same
qualifier \texttt{*.<name>} as in the previous run where the
configuration has been saved.


The input parameters concerning electronic propagation are obvious from
\scn{cpc:sec:numdynel}. Some more explanations are needed here for
ionic propagation. Leap-frog and velocity Verlet stepping are
explained in \scn{cpc:sec:numdynion}. The (possibly larger) ionic time
step $\delta t_I$ is regulated by \texttt{modionstep} which, however,
is applicable only for the velocity Verlet scheme.  A subtle option is
\texttt{ifredmas}. It sets each ionic mass to a much smaller value,
namely half the proton mass. This produces much smaller ionic
propagation at the price of destroying the relation between ionic and
electronic motion. It makes sense for ionic motion where the electron
cloud stays close to its ground state (adiabatic motion). And it is
particularly designed for pseudo-dynamics on the way to the ionic
ground state configuration. This points to option \texttt{icooltyp}
which activates ionic cooling rather than ion-electron dynamics, see
\scn{cpc:sec:numstaticion}. The most educated scheme is simulated
annealing (Monte Carlo) which, however, requires a lot of experience,
see the header of \texttt{carlo.F90} for a summary of the parameters
of the method and a proposed setting. For the typical practical task
of fine-tuning a configuration given from elsewhere, the recommended
method is pseudo-dynamics, \texttt{icooltyp=1}.

Ionic propagation allows still more advanced options.  The
\texttt{tfixcmion} activates restoration of ionic center-of-mass after
each ionic step such that the molecule stays always centered at the
origin of the box. The point is that external fields have also an
impact on ionic motion and can move also the ionic center of mass.
Fixing the ionic center can be convenient to keep the process
localized. But one has to keep in mind that one changes the dynamics
of the system, more precisely that one changes the frame of reference.
%
The parameter \texttt{tfreezekspot} turns off the update of the KS
potential in a dynamical calculation.  
The dynamics then runs in a mean field frozen in the ground-state
configuration after static iterations.
This switch offers a tool for
theoretical investigations. It is useful to study the impact of
the KS self-consistent
 interaction beyond the ground state by comparing
a full dynamical run from \texttt{tfreezekspot=.FALSE.}
with a motion in the frozen case  \texttt{tfreezekspot=.TRUE.}.
The latter delivers the properties of pure one-particle-one-hole
excitations in case of small amplitudes.
%

The next block in \tab{tab:dyn-input-params-general} defines the
absorbing boundary conditions. Although this defines grid properties,
we place it in namelist \texttt{DYNAMIC} because it is used only in a
dynamical context.  The switch \texttt{ispherabso=.TRUE} activates
spherical absorbing zones as indicated in figures \ref{cpc:fig:mask},
else the rectangular bounds are employed.  The meaning of
\texttt{nabsorb} and \texttt{powabso} is clear from eqs.
\eqref{cpc:eq:mask} and \eqref{cpc:eq:abspoints}.  The
\texttt{iangabso} is relevant in case of spherical absorbing
bounds. It defines the position of the center of the absorbing sphere.

Finally, the parameter \texttt{jrtaint} regulates the steps size for
RTA.  The case \texttt{jrtaint=0} overrides RTA at all and larger
values activate it.  We remind that there are two time steps: one coarse
grained $\Delta t$ for evaluation of relaxation in RTA, and one fine
grained for TDLDA $\delta t=$\texttt{dt1}. They are related as $\Delta
t=$\texttt{jrtaint}$*$\texttt{dt1}. Typical values for
\texttt{jrtaint} are such that the order of magnitude of $\Delta t$ is
about $\sim 10$ Ry$^{-1}$. E.g. for water, we typically use \texttt{dt1=0.0125},
\texttt{jrtaint=1000} such that $\Delta t=12.5$ Ry$^{-1}$.



\subsection{Input parameters for initial excitation in namelist \texttt{DYNAMIC}}

True dynamics requires excitation and the code supplies a lot of
different choices for that, see \scn{cpc:sec:examples} of the reference paper. Thus we split
the presentation into the two different types: instantaneous
excitations as, e.g., initial boost or excitation by external fields
over a certain time span. For all types excitation holds in
  general that the default is set to ``no excitation''. Any wanted
  excitation has to be asked for explicitly and one should take care not to invoke
  possibly interfering excitations simultaneously.

%\begin{longtable}{|p{2.5cm}|p{1cm}|p{11.2cm}|}
\begin{table}
\caption{Parameters for instantaneous initial excitations in a
  dynamical calculation, in the  \texttt{DYNAMIC} namelist of
  \texttt{for005.<name>} 
\label{tab:dyn-input-params-instantaneous}}
%\endfirsthead
\begin{tabular}{|p{1.8cm}|p{1.8cm}|p{0.5cm}|p{9.2cm}|}
%				\begin{tabular}{|p{3.5cm}|p{11.2cm}|}
\hline
\multicolumn{4}{|c|}{\texttt{DYNAMIC} namelist}\\
\hline
\multicolumn{4}{|c|}{\textit{\color{activeColor}Parameters of
    instantaneous electron excitation}} \\
\hline
\texttt{centfx}&$p_{0,x}$ [a$_0^{-1}$]&\tgreen{0.0}& Electronic boost in $x$ direction \\
\texttt{centfy}&$p_{0,y}$ [a$_0^{-1}$]&\tgreen{0.0}& Electronic boost in $y$ direction \\
\texttt{centfz}&$p_{0,z}$ [a$_0^{-1}$]&\tgreen{0.0}& Electronic boost in $z$ direction \\
\hline
\texttt{shiftinix} &$s_{0,x}$ [a$_0$]&\tgreen{0.0}&  Initial electronic shift in $x$ direction \\
\texttt{shiftiniy} &$s_{0,y}$ [a$_0$]&\tgreen{0.0}&  Initial electronic shift in $y$ direction  \\
\texttt{shiftiniz} &$s_{0,z}$ [a$_0$]&\tgreen{0.0}&  Initial electronic shift in $z$ direction  \\
\hline
\texttt{tspindip}&&\tgreen{F}& Optional different boosts in each spin subspace \\
  &&& F $\rightarrow$ dipole boost, same for spins up and down \\
  &&& T $\rightarrow$ spin-up and -down boosted in opposite directions\\
\hline
\texttt{phirot} &$\phi_\mathrm{initrot}$&\tgreen{0.0}& Angle of initial rotation of ionic background, in degrees (only with \texttt{irotat>0})\\
\hline
\texttt{irotat} &&\tgreen{0}& Axis of rotation for scissor mode excitation\\
 &&& 0 $\rightarrow$ no rotational initialization \\
 &&& 1 $\rightarrow$ $x$ axis\\
 &&& 2 $\rightarrow$ $y$ axis\\
 &&& 3 $\rightarrow$ $z$ axis\\
 &&& 4 $\rightarrow$ diagonal of the numerical box\\
\hline
\multicolumn{4}{|c|}{\textit{\color{activeColor}Parameters of instantaneous ion excitation}} \\
\hline
\texttt{tempion}&$T_\mathrm{ion,init}$ [K]&\tgreen{0.0}
 & temperature to produce an initial thermal distribution of ionic velocities (only with \texttt{ionmdtyp=1}) \\
\hline
\end{tabular}
\end{table}
Table~\ref{tab:dyn-input-params-instantaneous} summarizes the
parameters for all instantaneous excitation mechanisms in the code.
The most used instantaneous electronic excitation is the initial boost
Eq. (\ref{cpc:eq:boost}) with the input parameters
\texttt{centfx,centfy,centfz}. There is also the option of an initial
shift
\begin{equation}
  \varphi_\alpha(\mathbf{r},t\!=\!0)
  =
 \varphi_{\alpha,\mathrm{g.s.}}(\mathbf{r}-\mathbf{s}_0)
\end{equation}
of the whole electron cloud. Both can serve to excite the dipole mode
of a molecule, as an idealization of an ultra-fast excitation delivered
by a by-passing highly charged projectile or a very short laser pulse
(see \cite{Cal97b} for a discussion of the different effects of shift
versus boost).  Setting the switch \texttt{tspindip=.TRUE.} replaces
mere dipole boost by excitation of a spin-dipole boost
\begin{equation}
    \varphi_\alpha(\mathbf{r},t\!=\!0)
  =
  e^{\mathrm{i}\sigma_\alpha\mathbf{p}_0\cdot\mathbf{r}}
  \varphi_{\alpha,\mathrm{g.s.}}(\mathbf{r})
\label{eq:spindipboost}
\end{equation}
where $\sigma_\alpha$ is the spin associated with state $\alpha$.

A non-zero angle \texttt{phirot} generates an instantaneous rotational excitation.
It does that by rotating the ionic background 
\begin{equation}
  \mathbf{R}_I
  \longrightarrow
  \hat{D}(\mathbf{\phi_\mathrm{inirot}})\mathbf{R}_I
\label{eq:rotboost}
\end{equation}
where $\hat{D}(\mathbf{\phi_\mathrm{initrot}})$ is the 3$\times$3
matrix of a rotation by the angle
$\phi_\mathrm{initrot}=|\mathbf{\phi_\mathrm{initrot}}|$ about the
axis defined by the direction of $\mathbf{\phi_\mathrm{initrot}}$.
Actually, we supply four different choices for the axis of rotation
with the parameter \texttt{irotat}, see
\tab{tab:dyn-input-params-instantaneous}.  The same rotational
transformation can also be used in connection with a jellium
background. This excitation is useful in connection with studying the
scissors mode, the dominating orbital magnetic mode, having angular
momentum $J=1^-$. Although this rotational excitation acts on the
ions, it excites eventually an electronic mode because electrons are
not rotated together with ions.
Note that instantaneous initial rotation is not compatible with
initial boost or shift. The code stops if one tries to activate both
mechanisms simultaneously.

The parameter \texttt{tempion} gives the ionic configuration an
initial temperature $T_\mathrm{ion,init}$. This is achieved by
producing in Monte-Carlo fashion a thermal Boltzmann distribution of ionic
velocities. 

%\clearpage

\begin{table}[ptbh]
\caption{Parameters for initial excitation by external fields in a
  dynamical calculation, in the  \texttt{DYNAMIC} namelist of
  \texttt{for005.<name>}. 
\label{tab:dyn-input-params-excitation}}
%\endfirsthead
\begin{tabular}{|p{2.2cm}|p{2.35cm}|p{0.7cm}|p{8.1cm}|}
\hline
\multicolumn{4}{|c|}{\textit{\color{activeColor}Laser parameters, 
see \scn{cpc:sec:laser} and Eq. \ref{eq:pulse2}}}\\
\hline
\texttt{itft}&&\tgreen{3}& Choice of shape of laser pulse envelop \\
  &&&   1 $\rightarrow$ ramp \\ %, sine switching on/off\\
  &&&   2 $\rightarrow$  Gaussian\\
  &&&   3 $\rightarrow \sin^2$, see Eq. (\ref{eq:pulse2}) \\
  &&&   4 $\rightarrow \sin^4$ \\
\hline
\texttt{tnode}&$t_0$ [fs]&\tgreen{0.0}& Time at which pulse
computation starts,
relevant for \texttt{itft}=1,3, and 4  \\
\hline
\texttt{deltat}&$T_\mathrm{pulse}$ [fs]&\tgreen{0.0}& Pulse duration\\
\hline
\texttt{tpeak}&$T_\mathrm{peak}$ [fs]&\tgreen{0.0}& Time relative
to \texttt{tnode} at which peak is reached,  only for \texttt{itft=1} and \texttt{2}  \\ 
\hline
\texttt{omega}&$\omega_\mathrm{las}$ [Ry]&\tgreen{0.0}& Photon frequency \\
\hline
\texttt{e0}&$|\mathbf{E}_0|$ [Ry/a$_0$]&\tgreen{0.0}& Photon field strength in Ry/$a_0$ \\
  \hline
\texttt{e1x,e1y,e1z}&$E_{0,i}/|\mathbf{E}_0|$&\tgreen{1,0,0}& Cartesian components of photon polarization  \\
\hline
\texttt{phi} &$\phi_\mathrm{las}$ [$^\circ$] &\tgreen{0.0}& Phase of laser pulse relative to
peak of envelope  \\
\hline
\texttt{tstart2}&$t_0^{(2)}$ [fs]&\tgreen{0.0}& Initial time of 2nd pulse \\
\hline
\texttt{omega2}&$\omega_\mathrm{las}^{(2)}$ [Ry]&\tgreen{0.0}& Photon frequency of 2nd pulse \\
\hline
\texttt{deltat2} &$T_\mathrm{pulse}^{(2)}$ [fs]&\tgreen{0.0}& Pulse duration of 2nd pulse \\
\hline
\texttt{e0\_2}&$|\mathbf{E}_{0}^{(2)}|$[Ry/a$_0$]&\tgreen{0.0}& Field strength
of 2nd laser pulse (only with \texttt{itft=3}, same pulse envelope for 2nd pulse)  \\
\hline
\texttt{e2x,e2y,e2z}&$E_{0,i}^{(2)}/|\mathbf{E}_0^{(2)}|$&\tgreen{0,0,0}& Cartesian components of photon polarization of 2nd pulse  \\
\hline
\texttt{phase2}&$\phi_\mathrm{las}^{(2)}$ [$^\circ$]&\tgreen{0.0}& Phase of 2nd pulse  \\
\hline
\multicolumn{4}{|c|}{\textit{\color{activeColor}Excitation by a by-passing
    ion, see Eq. (\ref{cpc:eq:ionext})}}\\
\hline
\texttt{projcharge} &$Z_\mathrm{ext}$&\tgreen{0.0}& Charge of a point-charge (classical) projectile \\
\hline
\texttt{projvelx} &$\dot{R}_{\mathrm{ext},x}(0)$[a$_0$/fs]&\tgreen{0.0}& Initial $x$-velocity
of projectile \\
\texttt{projvely} &$\dot{R}_{\mathrm{ext},y}(0)$[a$_0$/fs]&\tgreen{0.0}& Initial
$y$-velocity of projectile \\
\texttt{projvelz}
&$\dot{R}_{\mathrm{ext},z}(0)$[a$_0$/fs]&\tgreen{0.0}& Initial
$z$-velocity of projectile\\
\hline
\texttt{projinix} &$R_{\mathrm{ext},x}(0)$[a$_0$]&\tgreen{0.0}& Initial $x$-position of projectile, in $a_0$ \\
\texttt{projiniy} &$R_{\mathrm{ext},y}(0)$[a$_0$]&\tgreen{0.0}& Initial $y$-position of projectile, in $a_0$ \\
\texttt{projiniz} &$R_{\mathrm{ext},z}(0)$[a$_0$]&\tgreen{0.0}& Initial $z$-position of projectile, in $a_0$ \\
\hline
\end{tabular}
\end{table}
%\end{longtable}

Table~\ref{tab:dyn-input-params-excitation} presents the parameters for
excitation by external electric pulses, either from laser or from a
highly charge by-passing ion.  The main switch for photon pulses is the
field strength $\mathbf{E}_0$. Any value $|\mathbf{E}_0|>0$ activates
this excitation mechanism. The next important parameter is then
\texttt{itft} which defines the laser pulse profile.  Most used
standard is \texttt{itft=3}, the sin$^2$ pulse given in
\scn{cpc:sec:laser}. It reads in detail:
\begin{subequations}
\label{eq:pulse2}
\begin{equation}
  V_\mathrm{ext}(\mathbf{r},t;
  \omega_\mathrm{las},\mathcal{\mathbf{E}}_0,T_\mathrm{pulse},\phi_{\rm las},t_0)
  =
  e\,{\cal\bf E}_0\, f(t)\!\cdot\!\mathbf{r}
%  \cos{\left(-\mathrm{i}(\omega_\mathrm{las}\tilde{t}
%   -\phi_\mathrm{las})\right)}
\ \cos\left(\omega_\mathrm{las}(t-t_0-{\textstyle{\frac{1}{2}}}T_\mathrm{pulse})
   -\phi_\mathrm{las}\right)
\label{eq:laserr}
\end{equation}
\begin{eqnarray}
  f(t)
  &=&
  \left\{\begin{array}{ll}
    \sin^2{\left(\displaystyle \pi\frac{t-t_0}{T_\mathrm{pulse}}\right)}
        &\mbox{for}\quad t-t_0\in[0,2T_\mathrm{pulse}]  \\
    0  &\mbox{else}
  \end{array}\right.
  \label{eq:laserenv}
%\\
%  \tilde{t}
%  &=&
%  t-t_0-T_\mathrm{pulse}/2
  \quad.
\label{eq:sinpulse}
\end{eqnarray}
\end{subequations}
This case allows also to be combined with a
second pulse of same profile with an offset shifted by a time
$t_0^{(2)}$ which altogether leads to 
\begin{equation}
\begin{split}
  V_\mathrm{ext}^\mathrm{(double)}(\mathbf{r},t)
  =  &V_\mathrm{ext}(\mathbf{r},t;
  \omega_\mathrm{las},\mathcal{\mathbf{E}}_0,T_\mathrm{pulse},\phi_{\rm las},t_0)
  \\
  +&
  V_\mathrm{ext}(\mathbf{r},t;
  \omega_\mathrm{las}^{(2)},\mathcal{\mathbf{E}}_0^{(2)},
  T_\mathrm{pulse}^{(2)},\phi_{\rm las}^{(2)},t_0^{(2)})
  \end{split}
\label{eq:doublepulse}
\end{equation}
This double pulse can be used, e.g., to simulate pump-and-probe
experiments, see e.g. \cite{And04a}.

The choice \texttt{itft=4} goes for a sin$^4$ envelope and is
else-wise the same as the sin$^2$ pulse \texttt{itft=4}, however
working only for a single pulse (i.e. $\mathcal{\mathbf{E}}_0^{(2)}$, etc.,
are ineffective).

The choice \texttt{itft=1} produces a ramp pulse 
%\begin{equation}
%  V_\mathrm{ext}(\mathbf{r},t)
%  =
%  \mathcal{\mathbf{E}}_0\cdot\mathbf{r}
%  \cos{\left(\omega_\mathrm{las}\tilde{t}\right)}
%  \left\{\begin{array}{lcl}
%    \sin\left(\frac{\pi}{2}\,\frac{t-t_0}{T_\mathrm{peak}}\right)
%    &\mbox{for}&
%    0\leq t\!-\!t_0 < T_\mathrm{peak}
%  \\
%    1
%    &\mbox{for}&
%    T_\mathrm{peak}\leq t\!-\!t_0 \leq T_\mathrm{pulse}-T_\mathrm{peak}
%  \\
%    \sin\left(\frac{\pi}{2}\,\frac{T_\mathrm{pulse}-t+t_0}{T_\mathrm{peak}}\right)
%    &\mbox{for}&
%    T_\mathrm{pulse}-T_\mathrm{peak}\leq t\!-\!t_0 < T_\mathrm{pulse}
%  \\
%    0
%    &\mbox{for}&
%    t<t_0 \;\mbox{for}\; T_\mathrm{pulse}+t_0< t
%  \end{array}
%  \right.
%\end{equation}
%\begin{subequations}
\begin{equation}
\begin{split}
  V_\mathrm{ext}(\mathbf{r},t)
  &=
  \mathcal{\mathbf{E}}_0\cdot\mathbf{r}
  \cos{\big(\omega_\mathrm{las}(t-t_\mathrm{center}-\phi_\mathrm{las})\big)}
  f_\mathrm{ramp}(t-t_0)
  \quad,
\\
  t_\mathrm{center}
  &=
  t_0+\frac{T_\mathrm{pulse}}{2}+t_\mathrm{peak}
  \quad,
\\
  f_\mathrm{ramp}(\tilde{t})
  &=
  \left\{\begin{array}{lcl}\displaystyle
    \sin\left(\frac{\pi}{2}\,\frac{\tilde{t}}{T_\mathrm{peak}}\right)
    &\mbox{for}&
    0\leq \tilde{t} < T_\mathrm{peak}
  \\
    1
    &\mbox{for}&
    T_\mathrm{peak}\leq \tilde{t}\leq T_\mathrm{pulse}-T_\mathrm{peak}
  \\
    \displaystyle
    \sin\left(\frac{\pi}{2}\,\frac{T_\mathrm{pulse}-\tilde{t}}{T_\mathrm{peak}}\right)
    &\mbox{for}&
    T_\mathrm{pulse}-T_\mathrm{peak}\leq \tilde{t} < T_\mathrm{pulse}
  \\
    0
    &\mbox{for}&
    \tilde{t}<0 \mbox{  or  } T_\mathrm{pulse}< \tilde{t}
  \end{array}
  \right.
  \quad.
\end{split}
\label{eq:itft=1}
\end{equation}
%\end{subequations}
with soft switching at beginning and end, thus reducing unwanted long
tails in the frequency distribution of the pulse. The
$t_\mathrm{center}$ is the time in the middle of the ramp pulse.
It enters the fast oscillations to define the phase
$\phi_\mathrm{las}$ relative to the center of the pulse as in all
other pulses too.



The choice \texttt{itft=2} produces a Gaussian pulse
\begin{equation}
  V_\mathrm{ext}(\mathbf{r},t)
  =
  \mathcal{\mathbf{E}}_0\cdot\mathbf{r}
  \cos{\big(\omega_\mathrm{las}(t-T_\mathrm{peak})-\phi_\mathrm{las}\big)}
  \exp\left(-\frac{(t-T_\mathrm{peak})^2}{T_\mathrm{pulse}^2}\right)
  \quad.
\end{equation}

The ramp with its long constant piece in the middle allows a high
frequency selectivity around the peak frequency. However, the short
switching intervals produces long tails in the spectrum.  The Gaussian
pulse produces the best peaked spectrum but only if it extends to
infinity. In practice, it is cut to finite interval which means, in
particular, that it starts with a small step unavoidably polluting the
spectrum a bit. The sin$^n$ pulses provided by the options
\texttt{itft=2} or \texttt{itft=4} are efficient compromises combining
high spectral selectivity with a finite extension in time. We prefer
the sin$^2$ pulse as standard.

The lower entries in \tab{tab:dyn-input-params-excitation} are related
to an excitation by the Coulomb field of a bypassing charged ion as
given in Eq. (\ref{cpc:eq:ionext}).  This mechanism is
activated by setting the parameter \texttt{projcharge}$=Z_\mathrm{ext}>0$.
The trajectory of the external ion is assumed to be a
straight line defined by initial position and velocity relative to the
origin of the numerical box.  The impact parameter, needed if one
wants to compute an excitation cross-section, has to be deduced from
$\mathbf{R}_\mathrm{ext}(0)$ and $\dot{\mathbf{R}}_\mathrm{ext}(0)$
where a decision has to be made whether one computes that relative to
ionic or electronic center of mass. 


\subsection{Parameters for printing results in namelist \texttt{DYNAMIC}}

%\begin{longtable}{|p{3.5cm}|p{11.2cm}|}
\begin{table}
  \caption{Flags that control the printing of global observables in
    namelist \texttt{DYNAMIC} of \texttt{for005.<name>}}
\label{tab:dyn-input-params-observables}
%\endfirsthead
\begin{tabular}{|p{2.2cm}|p{0.7cm}|p{10.7cm}|}
  \hline
  \multicolumn{3}{|c|}{namelist \texttt{DYNAMIC}}\\
  \hline
  \multicolumn{3}{|c|}{\textit{\color{activeColor}Flags for writing general dynamical electronic observables}} \\
  \hline
  \texttt{jinfo}&\tgreen{10}& Frequency for writing basic stepping information to \texttt{infosp.<name>}  \\
  \hline
  \texttt{jenergy}&\tgreen{10}& Frequency for writing energy information to \texttt{penergies.<name>}   \\
  \hline
  \texttt{jdip}&\tgreen{$\emptyset$}& Frequency for writing dipole moments to \texttt{pdip.<name>}  \\
  \hline
  \texttt{jdiporb}&\tgreen{0}& Frequency for writing s.p. dipole moments to \texttt{pdiporb.*.<name>}  \\
  \hline
  \texttt{jspdp}&\tgreen{0}& Frequency for writing spin-dipole moments \texttt{pspdip.*.<name>}  \\
  \hline
  \texttt{jquad}&\tgreen{0}& Frequency for writing quadrupole moments to \texttt{pquad.<name>}  \\
  \hline
  \texttt{jgeomel}&\tgreen{0}& Frequency for writing electronic
  geometry parameters (c.m. moments, radius, quadrupole tensor) to \texttt{pgeomel.<name>}  \\
  \hline
  \texttt{jang}&\tgreen{0}& Frequency for writing angular momentum, see Eq.~(\ref{cpc:eq:angmom}), to \texttt{pangmo.<name>}  \\
  \hline
  \texttt{jstinf}&\tgreen{0}& Frequency for writing s.p. observables (energies, variances, etc.) to \texttt{psp*.<name>}  \\
  \texttt{jlaser}&\tgreen{0}& Frequency for writing laser information to \texttt{plaser.<name>} (only if \texttt{e0}$>0$)  \\
  \hline
  \texttt{jcharges}&\tgreen{0}& Frequency for writing number of electron in
  analyzing spheres
  of different radii \texttt{pcharges.<name>}  \\
  \hline
  \texttt{drcharges} &\tgreen{5.0}& Size of radial slices for the analyzing spheres \\
   \hline
  \multicolumn{3}{|c|}{\textit{\color{activeColor}Flags for writing simple observables related to emission}} \\
  \hline
  \texttt{jesc}&\tgreen{$\emptyset$}& Frequency for writing electron emission, see Eq.~(\ref{cpc:eq:nesc}), to \texttt{pescel.<name>}   \\
  \hline
  \texttt{jnorms}&\tgreen{0}& Frequency for writing s.p. electron emission to \texttt{pescOrb.<name>}, and probability to find a certain charge state in \texttt{pproba.<name>}, see \scn{cpc:sec:ionizprobab}  \\
  \hline
  \hline
  \multicolumn{3}{|c|}{\textit{\color{activeColor}Flags for writing dynamical ionic observables}} \\
  \hline
  \texttt{jpos}&\tgreen{$\emptyset$}& Frequency for writing ionic positions to
  \texttt{pposion.<name>}   \\
  \hline
  \texttt{jvel}&\tgreen{$\emptyset$}& Frequency for writing ionic velocities to \texttt{pvelion.<name>} (only if \texttt{ionmdtyp=1} and \texttt{nion>0})\\
  \hline
  \texttt{jposcm} &\tgreen{0}& Frequency for writing ionic center of mass coordinates to \texttt{pposCM.<name>} (only if \texttt{ionmdtyp=1})  \\
  \hline
  \texttt{jforce}& \tgreen{0}& Frequency for writing total forces on ions to \texttt{pforce.<name>}, and optionally from laser field to \texttt{plforce.<name>} or from point-charge projectile to \texttt{projforce.<name>} (only if \texttt{ionmdtyp>0}, only for \texttt{nion<10}) \\
  \hline
  \texttt{jgeomion} &\tgreen{0}& Frequency for writing global ionic geometry parameters to \texttt{pgeomion.<name>}  \\
  \hline
\end{tabular}
\end{table}
In this subsection, we discuss the parameters which determine what
results will be printed to file, how and how often.  With few
exceptions, all output files related to dynamical calculations names
start with the letter \texttt{p} for ``print''.  The presentation is
split into two parts. We start with
\tab{tab:dyn-input-params-observables} which deals with the ``simple''
observables. Most of them are obvious and the corresponding output
files explain in a header what comes in the appended columns.  In some
files (e.g. {\tt pdip.<name>} or {\tt pquad.<name>}), the last line of
the header shows a seemingly cryptic line of the sort \texttt{H: X YL
  YD YP}. This is an instruction telling the routine
\texttt{spectr2.f90} how to interpret the subsequent columns, for an
example see \scn{eq:applyboost}. Note also that ``frequencies'' in
\tab{tab:dyn-input-params-observables} are given in unit of multiples
of the electronic time step \texttt{dt1}. This also holds true for
ion-related observables.

We complement here a few details which may be not that obvious. While
\texttt{jdip} triggers output of the three components of the
electronic dipole momentum, the \texttt{jdiporb} does that for dipole
momenta $\mathbf{D}_\alpha(t)=\int
d^3r\,\mathbf{r}|\varphi_\alpha(\mathbf{r})|^2$ of each s.p. state
separately.  The \texttt{jspdp} calls for computing and printing the
spin dipole moment
\begin{equation}
  \mathbf{D}_S
  =
  \int d^3r\,\mathbf{r}\left(
    \varrho_\uparrow(\mathbf{r})-\varrho_\downarrow(\mathbf{r})
  \right)
\end{equation}
which is useful to analyze the spin-dipole mode, often the dominating
spin magnetic mode of a molecule. Spin-dipole excitations emerge
naturally in molecules with spin polarized ground state. In spin
saturated molecules, they are excited, e.g., by a spin-dipole boost
(\ref{eq:spindipboost}). The next higher order moments, the
  quadrupole moments (\ref{cpc:eq:quadmom}) are called for by
  \texttt{jquad} for electrons and by \texttt{jgeomel} for ions. Note
  that both files print the simple quadratic moments
  $\langle{r_ir_j}\rangle$. These are immediately the quadrupole
  moments $Q_{ij}$ for non-diagonal elements, but they have to be remapped
  for the diagonal elements, e.g., as
  $Q_{xx}=\langle{x^2}\rangle-\sum_i\langle r_i^2\rangle/3$.  

The parameter \texttt{jcharges} activates an analysis of the radial
distribution of the electronic density in interlaced spherical shells
and \texttt{drcharges} defines the difference of the radii $R_\nu$ of
the analyzing spheres. For each $R_\nu$, the electron content
$N_\nu=4\pi\int_0^{R_\nu}dr\,r^2\,\int d^2\Omega\varrho(\mathbf{r})$ is
computed and all $N_\nu$ at given time step are written to the file
\texttt{pcharges.<name>}. This yields a radially averaged map of the
density distribution which is simpler to visualize than the full 3D
density and allows, in particular, to scrutinize the time evolution of
particle emission. This option is thus especially useful in
connection with absorbing boundary conditions. 


The second block in \tab{tab:dyn-input-params-observables} deals with
simple observables specific to electron emission.  The parameter
\texttt{jesc} triggers output of the most basic emission observable,
namely total ionization Eq. (\ref{cpc:eq:nesc}).  The parameter \texttt{jnorms}
calls for more details, namely the electron emission from each
s.p. state separately and deduced from that the probability to find
the final total electron configuration in a certain charge state $Q$,
see paragraph \ref{cpc:sec:ionizprobab}.

Finally, \tab{tab:dyn-input-params-observables} contains a couple of
parameters related to ionic motion.  The parameters \texttt{jpos},
\texttt{jvel}, \texttt{jposcm}, and \texttt{jforce} are obvious and
need no further explanation, except for mentioning that they can grow
rather bulky for large molecules because information for each ion is
printed. Global information on the ion distribution (center of mass,
quadrupoles, ...) is triggered with \texttt{jgeomion}. For large
molecules, this latter file is probably the first to look at before
going into details with the other outputs.


\begin{table}
  \caption{Flags that control the printing of detailed observables in
    namelist \texttt{DYNAMIC} of \texttt{for005.<name>}}
\label{tab:dyn-input-params-observables2}
%\endfirsthead
\begin{tabular}{|p{2.4cm}|p{1.4cm}|p{0.8cm}|p{8.8cm}|}
  \hline
  \multicolumn{4}{|c|}{\textit{\color{activeColor}Parameters for PAD,
      see eq. (\ref{eq:PAD})}}\\
  \hline
  \texttt{jangabso}&&\tgreen{0}& Frequency for writing PAD to \texttt{pangabso.<name>} (only with \texttt{iangabso>0})  \\
  \hline
  \texttt{nangtheta} &$n_\theta^\mathrm{(PAD)}$&\tgreen{1}& Number of angular bins in $\theta$ direction for PAD   \\
  \hline
  \texttt{nangphi} &$n_\phi^\mathrm{(PAD)}$&\tgreen{1}& Number of angular bins in $\phi$ direction for PAD   \\
  \hline
  \texttt{delomega} &$\Delta\Omega$&& Space angle of angular cones in PAD \\
  \hline
  \texttt{angthetal} &$\theta_\mathrm{low}^\mathrm{(PAD)}$[$^o$]&\tgreen{0.0}& Lowest $\theta$ angle for PAD   \\
  \hline
  \texttt{angthetah} &$\theta_\mathrm{up}^\mathrm{(PAD)}$[$^o$]&\tgreen{90.0}& Uppermost $\theta$ angle for PAD  \\
  \hline
  \texttt{angphil} &$\phi_\mathrm{low}$[$^o$]&\tgreen{0.0}& Lowest $\phi^\mathrm{(PAD)}$ angle for PAD  \\
  \hline
  \texttt{angphih} &$\phi_\mathrm{up}$[$^o$]&\tgreen{90.0}& Uppermost $\phi^\mathrm{(PAD)}$ angle for PAD  \\
  \hline
  \multicolumn{4}{|c|}{\textit{\color{activeColor}Parameters for PES}}\\
  \hline
  \texttt{jmp} &&\tgreen{0}& Frequency for writing of raw data to be post-processed for a photo-electron spectrum, to \texttt{pMP.<name>} \\
  \hline
  \texttt{nmptheta} &$n_\theta^\mathrm{(PES)}$&\tgreen{2}& Number of measuring points for PES in $\theta$ direction   \\
  \hline
  \texttt{nmpphi} &$n_\phi^\mathrm{(PES)}$&\tgreen{1}& Number of measuring points for PES in $\phi$ direction   \\
  \hline
  \multicolumn{4}{|c|}{\textit{\color{activeColor}Flags for detailed
      printing of densities}}\\
  \hline
  {\texttt{jdensity2d}}&&\tgreen{0}& Frequency for plotting 2D cuts of
  electron density   \\
\hline
  \texttt{jdensity1d}&&\tgreen{0}& Frequency to compute 1D reduced electronic densities   \\
  \hline
  {\texttt{jdensitydiff}}&&\tgreen{0}& Frequency for for printing the difference of actual density 'rho' with initial density   \\
  \hline
  \multicolumn{3}{|l|}{\texttt{jdensitydiff2d}}& Same as \texttt{jplotdensitydiff} but 2D   \\
  \hline
  \multicolumn{4}{|c|}{\textit{\color{activeColor}Flags for elaborate
      observables}}\\
  \hline \texttt{jelf}&&\tgreen{0}& Frequency for writing electron localization function to \texttt{pelf*.<name>}\\
  \hline					
  \texttt{jstateoverlap}&&\tgreen{0}& Frequency for writing overlap between
  actual and ground state wave function to \texttt{povlp.<name>}  \\
  \hline
\end{tabular}
\end{table}
%\end{longtable}
\tab{tab:dyn-input-params-observables2} continues the list of
parameters governing output of observables with the more involved
part. The PAD, triggered by \texttt{jangabso}, is collected from
density of emitted electrons, i.e.  electrons removed by the mask in
the absorbing zone, denoted $n_{\mathrm{esc},\alpha}(\mathbf{r})$ in
\scn{cpc:sec:pad}. It is a function of emission angle and evaluated on
a grid $\mathbf{\Omega}_{\nu\mu}$ over the unit sphere in terms of
angles $\theta$ and $\phi$ which read
\begin{equation}
  \theta_\nu
  =
  \theta_\mathrm{low}^\mathrm{(PAD)}
  +\nu\frac{\theta_\mathrm{up}^\mathrm{(PAD)}-\theta_\mathrm{low}^\mathrm{(PAD)}}{n_\theta^\mathrm{(PAD)}}
  \;,\;
  \phi_\mu
  =
  \phi_\mathrm{low}^\mathrm{(PAD)}
  +\mu\frac{\phi_\mathrm{up}^\mathrm{(PAD)}-\phi_\mathrm{low}^\mathrm{(PAD)}}{n_\phi^\mathrm{(PAD)}}
  \quad.
\label{eq:PAD}
\end{equation}
The PAD is obtained from integrating in an angular cone of space angle
$\Delta\Omega$ centered around direction of the angles
$\theta_\nu,\phi_\mu$, as explained in
Eq. (\ref{cpc:eq:PADfixed}).

The parameter \texttt{jMP} invokes writing the information needed for
later analysis of PES according to Eq. (\ref{cpc:eq:PESformula}),
namely a protocol of the time evolution of all s.p. wave functions at
certain measuring points $\mathbf{r}_\mathcal{M}$ near the absorbing
bounds.  These are placed, similar as with PAD, on an angular grid
over the unit sphere, however, with own, hardwired bounds
\begin{equation}
  \theta_\nu
  =
  \nu\frac{180^o}{n_\theta^\mathrm{(PES)}}
  \;,\;
  \phi_\mu
  =
  +\mu\frac{360^o}{n_\phi^\mathrm{(PES)}}
  \quad.
\label{eq:PES}
\end{equation}
The total number of measuring points
$N_\mathcal{M}=n_\phi^\mathrm{(PES)}n_\theta^\mathrm{(PES)}$ plays a
role communicating the results to the post-processing routine, see
\scn{sec:PES}.

The third block in \tab{tab:dyn-input-params-observables2} presents a
couple of detailed observables.  The
\texttt{jdensity2d} and \texttt{jdensity1d} trigger writing
densities in reduced dimensions, as 2D cuts or 2D integrated over the
other dimensions, e.g. in $z$ direction as $\iint \textrm dx\, \textrm
dy \,\varrho(x,y,z,t)$. The density as such is often not instructive
enough because the large background of the ground-state density
over-shines the sometimes subtle dynamical changes. To extract them
more clearly serves the parameter \texttt{jdensitydiff} which
triggers writing of the difference
$\varrho(\mathbf{r},t)-\varrho(\mathbf{r},0)$ which subtracts the offset of
the ground-state density. Be careful, this parameter orders writing of
the whole 3D density what may produce huge output files. To reduce the
expense, 2D cuts of the difference density can be ordered by
\texttt{jdensitydiff2d}.

Finally come two rather advanced observables.  The parameter
\texttt{jelf} triggers computation and writing of the electron
localization function Eq. (\ref{eq:local}) which was already defined
and discussed in paragraph \ref{sec:statobs}.  The parameter
\texttt{jstateoverlap} triggers computation and writing of the overlap
of actual and initial Slater state $\langle\Phi(t)|\Phi(0)\rangle$.
This observable is useful for theoretical studies of changes of the
many-body states with time, e.g., distance to the initial state,
recurrence times etc. However, it can presently only be applied to pure
Slater states. An attempt to use it else-wise terminates the program.




\subsection{Output files}



%			\begin{table}[htbp]
%\begin{longtable}{|p{3.8cm}|p{10.0cm}|}
\begin{table}
  \caption{Output files generated during a dynamic calculation. The
    second column indicates the input parameters which activates
    writing the corresponding file. To save space, we indicate the
    general qualifier \texttt{<name>} by a \texttt{*}.
\label{tab:dynamic-output-files}}
%\endfirsthead
%				\begin{tabular}{|p{4.5cm}|p{10.2cm}|}
\begin{tabular}{|p{3cm}|p{1.8cm}|p{8.9cm}|}
  \hline
  \multicolumn{1}{|c}{\em Filename} &
  \multicolumn{1}{|c|}{\em Switch} &  \multicolumn{1}{c|}{\em Explanation} \\
  \hline
  \texttt{poptions.*}&& Overview of basic parameters
  and chosen options on solvers, compiler options, etc.\\
  \hline
  \texttt{out\_detail.*}&& Detailed protocol file,
  similar information as screen output\\
  \hline
  \texttt{SAVE.*} &\texttt{isaved}& Saves the dynamical state of the system\\
  \hline	
%  \texttt{*.bs} &\texttt{jpos}& Output for 3D structure plots of molecular configurations\\
%  \hline
  \texttt{pdip.*} &\texttt{jdip}&  Dipole moment in x, y, z direction \\
  \hline
  \texttt{pquad.*} &\texttt{jquad}&  Cartesian quadrupole moments \\
  \hline
  \texttt{pgeomel.*} &\texttt{jgeomel}& Global geometry parameters of
  electron distribution \\
  \hline
  \texttt{pdiporb.i.*} &\texttt{jdiporb}&  Dipole moment in
  \texttt{i}=\texttt{x,y,z} direction\\
  \hline
  \texttt{pcharges} & \texttt{jcharges} & Radial charge distribution\\
  \hline
  \texttt{pescel.*} &\texttt{jesc}& Ionization observables\\
  \hline
  \texttt{plaser.*} &\texttt{jlaser}& Laser parameters, as actual field
  strength etc\\
  \hline
  \texttt{povlp.*} &\multicolumn{2}{l|}{\texttt{jstateoverlap} $\quad$ Overlap between actual and initial Slater state}\\
  \hline
  \texttt{penergies.*} &\texttt{jenergy}& Total energy and separate
  contributions to it\\
  \hline
  \texttt{pescOrb.*} &\texttt{jnorms}& Number of emitted electrons
  lost for each orbital\\
  \hline
  \texttt{pproba.*} &\texttt{jnorms}& Probabilities of charge states\\
  \hline
  \texttt{pspenergies.*} &\texttt{jstinf}& s.p. energies\\
  \hline
  \texttt{pspvariances.*} &\texttt{jstinf}& variances of s.p. energies\\
  \hline
  \texttt{pMP.*} &\texttt{jmp}& information for later evaluation of PES\\
  \hline
  \texttt{pkinenion.*} &\texttt{jvel}& kinetic energy of ions x,y,z directions and total\\
  \hline
  \texttt{pposion.*} &\texttt{jpos}& positions of the ions in
  $x,y,z$, and distance to center\\
  \hline
  \texttt{pposCM.*} &\texttt{jposcm}& center-of-mass of the ionic configuration
  \\
  \hline
  \texttt{pvelion.*} &\texttt{jvel}& ion velocities\\ %, triggered by \texttt{jvel}\\
  \hline
  \texttt{pgeomion.*} &\texttt{jgeomion}& global geometry observables\\
  \hline
  \texttt{pforce.i.*} &\texttt{jforce}& forces on ions, 
\texttt{i}$\in\{$\texttt{x,y,z}$\}$\\
  \hline
  \texttt{plforce.i.*} &\multicolumn{2}{l|}{\texttt{jforce},\texttt{jlaser}$\;$ forces
  on ions from photon field}\\
  \hline
  \texttt{pdensdiff.*} &\multicolumn{2}{|l|}{\texttt{jdensitydiff} $\quad$ density difference to  g.s. density}\\
  \hline
  \texttt{pdens2DXX.*} &\texttt{jdensity2d} & XX$\in\{$xy,yz,xz$\}$, 2D cuts of density in XX-plane\\
  \hline
  \texttt{pdensdiff2DXX.*} &\texttt{jdensity2d} &
  XX$\in\{$xy,yz,xz$\}$, 2D cuts of density difference in XX-plane\\
  \hline
  \texttt{rho1DXX.*} &\texttt{jdensity1d} & XX$\in\{$x,y,z$\}$, 1D
  reductions of density to X-direction\\
  \hline
\end{tabular}
\end{table}
%\end{longtable}
%
%
\tab{tab:dynamic-output-files} lists the output files generated during
a dynamical simulation (except those related to RTA). Most of them are
produced on demand.  The corresponding switch (or print-frequency)
parameter is indicated in the second column. Explanations are kept
short because the printed observables were already explained in
connection with the switch parameters in tables
\ref{tab:dyn-input-params-observables} and
\ref{tab:dyn-input-params-observables2}. The output structure is
explained in the headers of the output files.

Nonetheless, a few more explaining words are in order for particular
cases. The file \texttt{pescel.<name>} records the electron loss, or
emission respectively, from the absorbing boundary conditions. It does
that in three different formats.  Column three prints the ionization
computed as in Eqs.  (\ref{cpc:eq:depletion},\ref{cpc:eq:nesc}). Column two
prints ionization relative to the initial electron number. Column four
prints ionization computed alternatively from
$n_{\mathrm{esc},\alpha}(\mathbf{r})$, the accumulated density of
electrons removed by the mask, see Eq. (\ref{cpc:eq:PADfixed}).  It serves
as a check of stability of the calculation: columns 3 and 4 that
should show the same. Note that this is the total ionization. Ionization
per s.p. state is given in \texttt{pescOrb.<name>}.

The file \texttt{ptempion.*} contains the total ionic kinetic energy
in terms of ionic temperature computed as
\begin{equation}
  T_\mathrm{ion}
  =
  2\frac{E_\mathrm{kin,ion}-E_\mathrm{kin,cm,ion}}{3N_\mathrm{ion}-3}
\end{equation}
where $E_\mathrm{kin,cm,ion}$ is the kinetic energy of the ionic
c.m. motion (usually negligible).


%
Most output files can immediately be used in standard plot software, as
e.g. GNUPLOT, by simply plotting the wanted column against time in the
first column. Exception are the files \texttt{pcharges.<name>},
\texttt{pposion.<name>}, \texttt{pvelion.<name>},
\texttt{pforce.i.<name>}, and \texttt{plforce.i.<name>}.  Here, time
is still running in the first column, but in blocks where several
subsequent lines refer to the same time to work up the many
observables printed, e.g. all the ionic positions in
\texttt{pposion.<name>}. Preprocessing is needed before plotting
information from these files.
%
Finally, the file \texttt{pMP.<name>} is not designed for immediate
use. It supplies all information for post-processing with a separate
program, see \scn{sec:PES}.
			
\subsection{RTA dynamics}
\label{sec:RTA}

\subsubsection{The namelist \texttt{RTAparams}}
\label{sec:rtaparams}

%\begin{longtable}{|p{3.1cm}|p{1.3cm}|p{10.2cm}|}%[htbp]
\begin{table}
  \caption{Parameters in the \texttt{RTAparams} namelist in
    \texttt{for005.<name>} that control the RTA
    procedure}\label{tab:dyn-input-params-rta}
%\endfirsthead
\begin{tabular}{|p{2.55cm}|p{1.4cm}|p{0.90cm}|p{8.3cm}|}%[htbp]
  \hline
  \multicolumn{4}{|c|}{namelist \texttt{RTAparams}}\\
  \hline
  \texttt{rtasigee}
  &$\sigma_{ee}$ [a$_0^2$]&\tgreen{$\emptyset$}& In-medium $e^-e^-$-cross
  section, used in Eq. (\ref{cpc:eq:relaxtime})\\ 
  \hline
  \texttt{rtars} & $r_s$ [a$_0$]&\tgreen{$\emptyset$}& Effective Wigner-Seitz
  radius used in Eq. (\ref{cpc:eq:relaxtime})\\ 
  \hline
  \texttt{rtatempinit} & [Ry]&\tgreen{0.0}& Initial electron temperature to
  stabilize first RTA steps, see text\\ 
  \hline 
  \texttt{rtaferminit} & [Ry]&\tgreen{0.5}& Upper bound for temperature in
  the first step   adjusting the closest Fermi distribution
  Eq. (\ref{cpc:eq:Fermi})\\ 
  \hline
  \texttt{rtamu} &$\mu$&\tgreen{20}& Lagrange multiplier for
  the quadratic density constraint in the DCMF Hamiltonian
  Eq. (\ref{cpc:eq:hDCMF}).\\ 
  \hline 
  \texttt{rtamuj}  &$\mu_j$&\tgreen{2}& Lagrange multiplier
  for the quadratic  current constraint in the DCMF Hamiltonian
  Eq. (\ref{cpc:eq:hDCMF}).\\ 
  \hline 
  \texttt{rtaeps}
  &$\delta^\mathrm{(RTA)}$&0.1& Stepsize in the damped
  gradient procedure (\ref{cpc:eq:dampstep}) \\ 
  \hline
  \texttt{rtae0dmp} &
  $E_{0,\mathrm{damp}}^\mathrm{(RTA)}$ [Ry]&\tgreen{e0dmp}& Damping parameter 
  for the damped gradient procedure(\ref{cpc:eq:dampstep}) in
  DCMF\\ 
  \hline 
  \texttt{rtaDt1} & [Ry]&\tgreen{1D-3}& maximum temperature step
  in searching Fermi distribution\\  
  \hline 
  \texttt{rtasumvar2max} & $\epsilon_0$ [Ry]&\tgreen{1D-4}& Termination
  criterion for s.p. variance  in the DCMF loop, see
  \fig{cpc:fig:dcmf} \\ 
  \hline 
  \texttt{rtadiffenrmax} & $\epsilon_2$ [Ry]&\tgreen{$\epsilon_0$}& Termination
  criterion for s.p. variance  in the DCMF loop
  \\
  \hline
  \texttt{rtaExitErr} &$\epsilon_\rho$ &\tgreen{1D-2}& Termination
  criterion for relative density in the DCMF loop, see \fig{cpc:fig:dcmf}  \\ 
  \hline
  \texttt{rtaExitErrj} &$\epsilon_j$ &\tgreen{5$\epsilon_\rho$}& Termination
  criterion for relative current in the DCMF loop \\ 
  \hline 
  \texttt{rtaerr1} &&\tgreen{100$\epsilon_\rho$}&
  threshold on relative difference-density to start the temperature correction in
  the DCMF loop, see \fig{cpc:fig:dcmf}\\ 
  \hline 
  \texttt{itmaxDCMF} &&\tgreen{2000}& Maximum number of DCMF
  iterations\\ 
  \hline
\end{tabular}
\end{table}
%\end{longtable}
\tab{tab:dyn-input-params-rta} lists the input parameters specific to
the RTA procedure. Most parameters are explained in
\scn{cpc:sec:summaryRTA}, however in rather compact manner. We
complement that here with more details where necessary.
% 
There are two parameters which quantify the physical properties of
dissipation, namely in-medium electron scattering cross section
$\sigma_{ee}\equiv$\texttt{rtasigee} and the effective Wigner-Seitz
radius of the electron cloud $r_s\equiv$\texttt{rtars}. They are
explained in section \ref{cpc:sec:summaryRTA}.
All other parameters are of more technical nature. The first group of
them concern initialization of an RTA step:
\begin{description}

 \item{\texttt{rtatempinit}} sets an initial (small) electron
   temperature.  This is applied already at the beginning of dynamics.
   This serves to enter RTA with a Fermi distribution having small,
   but finite width. The parameter serves a double purpose.  The value
   \texttt{rtatempinit/Tfrac} is used as lower bound for the search of a
   temperature \texttt{SUBROUTINE fermi1}.
   The fraction parameter is set to \texttt{Tfrac=10} in the header
   of \texttt{rta.F90}. 
   \\
   It is highly recommended to use the same temperature already
   in the static calculation. This is achieved by
   setting the parameter \texttt{temp}, see table
   \ref{tab:input-params-sys-choice}, to \texttt{temp=rtatempinit}.
   This avoids spurious excitation by suddenly changing temperature.

  \item{\texttt{rtaferminit}} is upper bound for the temperature in
    the first call to \texttt{SUBROUTINE fermi1} which determines
    a Fermi distribution of occupation numbers reproducing a given
    energy. The default setting is robust such that one rarely needs to 
    enter it explicitly in the input file.
\end{description}
The next group of parameters regulates the DCMF step:
\begin{description}

  \item{\texttt{rtamu}$\equiv\mu$,\texttt{rtamuj}$\equiv\mu_j$} 
  are the weight factors
  in front of the quadratic constraint in the DCMF Hamiltonian,
  see eq. (\ref{cpc:eq:hDCMF}). A larger value exerts more pressure to
  fulfill the density and current constraint, however, at the price
  of a numerically more critical DCMF-Hamiltonian. One has to find a 
  proper balance between strength of constraint and the parameters
  \texttt{rtaeps} and \texttt{rtae0dmp} of the DCMF step. The default
  values for $\mu$ and $\mu_j$ represent a working
  compromise. You may try larger or smaller values. But it is crucial
  to make $\mu_j$ much  smaller than $\mu$. A ratio of 10 is
  highly recommended.

  \item{\texttt{rtaeps},\texttt{rtae0dmp}} are the damping parameters
  for the damped gradient step in the solution of the static 
  Kohn-Sham problem. One may wonder why not using the same values as
  for the \texttt{epswf} and \texttt{e0dmp} from the standard static
  Kohn-Sham  step. The point is that density and current constraints
  change the spectral properties of the mean-field Hamiltonian which,
  in turn, may require different stepping parameters. The default
  \texttt{rtaeps=0.1} is a conservative choice. One may use larger
  values together with \texttt{rtae0dmp}$<$\texttt{e0dmp}, and,
  of course, both parameters may be reconsidered when changing
  \texttt{rtamu} and \texttt{rtamuj}.

  \item{\texttt{rtaDt1}} is the allowed change of temperature in
  early calls to the Fermi routine. This serves to stabilize the
  DCMF iterations because fast changes of occupation numbers
  in early DCMF iterations can delay, or even inhibit, convergence.
  The default setting is rigid. One may explore larger values.

  \item{\texttt{rtaerr1}}$\equiv\epsilon_1$ is the threshold for the
    average error on density, see eq. (\ref{eq:critrho}),
    below which recomputation of the Fermi occupation distribution
    in \texttt{SUBROUTINE fermi1}
    is activated.

\end{description}
Finally comes the group of criteria for terminating the DCMF
iterations:
\begin{description}

  \item{\texttt{rtasumvar2max}}$\equiv\epsilon_0$ is the maximal
  allowed value for the average variance of s.p. energies.
  The solution must fulfill
  \begin{equation}
    \sqrt{
    \frac{\sum_\alpha w_\alpha\left(
          \langle\varphi_\alpha|\hat{h}_\mathrm{DCMF}^2|\varphi_\alpha\rangle
          -
          \langle\varphi_\alpha|\hat{h}_\mathrm{DCMF}|\varphi_\alpha\rangle^2
          \right)}
         {\sum_\alpha w_\alpha}
         }
    <
    \epsilon_0
  \end{equation}
  before ending the DCMF iterations.
 
  \item{\texttt{rtadiffenrmax}}$\equiv\epsilon_0$ is the maximal
  allowed deviation of total s.p. energies. The solution must fulfill
  \begin{equation}
    \left|E_\mathrm{s.p.}^\mathrm{DCMF}
          -
          E_\mathrm{s.p.}^\mathrm{goal}\right|
    <
    \epsilon_2
    \;,\;
    E_\mathrm{s.p.}^{\mbox{}}
    =
    \sum_\alpha w_\alpha\varepsilon_\alpha
    \;.
  \label{eq:misenergy}
  \end{equation}
%  One may wonder why using here the total deviation and not
%  the average one. This is because energies converge faster
%  than variances of energy. Using the total energy, but the average
%  variance, allows to chose comparable
%%  values for $\epsilon_2$ and $\epsilon_0$.

\item{\texttt{rtaExitErr}}$\equiv\epsilon_\rho$ checks the
  quality of reproduction of local density with the criterion
  \begin{equation}
    \frac{\int d^3r\,\left|
     \rho_\mathrm{DCMF}(\mathbf{r})
     -
     \rho_\mathrm{goal}(\mathbf{r})
    \right|}
    {\int d^3r\,\rho_\mathrm{goal}(\mathbf{r})}
    <\epsilon_\rho
    \;.
    \label{eq:critrho}
  \end{equation}
  This quantity is dimensionless and measures the relative deviation  
  of density. It is thus rather independent of the systems scales.
  Similar values of $\epsilon_\rho$ should apply for all systems.

  \item{\texttt{rtaExitErrj}}$\equiv\epsilon_j$ checks the quality
  of reproduction  of the current with
  \begin{equation}
    \frac{\int d^3r\,\left(
     \mathbf{j}_\mathrm{DCMF}(\mathbf{r})
     -
     \mathbf{j}_\mathrm{goal}(\mathbf{r})
    \right)^2}
    {\int d^3r\,\left(
     \mathbf{j}_\mathrm{goal}(\mathbf{r})
     \right)^2}
    < 
    \epsilon_j
    \;.
  \label{eq:squarecurr}
  \end{equation}
  This is also relative deviation. A possible problem comes here in
  regimes of very small currents where we may require too much
  because a large error on a small quantity is harmless. So far, we
  did not run into trouble with that.

\item{\texttt{itmaxDCMF}} is the maximum number of iterations.
  It overrules the other criteria and terminates even if density,
  current, or energy variance has not yet reached its goal.
  An exception if the mismatch of energies (\ref{eq:misenergy}).
  If this has not yet reached a satisfying level,
  a second round of iterations is appended up to
  twice \texttt{itmaxDCMF}.

\end{description}
A few words are in order about chosing termination criteria.  DCMF is
a highly demanding task. One should not expect a quality of
convergence as one usually sees in unconstrained Kohn-Sham
equations. In particular, the current is hard to tame. One has the
choice between high demands and awfully long iterations or an
affortable number of DCMF iterations with moderate quality. We opt for
the latter choice. This is legitimate as true thermal states are
anyway somewhat noisy. The default setting in table
\ref{tab:dyn-input-params-rta} give you an idea of what to
expect. Mind that a larger deviation for the currents is advisable
because the current constraint is particularly obstinate. 

When chosing convergence criteria too rigid, it can happen
that iterations fluctuate and terminate with a mismatch in energy.
In that case, a warning is printed in \texttt{peqstat.<name>}.
This is acceptable if it happens rarely with only small mismatches
because the final energy correction can cope with  that.
But be careful, the choice of convergence parameters has to be revised if
that  happens regularily.

Typical values for the RTA parameters in
\tab{tab:dyn-input-params-rta} can be found in the following
subdirectories of directory
\texttt{examples}: \texttt{H2O/laser/Onres-rta},
\texttt{H2O/laser/Offres-rta}, and \\
\texttt{Na11p/boost-XXX-rta/prta.Na11p}
with \texttt{XXX=015, 020, 025}.
We reproduce here two RTA input sequences, one from covalent H$_2$O
(left) and one from metallic Na$_{11}^+$ (right):
\\[8pt]
\begin{center}
\begin{minipage}{0.4\linewidth}
\begin{Verbatim}[frame=single,label=RTAparams in for005.H2O,fontsize=\normalsize]
&RTAparams
itmaxDCMF=800,
rtamu=20.0d0,
rtamuj=2.0d0,
rtaeps=0.2D0,
rtae0dmp=2D0,
rtasigee=0.91,
rtars=1.0,
rtatempinit=0.005D0,
rtasumvar2max=1D-3,
rtaExitErr=1D-2,
rtaExitErrj=3D-2,
rtadt1=0.08,
&END
\end{Verbatim}
\end{minipage}
\hspace{1em}
\begin{minipage}{0.4\linewidth}
\begin{Verbatim}[frame=single,label=RTAparams in for005.na11p,fontsize=\normalsize]
&RTAparams
itmaxDCMF=500,
rtamu=20D0,
rtamuj=2D0,
rtaeps=0.2D0,
rtae0dmp=0.9D0,
rtasigee=6.5D0,
rtars=3.7D0,
rtatempinit=1D-2,
rtartaexiterr=1D-2,
rtaexiterrj=3D-2,
sumvar2max=7D-3,
rtaerr1=0D0,
rtadT1=0.001D0,
rtaexiterr=0.01D0,
&END
\end{Verbatim}
\end{minipage}
\end{center}
The input for H$_2$O is shorter because it relies on defaults for
parameters not given explicitly. 

\paragraph{A note on obtaining PES} When RTA is activated the PES cannot be computed, because the wave functions are mixed and receive random phases in the DCMF diagonalization process. There is a need to correct this in order to get a continuous signal at the measuring points. Supplemental corrections will be added in a future release. 
		
\subsubsection{RTA output files}
			
\begin{table}[htbp]
  \caption{Output files that are specific to a RTA-enabled calculation}\label{tab:dynamic-output-files-rta}
  \begin{tabular}{|p{4.3cm}|p{10.1cm}|}
    \hline
    \texttt{convergenceRTA.<name>} &  Protocol of convergence
    criteria along  DCMF iterations \\
    \hline
    \texttt{peqstate.<name>} &  Final convergence criteria of the DCMF
    process
% current iteration number, cycles to convergence, variance, residual
% err. on density, residual err. on current, parameters mu, muj,
% energy achieved
\\
    \hline
    \texttt{prta.<name>} & Time evolution of entropy, laser energy,
    $\mu^\mathrm{(eq)}$, and $T^\mathrm{(eq)}$\\
    \hline
    \texttt{pspeed.<name>} & Prints at each RTA step, along x axis, the reference density (spin up and down), achieved density (spin up and down), target x current, achieved x current \\
    \hline
  \end{tabular}
\end{table}
%
\tab{tab:dynamic-output-files-rta} summarizes the output files written
by the RTA procedure. These are mostly protocols of RTA iterations
and their convergence. Particular information of physics interest is
given in the file \texttt{prta.<name>}, namely entropy
Eq. (\ref{cpc:eq:entropy}) of the
occupation distribution and temperature as well as chemical potential
of the instantaneous equilibrium distribution Eq. (\ref{cpc:eq:Fermi}).

\clearpage

\section{Dynamic examples}
\label{sec:dyn_examples}
	
\subsection{Ionic cooling}

Finding the ionic coordinates that minimize the system energy is easy
when there are only two ions, as exemplified in Sec.
\ref{sec:Na2}. But for more than two ions it grows quickly tedious.
%			
To solve this problem the code offers three strategies, dynamic
cooling by pseudo-dynamics (\texttt{icooltyp=1}), steepest descent
downhill the ionic energy surface (\texttt{icooltyp=2}), and simulated
annealing with Monte-Carlo techniques (\texttt{icooltyp=3}).  We show
here an example for dynamic cooling.  It works by starting a dynamic
TDLDA-MD calculation, see \scn{cpc:sec:mf}. Initially, the ions are at
rest, but in a possibly uncomfortable configuration.  Naturally, ions
will start to move. Like in a mass-spring system, the ions will be
accelerated at first, will then reach their maximum velocity once they
pass close to their equilibrium positions, and then start to slow
down.  This is the point where the ion-cooling algorithm comes into
play.  It monitors the total ionic kinetic energy and interrupts the
dynamics as soon as it observes the kinetic energy turning down. All
ionic velocities are then reset to zero and TDLDA-MD is restarted from that
configuration. This process is repeated until the gain in ionic
kinetic energy is negligible after a restart. Then we have obviously
reached a stable situation, a minimum in the ionic energy surface. It
may be the true ground state or some isomeric minimum. Thus one should
repeat the whole cooling procedure from different initial ionic
configurations. With some patience one will so collect enough insight
into the isomeric landscape of a system.  Finally, a word about
naming. The propagation is called {\it pseudo dynamics} because it
intervenes the free ionic dynamics by resetting velocities,
even-though it is technically a dynamical calculation.

As test case of some complexity we take the  Na$_8$ cluster, to
increase the number of ionic degrees of freedom.  The corresponding
input files are located in
\begin{verbatim}
  $QDD_ROOT/examples/Na8-ioncool/
\end{verbatim}
The following eight dynamics parameters have to be set in the
\texttt{DYNAMIC} namelist in the \texttt{for005.<name>} input file:
\begin{verbatim}
  ionmdtyp = 2
  icooltyp = 1
  ifredmas = 1
  nabsorb = 0, jesc = 0
  itmax = 25000
  jpos = 10, jvel = 10	
\end{verbatim}
Make sure that \texttt{itmax} is set to a sufficiently large  enough value
for the cooling process to finish. Nonetheless, it is a good idea to
activate saving of the state by setting \texttt{isaved} appropriately.
This allows to continue the procedure from the saved state if it was
found not sufficiently converged within the given \texttt{itmax}.  To
estimate the amount of time steps, you need to have an idea about the
time scale of ionic motion and to recall that the full propagation
time is given by $t_\mathrm{max}=$\texttt{dt1}$*$\texttt{itmax}.  But
that is guesswork anyway, and one has to monitor the run to either
interrupt a run early or continue from a saved state.  Also it is
better not to use any absorbing points (\texttt{nabsorb=0}, this
automatically implies \texttt{jesc=0}), and to use reduced masses for
the ions in the dynamics (\texttt{ifredmas=1}). The latter reduces the
time scale for ionic motion and thus shortens the time it takes for
the cooling. You can watch the cooling progress  by looking at the
kinetic energy of the ions in the file
\texttt{penergies.Na8-ioncool}. 
%
\begin{figure}[tbph]
  \centering
  \includegraphics[width=0.85\linewidth]{ion-cooling}
  \includegraphics[width=0.85\linewidth]{ion-cooling-zoom}
  \caption{Kinetic energy of the ion cores of Na$_8$ as a function of time. After about 69~fs the energy is well below the set variance of \texttt{epsoro}=$8\!\times\! 10^{-7}~\mathrm{Ry}$.}\label{fig:ion-cooling}
\end{figure}
%
Figure~\ref{fig:ion-cooling} show a plot of this energy using columns
1 and 4. One spots nicely the cooling steps where the kinetic energy
is reset to zero. One observes also that the amount of kinetic energy
extracted decreases from one cooling step to the next until from
$\sim$70~fs onward the kinetic energy of the ions is well below the
electronic variance of \texttt{epsoro}=$8\!\times\!
10^{-7}~\mathrm{Ry}$. At this point the calculation is stopped. In
fact, the limit of $10^{-7}~\mathrm{Ry}$ is extremely demanding. The
limit for ionic cooling can be chosen independently from the
electrons. This depends on the application one has in mind after
cooling.

In case, one want to scrutinize the cooling dynamics more deeply, one
can  visualize positions and velocities of the individual ions from
the files \texttt{pposion.Na8-ioncool} and
\texttt{pvelion.Na8-ioncool}. Recall that during the calculation, each
\texttt{jpos} and \texttt{jvel} iterations the time, current position
and velocity are written to this file, per time step, one line for
each ion. To plot this with, e.g., GNUPLOT one can take advantage of
the command \texttt{every N::k}, where \texttt{N} is the number of
ions in the cluster and \texttt{k} runs from \texttt{0} to
\texttt{N-1}.

Thus far the realistic test case Na$_2$. For those who want to gather their own
experience, it might be insightful to repeat this calculation for
Na$_2$ where the inter-sodium distance in the
\texttt{for005ion.Na2-gs} is initially deliberately increased or
decreased, and then compare it to the value of the equilibrium
distance found in the previous section by the manual method.


\subsection{Applying a boost: the excitation spectrum of Na$_\mathsf{41}^+$}
\label{eq:applyboost}

In this section, we present an example for computing the optical
response of a molecule as explained in \scn{cpc:sec:specan}. The
strategy is to apply an initial boost to the electronic wave functions
and then to record the dipole oscillations of the electron cloud. The
dipole moment is then Fourier-transformed from the time domain to
the frequency domain to obtain the eigenfrequencies of the system \cite{Cal97b}.
			
The input files for this example are can be found in
\begin{verbatim}
  $QDD_ROOT/examples/Na41p/boost/
\end{verbatim}
Three parameters control the amplitude of the boost
\texttt{centfx,centfy,centfz} (see
\tab{tab:dyn-input-params-instantaneous}), one parameter for each
direction. There are also three protocols in the vector
$\mathbf{D}(t)$. This would allow to map altogether the full dipole
response tensor. This tensor can be assumed to be diagonal if the
principal axes of the molecule are aligned with the axes of the
numerical box. In that case, one can excite all three direction
simultaneously and read off the $x$-, $y$-, $z$-spectra directly from
$D_x(t)$, $D_y(t)$, and $D_z(t)$. That is the strategy we follow here.



A few more precautions are in order.  
%
When applying a boost, make sure it is not too intense. Too large a
boost will induce a sizable ionization, see column 3 in
\texttt{pescel.<name>}.  You can also check how much energy is
absorbed by looking in \texttt{infosp.<name>} at the energy at the end
of the static calculations and at the beginning of dynamics.
			
To obtain the dipole spectrum, an additional program `\texttt{spectr2}'
needs to be compiled and used on the dipole response file
\texttt{pdip.<name>}. The program can be found in
\begin{verbatim}
  $QDD_ROOT/src/auxiliary/spectr2.F90
\end{verbatim}
To build the program do
\begin{verbatim}
  $ cd $QDD_ROOT/src/auxiliary/
  $ make spectr
\end{verbatim}
After this is done one simply runs
\begin{verbatim}
  $QDD_ROOT/bin/spectr2 < pdip.<name> > spectrum.dat
\end{verbatim}
The dipole spectrum can then be visualized by plotting columns 4, 7
and 10 (the real parts of the Fourier transformed signal in the
$(\mathbf{\hat{e}}_x,\mathbf{\hat{e}}_y,\mathbf{\hat{e}}_z)$-directions)
versus column 2 (frequency $\omega$ in eV).
		

\begin{figure}[htbp]
  \centering
  \includegraphics[width=\linewidth]{spectrum_orange-blue}
  \caption{Top panel: real-time dipole signal for Na$_{41}^+$ after a
    small boost in the $(x,y,z)$-direction. Bottom panel: Fourier
    transform of the same dipole signal. The vertical dotted line
    indicates the ionization potential of
    Na$_{41}^+$.}\label{fig:spectrum}
\end{figure}
Figure \ref{fig:spectrum} shows the results for the case of
Na$_{41}^+$, dipole signals in time domain as well as in frequency
domain. The spectrum (lower panel) shows dominantly the fragmented
plasmon peak and a few minor side peaks. The overall position of the
plasmon peak is nearly the same for all three modes. This indicates
that Na$_{41}^+$ is close to spherical shape. The modes in $x$- and
$y$-direction are even exactly degenerated which can be connected with
the symmetry of the cluster. The example indicates that optical
response and cluster shape are closely connected, see
e.g. \cite{Hee93}.


\subsection{Applying a laser excitation to H$_\mathsf{2}$O}

In this example we will excite the water molecule by applying a
resonant and off-resonant LASER pulse. The example input files can be
found in
\begin{verbatim}
  $QDD_ROOT/examples/H2O/laser
\end{verbatim}
			
The most important parameters that control the laser excitation are
the following
\begin{verbatim}
  itft = 3
  deltat = 36.0
  e1x = 0.0
  e1y = 0.0
  e1z = 1.0
  omega = 0.79
  e0 = 0.08
\end{verbatim}
To be on-resonance with one of the dipole oscillation modes in the
$z$-direction we need an \texttt{omega} of ~0.79~Ry which corresponds
to 10.7~eV. This is slightly off-resonance on the blue side of the
peak. The reason for this is that during the pulse, the ensuing
ionization will cause the dipole spectrum to shift slightly to the
blue. Just enough for the laser pulse to end up in the resonance
region. In this case the shape of the pulse envelope is a $\sin^2$,
the pulse length is 36~fs and the pulse polarization is in the
$z$-direction. A list of all the excitation parameters can be found in
\tab{tab:dyn-input-params-excitation}.
			
For the off-resonant case, we simply shift the value of \texttt{omega}
farther away from resonance to another value, here to 0.84~Ry
(11.4~eV).
		
%\paragraph{Results and observables}	
The resulting time evolution of three basic observables, ionization,
dipole moment, and energy absorbed from the photon pulse, are shown in
\fig{cpc:fig:H2Oonoff} of the paper and discussed therein.

\subsubsection{Calculating Photo-electron Spectra  (PES)}
\label{sec:PES}

Pure TDLDA calculations, with and without SIC, allow to compute PES,
i.e. the kinetic-energy spectra of emitted electrons, see
\scn{cpc:sec:pes}.  That requires, of course, to activate absorbing
boundary conditions by setting \texttt{nabsorb>0}. A proper compromise
has to be found for this parameter. Large values are desired to
improve absorption. These, however, require larger grids and so
enhance the computational expense. A minimum value is
\texttt{nabsorb=4}, a good compromise is often \texttt{nabsorb=8},
and critical cases may ask for even larger values.

Next comes the choice of the shape of the absorbing bounds
(rectangular, spherical, or ellipsoidal). Cleanest results are usually
obtained with spherical absorbing zones, \texttt{ispherabso=1}.

Then one has to decide how many measuring points one wants to set.
This is done by the parameters \texttt{nmptheta} and \texttt{nmpphi}
which define the measuring points on a grid of the unit sphere of
emission directions, see 
\tab{tab:dyn-input-params-observables2} and discussion thereof.
Taking a small or large number of measuring points has negligible
effect on computing time. It is a matter of storage space. Mind that
the values of all s.p. wave functions at the measuring point are written
to the file \texttt{pMP.<name>} in each \texttt{jMP}th time step.
This is a small problem in small systems, but can grow space consuming
for a large molecule like C$_{60}$ with its 240 electrons.

Finally, we activate writing the information for later PES analysis by
setting the corresponding write frequency \texttt{jMP}. Its choice
determines the span of kinetic energies, we can evaluate. The time
difference  between two writings is 
%
$\Delta t_\mathrm{MP}=\delta t*$\texttt{jMP}
%
and the maximum kinetic energy after Fourier analysis is
$\varepsilon_\mathrm{kin,max}=\hbar\pi/\Delta t_\mathrm{MP}$. To be
on the safe side, we should go only up to half that value, i.e.  we
can expect safe results up to energy 
%
$\hbar\pi/(2\Delta t_\mathrm{MP})$.
%
This inhibits automatically  large \texttt{jMP}. Values of order
1...10 are typical.
%
The spectral resolution is determined by the length of the simulation
$t_\mathrm{max}=\delta t*$\texttt{itmax} as
$\Delta\varepsilon_\mathrm{kin}=2\pi\hbar/t_\mathrm{max}$.

Having then run the code successfully to the end, we use the protocol
file \texttt{pMP.<name>} thus generated for further processing. This is done 
in two steps. The first post-processing is done  with the code 
in the file \texttt{analyze-MP.f90} found 
in the directory \texttt{\$QDD\_ROOT/src/auxiliary}.
%As this code has to be edited, it is useful to copy it into
%the working directory where the  \texttt{pMP.<name>} resides.
%Before compilation, there are three parameters in the source code to
%be adjusted:
%\begin{description}
%  \item[\texttt{name}:] the \texttt{<name>} of the calculation as
%    given in \texttt{for005} and used as qualifier of the output files.
%  \item[\texttt{kpoints}:] the number of measuring points. It
%    can be read off from the header of the \texttt{pPES.<name>}.
%  \item[\texttt{kstate}:] the number of s.p. states. It
%    can be read off from the header of the \texttt{pPES.<name>}.
%\end{description}
%After editing \texttt{analyze-MP.f90}, this Fortran file is compiled
This code has to be compiled and executed. This means, e.g., in
  case of the GNU Fortran compiler under Linux to issue
  \texttt{gfortran -O3 \$QDD\_ROOT/src/auxiliary/analyze-MP.f90}
  followed by \texttt{./a.out}.  The executable asks first for the
  qualifier of the \texttt{pMP.<name>} file. Then it will read the
  phase information from \texttt{pMP.<name>}, process it, and produce
  an output file \texttt{pPES.<name>}. The number of measuring points
  as well as the whole table of measuring points is taken from
  \texttt{pMP.<name>} and copied to the header of
  \texttt{pPES.<name>}.  In the rows after the header, the file
\texttt{pPES.<name>} contains in first column the kinetic energy (in
Ry) and the subsequent columns the PES at each measuring point as:
\begin{displaymath}
    \epsilon_{kin}/\mathrm{Ry} \quad 
    \sum_i\left|\tilde\varphi_i\right|^2 \text{at MP}_1 
    \quad \text{at MP}_2 \quad 
    \text{at MP}_3 \quad \ldots \quad \text{at MP}_\mathrm{jMP}
\end{displaymath}
The relation between angles $\theta$ and $\phi$ with the number along the
columns is given in the header of  \texttt{pPES.<name>}.
%
%We exemplify the relevant lines in the header of
%  \texttt{analyze-MP.f90} for the test case
%  H$_2$O from \texttt{\$QDD\_ROOT/examples/H2O/laser/pespad}.
%\begin{verbatim}
%! Parameters for input                                            
%INTEGER, PARAMETER :: ktimes=170000     ! max. nr. of time points -- to be set
%INTEGER :: kpoints=92                   ! nr. measuring points 
%INTEGER :: kstate=8                     ! nr. states                   
%CHARACTER (LEN=13) ::  name = 'H2O'     ! qualifier                  
%\end{verbatim}
%The freshly edited Fortran file can then be compiled and executed.
%, e.g., as
%\textcolor{black}{Again, this
%    Fortran file can be compiled and executed, e.g., by a tool
%    provided with the QDD package.  The latter proceeds with the steps
%    (starting in the sub-directory \texttt{src/auxiliary})
%\begin{verbatim}
% $ make analyze-PES
% $ cd </path/to/pPES.<name>-file/>
% $ $QDD_ROOT/bin/analyze-MP && $QDD_ROOT/bin/analyze-PES
%\end{verbatim}
%\begin{verbatim}
%  $ <compiler> <options> -o analyze-MP $QDD_ROOT/src/auxiliary/analyze-MP.f90
%  $ ./analyze-MP
%\end{verbatim}
The routine \texttt{\$QDD\_ROOT/src/auxiliary/analyze-MP.f90} contains
in its headers the setting for the grid of kinetic energies
in the output \texttt{pPES.<name>} in terms of energetic grid spacing
and number of grid points. If one wants to have other setting than
provided (spacing 0.01 Ry and number of grid points 1000) one has to
edit the file and compile again.

One can now plot the PES in the specific directions 
directly from \texttt{pPES.<name>}.  
However, in the case that we use a rather fine mesh of measuring points, we are
confronted with a swamp of data. There, we are often particularly
interested in the angular integrated total PES. This is achieved by
running a second routine contained in the file
\texttt{analyze-PES.f}, again found
in the directory \texttt{\$QDD\_ROOT/src/auxiliary}.
%and to be copied into the working directory.  Two parameters
%are to be adjusted before compilation:
%\begin{description}
%  \item[\texttt{name}:] the \texttt{<name>} name of the calculation as
%    given in \texttt{for005} and used as qualifier for most output
%    files.
%  \item[\texttt{kpoints}:] the number of measuring points, same as
%    used in \texttt{analyze-MP.f90}.
%\end{description}
%This reads, e.g., for the above example of H$_2$O:
%\begin{verbatim}
%      parameter(kpoints=92)       ! number of meas. points            
%      parameter(komega=3000)      ! max. nr. of frequencies in PES 
%      parameter(name='H2O')       ! set up the name   
%\end{verbatim}
This file \texttt{analyze-PES.f} is also to be compiled and
  executed. The executable asks first for the qualifier of the 
\texttt{pMP.<name>} file. Then it produces a couple of output files.
Most important is the file
\texttt{iPES.<name>} which contains integrated properties from
the PES:\\
\hspace*{1em}column 1: energy (in Ry),
\\
\hspace*{1em}column 2: total PES (arbitrary units),
\\
\hspace*{1em}column 3: quadrupole anisotropy ($\propto P_2(\cos\theta)$),
\\
\hspace*{1em}column 4: hexadecapole anisotropy ($\propto P_4(\cos\theta)$),
\\
\hspace*{1em}column 6: sixth order anisotropy ($\propto P_6(\cos\theta)$).
\\
%
The most interesting observable is here the total PES in column 2. A
logarithmic $y$-axis is recommended because the PES can span enormously
different values. For an example see \fig{cpc:fig:PES} in the paper.
		
The routine in file \texttt{analyze-PES.f} produces a couple of
further output files. The file \texttt{distavphi.<name>} contains the
$\phi$-averaged PES in the plane of energy and angle $\theta$
appropriate for a color-map plot in \texttt{gnuplot}.  The file
\texttt{distavphi-map.<name>} contains the same information, but
plotted in a plane the two outgoing momenta $p_z$ along the $z$-axis
and $p_\perp$ orthogonal to it, again suited for color-map plotting.
The file \texttt{parametric.<name>} plots also the $\phi$-averaged PES
in the plane of energy and angle $\theta$, but here with a smooth
reconstruction from $\theta$ average and quadrupole anisotropy which
is more robust in case the given grid in $\theta$ and $\phi$ was
sparse. Finally, the file \texttt{velomap.<name>} shows the same
smoothed information as \texttt{parametric.<name>}, but in the plane
of $p_z$ and $p_\perp$.
\begin{figure}[tb]
  \centering
  \includegraphics[width=\linewidth]{H2O-PESandmore.jpg}
  \caption{Different representations of PES for H$_2$O irradiated by a
    laser pulse with characteristics as indicated. The laser
    polarization is along the $z$ direction.  Upper panel: The angular
    integrated PES versus kinetic energy $E_\mathrm{kin}$ of the emitted
    electron (same as the PES in figure \ref{cpc:fig:PES}).  Lower left:
    The $\phi$-integrated PES in the plane of $E_\mathrm{kin}$ and
    angle $\theta$.  Middle left: The smoothed $\phi$-integrated PES
    in the plane of $E_\mathrm{kin}$ and angle $\theta$.  Lower right:
    The $\phi$-integrated PES in the plane of outgoing momenta $p_z$
    and $p_{x,y}\equiv p_\perp$.  Middle right: The smoothed
    $\phi$-integrated PES in the plane of outgoing momenta $p_z$ and
    $p_{x,y}\equiv p_\perp$.  }\label{fig:H2O-PESandmore}
\end{figure}
Fig.~\ref{fig:H2O-PESandmore} illustrates the use of the different
output files for visualizing the PES. Test case is the same as used
for Fig.~\ref{cpc:fig:PES}, namely H$_2$O excited by a laser
pulse. The upper panel repeats the fully angular integrated PES from
Fig.~\ref{cpc:fig:PES}. The other four panels illustrate four ways of
plotting the $\phi$-integrated PES in their two remaining dependences.
The smoothed PES just resolve the differences in forward-sideward
emission while the full PES (lower panels) can show more angular
fluctuations.  The left panels show clearly the angular distributions
for a given energy while the right panels in terms of outgoing
momenta, or velocities respectively, provide an imagination of how the
electrons flow apart in the different directions.

\subsection{Dual pulses and pump-and-probe scenarios}

The profiles for laser pulses provide also dual pulses, see
Eq. (\ref{eq:doublepulse}). These are useful to design pump-and-probe
scenarios which are widely used for time-resolved analysis of ionic
motion in a molecule \cite{Zew94}. We demonstrate the capability here
with an example for the Na$_9^+$ cluster, comparing two different
delay times. The detailed input and output files are found in the
directory \texttt{examples/Na9p/PandP}. The important ingredient is
here the input for the laser pulse. It reads, for one of two
cases:\\[3pt]
\begin{minipage}{\linewidth}
\begin{verbatim}
itft=3, deltat=48.0, tnode=0.025,
e1x=0.0, e1y=0.0, e1z=1.0,
omega=0.169, e0=0.014,
e0_2=0.0034, omega2=0.169, tstart2=200.0, deltat2=48.0,
e2x=0.0, e2y=0.0, e2z=1.0,
\end{verbatim}
\end{minipage}
\\[3pt] The first three lines define the pump pulse. We see that it
has frequency $\omega_\mathrm{las}=0.169~{\rm Ry}=2.3$~eV, pulse
length $T_\mathrm{pulse}=48$ fs, field strength $E_0=0.014$ Ry/a$_0$,
and linear polarization along $z$. The small
\texttt{tnode}$=t_0=0.025$ fs serves to stay away a tiny bit from the
initialization at $t=0$.  The last two lines stand for the probe
pulse (and are omitted if only a pump pulse is asked for). It starts
at \texttt{tstart2}~$=t_0^{(2)}=200$~fs and its envelope reaches
the peak value at $200+48/2$~fs. The envelope of the pump pulse has
its peak at $48/2=24$~fs.  The delay is defined from peak to peak and
it is thus $t_0^{(2)}=200$~fs for this case.

Figure \ref{fig:Na9p-PandP} shows the results for pure pump and two
different pump-and-probe pulses. 
%\begin{figure}[htbp!]
%  \centering
%  \includegraphics[width=\linewidth]{Na9p-PandP}
%  \caption{Illustration of a pump-and-probe scenario for Na$_{9}^+$.
%    Pump and probe pulses have the same frequency $\omega_\mathrm{las}=2.3$ eV and
%    pulse length $T_\mathrm{pulse}=48$ fs. The intensity of the pump
%    pulse is $I=1.6\times10^{12}$ W/cm$^2$ and of the probe pulse
%    $I=10^{11}$ W/cm$^2$.  Upper left: Time evolution of the ionic
%    r.m.s. radius after the pump pulse \MD{only}.  Lower left: Time evolution of
%    dipole \MD{moment in $z$-direction} for pure pump and pump-and-probe with two different
%    delays as indicated.  Lower right: Time evolution of the
%    ionization for pure pump and pump-and-probe with two different
%    delays as indicated. Note that the two delay times correspond
%    roughly to maximum and minimum of the ionic radius (upper left).
%  }\label{fig:Na9p-PandP}
%\end{figure}

\begin{minipage}{0.59\linewidth}
%\begin{figure}[htbp!]
  \centering
  \includegraphics[width=\linewidth]{Na9p-PandP}
%\end{figure}
\end{minipage}
\hfill
\begin{minipage}{0.4\linewidth}
  \captionof{figure}{Illustration of a pump-and-probe scenario for
    Na$_{9}^+$.  Pump and probe pulses have the same frequency
    $\omega_\mathrm{las}=2.3$ eV and pulse length
    $T_\mathrm{pulse}=48$ fs. The intensity of the pump pulse is
    $I=1.6\times10^{12}$ W/cm$^2$ and of the probe pulse $I=10^{11}$
    W/cm$^2$.  Upper left: Time evolution of the ionic r.m.s. radius
    after the pump pulse only.  Lower left: Time evolution of dipole
    moment in $z$-direction for pure pump and pump-and-probe with two
    different delays (identical to $t_0^{(2)}$) as indicated.  Lower
    right: Time evolution of the ionization for pure pump and
    pump-and-probe with two different delays as indicated. Note that
    the two delay times correspond roughly to maximum and minimum of
    the ionic radius (upper left).
  \label{fig:Na9p-PandP}}
\end{minipage}

The pump pulse (green curves in all panels)
produces rather quickly an ionization stage of one charge unit. The
Coulomb pressure thus generated induces significant ionic pulsations
with the typical frequency for ionic motion in Na clusters
\cite{Wop14aR}. The overall extension of a system has a direct impact
on the plasmon frequency which determines the electronic response to
an external photon field. 
%We place the probe pulses once at the
%maximum of the ionic radius (blue curves) and another at the minimum
%(red curves). 
We compare two different delay times of the probe pulse: the first one
(blue curves) is chosen so that the envelop peak of the probe pulse matches
the maximum of the ionic radius, while the second delay time (red curves) corresponds
to the minimum of the ionic radius, as indicated in the upper left panel.
The dipole responses in the direction of the laser polarization (lower left panel) are much different
in each case
because the plasmon frequency is lower at maximum radius, thus closer
to the photon frequency \cite{Sur00a}. As a consequence, we experience
a strong response whereas the probe pulse at the minimal radius has an
order of magnitude weaker response. The effect can be observed
experimentally from the extra ionization generated by the probe pulse, as shown in the
lower right panel: It is indeed dramatic. Thus one can map the ionic radius
oscillations by scanning the ionization for a dense series of probe
delay times~\cite{And02b}. %$t_\mathrm{delay}$ \cite{And02b}.

Experimental facilities have been much further developed to allow
meanwhile atto-second pulses with well controlled profile and delay.
What is an enormous effort at the experimental side shrinks to a mere
change of input parameters for theorists which allows them to apply
TDLDA in general and the QDD code in particular also for exploring
atto-second dynamics \cite{Bra19a}.  Note, however, that the often
used pump-and-probe setup with an IR pump and an XUV attopulse train
\cite{Wop14b} calls for more coding. This option will be available in
the next release of QDD.
			
			

\clearpage
\section{Some common mistakes}
\label{sec:mistakes}     

All the above shows that there are many numerical parameters which all
have to be chosen in appropriate ranges else things do not work out
properly.  In this section, we will address a few common misfits and
their effect on the observables. The idea is to illustrate what can go
wrong and how in the hope to make it easier to pinpoint a problem if
it appears.
	
		
\subsection{Too large time step}

\begin{figure}[htbp]
  \centering
  \includegraphics[width=\linewidth]{total-energy}
%				\includegraphics[width=\linewidth]{total-energy-zoom}
  \caption{Total energy as a function of time for H$_2$O, without any
    external excitation, for 3 common problem
    cases.}\label{fig:total-energy}
\end{figure}

\begin{figure}[htbp]
  \centering
  \includegraphics[width=\linewidth]{nesc}
  \caption{Ionization as a function of time for H$_2$O, without any
    external excitation, for 3 common problem cases.}\label{fig:nesc}
\end{figure}

\begin{figure}[htbp]
  \centering
  \includegraphics[width=1.0\linewidth]{dipole}
  \caption{Dipole response as a function of time for H$_2$O, without any external excitation, for 3 common problem cases.}\label{fig:dip}
\end{figure}

Figures \ref{fig:total-energy},~\ref{fig:nesc}~and~\ref{fig:dip}
illustrate what happens if the chosen time step is too large. The
total energy is not conserved and usually increases, often virtually
exploding at some time. After $\sim$100$\,\mathrm{fs}$, most of the
electrons are gone and spurious dipole moments are generated. This
mis-choice of time step causes by far the strongest effect on the
observables. The step from stable to unstable propagation is extremely
small as also documented in figure \ref{cpc:fig:Na9p-compsteps_2}.
			
\subsection{Numerical box too small}

A bit less severe but still strong enough to invalidate the
calculation is when too small a box is chosen in connection with
absorbing boundary conditions. In this case, the converged ground
state will not be dynamically stable. Once the dynamics is started the
electrons start to leak out of the box in quantities, the total energy
will increase and sooner or later, the ionic structure disintegrates
due to Coulomb pressure (if ionic motion was also enabled).
%			
A good rule of thumb to see if the box is big enough is to check the
value of the electron ground-state density $\varrho(\mathbf{r})$ at the
boundary of the box and make sure that
$|\varrho(\mathbf{r})\left|_\mathrm{boundary}\right. < 10^{-8}$.




In case of strong laser fields, a further effect comes into play.
Besides strong electron emission, the external field generates also
some electron dust. This dust can modify the dynamics due to its own
dynamical response. This is a valid physical effect. But it causes a
sensitivity to the box size because the extension of the dust, of
course, changes with the box volume.  Box size is most crucial if one
aims at effects from the ponderomotive motion of the emitted electron
cloud as done, e.g., in refs.~\cite{Li15a,Gao17a,Rei18e}.  However,
signals from the clusters volume, as dipole amplitude and energy
absoportion from the laser pulse, remain robust.  A bit more sensitive
is electron emission which remains qualitatively robust, but may vary
in overall size.  Consider, e.g., the example of laser excitation of
H$_2$O shown in figure \ref{cpc:fig:H2Oonoff}. This is computed on a
grid with $N_x=N_y=N_z=64$ points using an absorbing boundary with
$N_\mathrm{abs}=6$ points. This is a rather parsimonious choice done
for demonstration purposes. More appropriate, in case of strong
fields, would be e.g. $N_x=N_y=N_z=96$ with a rather good absorption
of $N_\mathrm{abs}=16$. This costs considerably more computing time
(factor 3.5), enhances energy absorption and emission by 15\%, but
does not change the pattern of time evolution. And the dipole signal
as well as energy absorption remain nearly unaffected.  This means in
practice, that economically chosen box sizes do well in most
cases. One may check larger boxes before drawing quantitative
conclusions on emission.  And huge boxes are required when focusing on
the ponderomotive dynamics of the emitted electrons. 
			
\subsection{Too large grid constant}

The proper choice of the grid spacing \texttt{dx}$\equiv\delta x$ is
another issue. It is never wrong to chose a too small grid spacing,
but can be unnecessarily expensive. Dangers lie at the side of too
large grid spacings. One problem is that the pseudopotential cannot be
properly represented on the grid and thus the interaction between the
electrons and the ions becomes too imprecise. Another point is that
the mesh has to be fine enough to resolve all electronic
wave functions, in dynamical cases covering also the possibly newly
created Fourier components.

The typical test is to run calculations, static as well as dynamic
ones, with different grid spacings and to see at which point the
results stabilize when going from larger spacings to smaller ones.
%
\begin{figure}[htbp]
  \centering
  \includegraphics[width=0.7\linewidth]{H2O-varygrid}
  \caption{\label{fig:H2O-varygid}Kinetic energy and s.p. energy of
    the HOMO in H$_2$O as function of grid spacing $\delta x=\delta
    y=\delta z$.}
\end{figure}
%
Fig.~\ref{fig:H2O-varygid} shows an example for the H$_2$O molecule
with the configuration and pseudo-potential parameters as explained in
\scn{sec:explions}.  The number of grid points is varied together with
grid spacing such that the size of the numerical box remains exactly
the same to avoid unwanted side-effects from changing box size.
The trend of the energies in the figures shows the typical behavior
\cite{Blu92}: Large variations of energies at the side of too large
grid spacings and nearly flat trends in the ``safe'' regime. For
ground state and not too high excitations, the 
limiting factor comes from the pseudopotential (or jellium background).
As a rule of thumb one can say that the ``safe'' regime lies in
the range  $\delta x\leq \sqrt{2\ln 2}r_\mathrm{PsP,min}$ where
$r_\mathrm{PsP,min}$ is the smallest of the pseudo-potential radii.
In rare cases, some s.p. states could have and even smaller
  radius.
Then one has to take this as limiting reference.
			
			
\subsection{Impact of initial electron wave functions}
\label{sec:initelec}

It is advisable to try, for the static iterations, a starting
  configuration which is close to the final one. Thereby, the spatial
  distribution is less important. What counts is the distribution of
  the nodes of the wave functions.  The initialization by harmonic
  oscillator wave functions, option \texttt{init\_ao=.FALSE.}, allows
  to tune rather flexibly that by the choice of initial deformations,
  see table \ref{tab:input-params-global-initwf} and discussion
  thereof.  The initialization by atomic orbitals, option
  \texttt{init\_ao=.TRUE.}, allows that (to a lesser extend) by the
  ordering instructions in the input file \texttt{for005ion.<name>}.
  We exemplify the impact of the choice of initialization for the C$_3$
  molecule.
\begin{figure}[htbp!]
  \centering
  \includegraphics[width=\linewidth]{C3-testinitv2.pdf}
  \caption{Evolution of average variance of s.p. energies (left) and
    with total energy (right) with number of iterations for the C$_3$
    molecule. Results for four different initializations are
    shown. For details see text.  }
  \label{fig:C3-testinit}
\end{figure}
Figure \ref{fig:C3-testinit} compares the convergence of variance and
energy for four different initializations. The case ``localized,12
states'' stands for initialization by atomic orbitals which are
naturally localized and which fill exactly as many states as there are
electrons. The three other case deal all with harmonic oscillators
initialization. The cases are related one line
in the input file \texttt{for005.<name>} as:
\begin{center}
\begin{tabular}{lcl}
``$\beta$=0.3,$\gamma$=30,12 states''
 &$\quad\longleftrightarrow\quad$&
\texttt{b2occ=0.3, gamocc=30.0, deocc=0.1}
\\
``$\beta$=0.3,$\gamma$=30,18 states''
 &$\quad\longleftrightarrow\quad$&
\texttt{b2occ=0.3, gamocc=30.0, deocc=0.4}
\\
``$\beta$=0.9,$\gamma$=0,12 states''
 &$\quad\longleftrightarrow\quad$&
\texttt{b2occ=0.9, gamocc=0.0, deocc=0.05}
\end{tabular}
\end{center}
Let us start with ``$\beta$=0.3,$\gamma$=30,12 states'' (green line).
Some $\beta$ deformation is needed to avoid the high degeneracy of
states in spherical basis which makes it impossible to select uniquely
a configuration with 12 states. Still, there remains a degeneracy of
two electron states. This problem is resolved by introducing also some
triaxiality $\gamma$. This case the starts to converge rapidly, but
then gets stuck in that the energy does not decrease any more and the
variance even goes back up. After same time around iteration 500, a
new phase of convergence starts which then leads to the final result.
The reason for this behavior is that the distribution of nodes in the
initial state differs from that in the final state. The 3D code is
capable to perform the change, but it takes some time for it. One way
to overcome that detour is to initialize more states than occupied in
order to supply a richer choice of node structures. This comes to
the case ``$\beta$=0.3,$\gamma$=30,18 states'' (blue). After some
initial finding phase, it turns soon to straightforward convergence
because the code performs occasional re-occupation of states filling
freshly the states in the order of the actual s.p. energies (set by
compile-time parameter \texttt{tocc} in \texttt{static.F90}).  The
other option is to use a different deformation, here a larger one
because C$_3$ is an elongated molecule. This is the case
\texttt{b2occ=0.9, gamocc=0.0, deocc=0.05} (red line) which shows the
best convergence of all because the distribution of nodes is correct
from the onset. Finally, the case ``localized,12 states'' (purple)
goes a long road to find the true minimum.  We spot at least two long
rearrangement phases from which we conclude that the initial
configuration is even worse than for the case
``$\beta$=0.3,$\gamma$=30,12 states''. The reason is that C$_3$ has
partially localized orbit and partially metallic orbits covering the
whole molecule. The fully localized initialization does not map this
structure.  The localized initialization can only be recommended for
purely covalent binding. Generally, the harmonic oscillator
initialization is to be preferred and even that has to be handled with
care.  Some error and trial is recommended when dealing with complex
geometries.

\subsection{Open-shell systems}\label{sec:open-shell}

If the ionic background has high symmetry, in worst case spherical
symmetry as in atoms, one finds highly degenerated electron
shells. This raises obnoxious problems if the number of electrons
does not fill exactly those degenerated shells. There are then a couple
of equivalent Slater states with different occupations, but exactly
the same energy.  In such situations, straightforward static iterations
do not converge, but are stuck in regular oscillations between the
equivalent configurations. In molecules, the situation can be defused
by breaking symmetry through a slight ionic displacement which is, in
fact a physical mechanism called Jahn-Teller effect \cite{Eng72}.  It is
not applicable for atoms which never loose spherical symmetry and
dimers which are always axially symmetric. A way out in such
unsolvable situations is to give the system  some electron
``temperature'' to overcome the undecided oscillations between
configurations. Each case has a certain minimum temperature above
which it achieves a stable ground state. It is of the order of the
typical level separation near the Fermi surface. A safe way to find the
optimal temperature is to start with a high temperature and to reduce it
until configuration oscillations take off.

			
\subsection{Spin assignment in spin-polarized systems}

A subtle problem, as already discussed in \scn{sec:ionbackinit} in
connection with the file \texttt{for005ion.<name>}, is the LCGO
initialization of the spin configuration (input parameter
\texttt{init\_ao=.TRUE.})  The filling strategy is explained in
\scn{sec:ionbackinit}. With a bit of care, we can arrange a wanted net
spin. Thus far the technical side. The problem remains that we often
do not know the optimal total spin for a system. There is often no
other choice than to check several possible spin settings and find out
the minimum.

	
\clearpage
			
\section{Advanced compilation options}
\label{sec:adv_compil}
In the directory \texttt{\$QDD\_ROOT/src/qdd/Makefiles} are pre-configured makefiles for different compilers and computer architectures. They have the following naming convention
\begin{verbatim}
  Makefile.<compiler>.<architecture>.mk
\end{verbatim}
\begin{description}
\item [Specifying computer architecture] The following predefined architectures are available
  \begin{itemize}
  \item Advanced Vector Extensions version 2 (AVX2). Oldest supported Intel CPUs: Haswell Q2 2013, Broadwell Q4 2014, Skylake Q3 2015
  \item  512-bit Advanced Vector Extensions (AVX-512). Oldest supported CPUs: Knights Landing 2016, Knights Mill 2017
  \item  512-bit Advanced Vector Extensions for Intel Xeon Phi (MIC-AVX512)
  \end{itemize}
  To see a list of supported CPU extensions on Linux, issue the command
\begin{verbatim}
  $ lscpu
\end{verbatim}
or
\begin{verbatim}
  $ cat /proc/cpuinfo
\end{verbatim}
or consult your cluster/supercomputer administrator.
\item [Different compilers] there are makefile templates offered for 
  \begin{itemize}
  \item Intel Fortran (\texttt{ifort})
  \item GNU Fortran (\texttt{gfortran})
  \end{itemize}
\item [Choosing FFT libraries] in all the makefiles there is support for the following external FFT libraries
  \begin{itemize}
  \item Intel MKL: (\texttt{TYPE\_FFT = MKL})
  \item GNU Fortran (\texttt{TYPE\_FFT = FFTW})
  \end{itemize}
\item [Enable OpenMP parallelism] Enable OpenMP threading in the code
  by putting \texttt{OMP = YES} in the makefile. Recompile. Before
  starting the calculation set \texttt{OMP\_NUM\_THREADS} roughly
  equal to the number of s.p. wave functions used in the calculation.
  % The difference lies in the time-evolution, the whole step for
  exponential evolution and the kinetic step in
  TV-splitting. \texttt{YES} uses parallel FFT in the dynamic
  evolution while \texttt{DYN} runs s.p. wave functions in parallel
  and uses sequential FFT for the wave functions. The option
  \texttt{DYN} becomes the more advantageous the more electrons are
  involved.

\item [Enable debugging] Two levels of debugging are available
  \begin{itemize}
  \item On the level of the compiler: (\texttt{DEBUG = YES})
  \item OpenMP debugging (\texttt{OMP\_DEBUG = YES})
  \end{itemize}
\end{description}
	
\clearpage
		
\section{Details of code structure}	
	
\subsection{Basic fields in QDD}
The basic arrays concern densities, potentials, and s.p. wave functions.  Densities and potentials distinguish spin which is denoted by $\uparrow=$ spin-up and $\downarrow=$ spin-down. Each s.p. wave function is associated with one unique spin carried in the array {\tt ispin(1:kstate)}. The fields are arranged the following way:

\begin{description}
\item{\tt rho(2*kdfull2):} electron number density, linearly stored in two blocks of length {\tt kdfull2}:\\
  {\tt rho(1:kdfull2):} total electron density $\varrho(\mathbf{r})=\varrho_\uparrow(\mathbf{r})+\varrho_\downarrow(\mathbf{r})$\\
  {\tt rho(kdfull2+1:2*kdfull2):} electron density difference $\varrho(\mathbf{r})=\varrho_\uparrow(\mathbf{r})-\varrho_\downarrow(\mathbf{r})$
			
\item{\tt aloc(2*kdfull2)} local Kohn-Sham potential,
  linearly stored in two blocks of length {\tt kdfull2}:\\
  {\tt aloc(1:kdfull2)} = local KS potential for spin-up
  $=U_\uparrow(\mathbf{r})$\\
  {\tt aloc(kdfull2+1:2*kdfull2)} = local KS potential for spin-down
  $=U_\downarrow(\mathbf{r})$
  
\item{\tt chpcoul(kdfull2)} = Coulomb potential, linearly stored
			
\item{\tt psi(kdfull2,kstate)} = set of complex s.p. wave functions 
  \\ 1. index for spatial distribution, 2. index counts the states
  \\ complex array {\tt psi} for dynamics, real array {\tt psir} for static case
  \\ each s.p. state has unique spin given in the array {\tt ispin(1:kstate)}
			
\item{\tt occup(kstate):} Occupation numbers for each of the \texttt{ksttot} number of electronic states, where \texttt{ksttot} is the total number of electronic states. 
\item{\tt ispin(kstate):} The spin for each of the \texttt{ksttot} number of electronic states.
\end{description}

	
\subsection{General subroutine calling tree}


The TDLDA packages is a rather complex collection of routines.  Thus
the tree structure of the code is sketched only at the major level of
callings and is presented in three separate diagrams: the main routine
with all initializations and two calls to the major drivers for static
and dynamic calculations in Figure \ref{fig:tree_main}, the static
driver in Figure \ref{fig:tree_static}, and the dynamic driver in
Figure \ref{fig:tree_dyn}. Finally, an oversight over the tree
structure of the RTA routines is given in figure \ref{fig:tree_rta}.


\begin{figure}[htbp]
  \centerline{\fbox{
      \begin{picture}(122,158)(5,-146)
	\put(10,8){\mbox{\tt main}}
	\put(11,7){\line(0,-1){147}}
	\put(50,1){\mbox{\large general initializations}}
	\put(20,-4){\mbox{\tt cpu\_time,  init\_parallele, init\_simann}}
	\put(11,-3){\line(1,0){8}}
	\put(20,-8){\mbox{\tt  initnamelists, checkoptions}}
	\put(11,-7){\line(1,0){8}}
	\put(20,-12){\mbox{\tt init\_baseparams, iparams}}
	\put(11,-11){\line(1,0){8}}
	\put(20,-16){\mbox{\tt iperio, changeperio}}
	\put(90,-16){\mbox{$\leftrightarrow$ PsP tables}}
	\put(11,-15){\line(1,0){8}}
	\put(20,-20){\mbox{\tt init\_grid, init\_fields}}
	\put(11,-19){\line(1,0){8}}
	\put(20,-24){\mbox{\tt init\_radmatrix}}
	\put(90,-24){\mbox{$\leftrightarrow$ tables for SIC}}
	\put(11,-23){\line(1,0){8}}
	\put(20,-28){\mbox{\tt ocoption, init\_output}}
	\put(11,-27){\line(1,0){8}}
	\put(20,-32){\mbox{\tt init\_jellium, initions}}
	\put(11,-31){\line(1,0){8}}
	\put(20,-36){\mbox{\tt pseudo\_external}}
	\put(90,-36){\mbox{$\leftrightarrow$ read explicit PsP}}
	\put(11,-35){\line(1,0){8}}
	\put(20,-40){\mbox{\tt initwf}}
	\put(11,-39){\line(1,0){8}}
	\put(20,-44){\mbox{\tt init\_homfield}}
	\put(11,-43){\line(1,0){8}}
	\put(20,-48){\mbox{\tt timer, init\_boxpara}}
	\put(11,-47){\line(1,0){8}}
	\put(50,-55){\mbox{\large static part}}
	\put(20,-60){\mbox{\tt init\_fsicr}}
	\put(11,-59){\line(1,0){8}}
	\put(20,-64){\fbox{\tt statit}}
	\put(11,-63){\line(1,0){8}}
	\put(20,-68){\mbox{\tt simann}}
	\put(11,-67){\line(1,0){8}}
	\put(20,-72){\mbox{\tt afterburn}}
	\put(11,-72){\line(1,0){8}}
	\put(50,-79){\mbox{\large dynamic part}}
	\put(20,-84){\mbox{\tt init\_absbc, init\_abs\_accum, initmeasurepoints}}
	\put(11,-83){\line(1,0){8}}
	\put(20,-88){\mbox{\tt init\_dynprotocol, evaluate}}
	\put(11,-87){\line(1,0){8}}
	\put(20,-92){\mbox{\tt init\_fsic, init\_scattel{\rm(??)}}}
	\put(11,-91){\line(1,0){8}}
	\put(20,-96){\mbox{\tt init\_dynwf}}
	\put(11,-95){\line(1,0){8}}
	\put(20,-100){\mbox{\tt restart2, addcluster}}
	\put(11,-99){\line(1,0){8}}
	\put(20,-104){\mbox{\tt calclocal, calc\_sic, calcpseudo}}
	\put(11,-103){\line(1,0){8}}
	\put(20,-108){\mbox{\tt dyn\_mfield, fermi\_init, info}}
	\put(11,-107){\line(1,0){8}}
	\put(20,-112){\mbox{\tt mergetabs, ordo\_per\_spin}}
	\put(11,-111){\line(1,0){8}}
	\put(20,-116){\mbox{\tt calc\_proj, calc\_projFine}}
	\put(11,-115){\line(1,0){8}}
	\put(20,-120){\mbox{\tt mpi\_barrier}}
	\put(11,-119){\line(1,0){8}}
	\put(20,-124){\mbox{\tt rhointxy, rhointxz, rhointyz}}
	\put(11,-123){\line(1,0){8}}
	\put(20,-128){\mbox{\tt lffirststep}}
	\put(11,-127){\line(1,0){8}}
	\put(20,-132){\mbox{\tt open\_protok\_el, analyze\_elect}}
	\put(11,-131){\line(1,0){8}}
	\put(20,-136){\fbox{\tt dyn\_propag}}
	\put(11,-135){\line(1,0){8}}
	\put(20,-141){\mbox{\tt mpi\_finalize}}
	\put(11,-140){\line(1,0){8}}
      \end{picture}
  }}
  \caption{\label{fig:tree_main}
    Schematic calling tree for the main routine in {\tt main.F90}.
    The calling trees for the two major 
    subroutines in framed boxes are explained in
    subsequent figures \ref{fig:tree_static} and  \ref{fig:tree_dyn}.
  }
\end{figure}

%\subsection{Subroutine calling tree for static calculations}
%See \fig{fig:tree_static}.

\begin{figure}[htbp]
  \centerline{\fbox{
      \begin{picture}(122,66)(5,-62)
	\put(10,0){\mbox{\tt static}}
	\put(11,-1){\line(0,-1){58}}
	\put(20,-4){\mbox{\tt calcrhor}}
	\put(11,-3){\line(1,0){8}}
	\put(20,-8){\mbox{\tt falr, solv\_possion}}
	\put(11,-7){\line(1,0){8}}
	\put(20,-12){\mbox{\tt calcpseudo, static\_mfield}}
	\put(11,-11){\line(1,0){8}}
	\put(20,-16){\mbox{\tt pricm, infor, prifld}}
	\put(11,-15){\line(1,0){8}}
	\put(20,-20){\mbox{\tt sstep}}
	\put(11,-19){\line(1,0){8}}
	\put(21,-21){\line(0,-1){22}}
	\put(30,-24){\mbox{\tt cpu\_time, system\_clock}}
	\put(21,-23){\line(1,0){8}}
	\put(30,-28){\mbox{\tt exchgr, subtr\_sicpot}}
	\put(21,-27){\line(1,0){8}}
	\put(30,-32){\mbox{\tt nonlocalr}}
	\put(21,-31){\line(1,0){8}}
	\put(30,-36){\mbox{\tt rftf, rfftback, rkin3D}}
	\put(21,-35){\line(1,0){8}}
	\put(30,-40){\mbox{\tt project, givens}}
	\put(21,-39){\line(1,0){8}}
	\put(30,-44){\mbox{\tt schmidt, reocc}}
	\put(21,-43){\line(1,0){8}}
	\put(20,-48){\mbox{\tt prifldz, pri\_pstat, printfield}}
	\put(11,-47){\line(1,0){8}}
	\put(20,-52){\mbox{\tt localizer, mtv\_field}}
	\put(11,-51){\line(1,0){8}}
	\put(20,-56){\mbox{\tt resume, rsave}}
	\put(11,-55){\line(1,0){8}}
	\put(20,-60){\mbox{\tt infor\_sic, diag\_lagr}}
	\put(11,-59){\line(1,0){8}}
      \end{picture}
  }}
  \caption{\label{fig:tree_static}
    Schematic calling tree for the static driver routine in {\tt static.F90}.
  }
\end{figure}

%\subsection{Subroutine calling tree for dynamic calculations}
%See \fig{fig:tree_dyn}.

\begin{figure}[htbp]
  \centerline{\fbox{
      \begin{picture}(122,70)(5,-66)
	\put(10,0){\mbox{\tt dyn\_propag}}
	\put(11,-1){\line(0,-1){62}}
	\put(20,-4){\mbox{\tt stimer, print\_densdiff, savings}}
	\put(11,-3){\line(1,0){8}}
	\put(20,-8){\fbox{\tt rta}}
	\put(11,-7){\line(1,0){8}}
	\put(90,-8){\mbox{$\leftrightarrow$ see section \ref{sec:RTA}}}
	\put(20,-12){\mbox{\tt init\_occ\_target, init\_psitarget}}
	\put(11,-11){\line(1,0){8}}
	\put(20,-16){\mbox{\tt tstep\_exp, CrankNicholson\_exp}}
	\put(11,-15){\line(1,0){8}}
	\put(20,-20){\mbox{\tt tstep}}
	\put(11,-19){\line(1,0){8}}
	\put(21,-21){\line(0,-1){18}}
	\put(30,-24){\mbox{\tt cpu\_time, system\_clock}}
	\put(21,-23){\line(1,0){8}}
	\put(30,-28){\mbox{\tt nonlocstep}}
	\put(21,-27){\line(1,0){8}}
	\put(30,-32){\mbox{\tt kinprop, fftf, fftback}}
	\put(21,-31){\line(1,0){8}}
	\put(30,-36){\mbox{\tt eval\_unitrot}}
	\put(21,-35){\line(1,0){8}}
	\put(30,-40){\mbox{\tt dyn\_meanfield, escmask, checkzeroforce}}
	\put(21,-39){\line(1,0){8}}
	\put(20,-44){\mbox{\tt rhointxy, rhointxz, rhointyz, testcurrent}}
	\put(11,-43){\line(1,0){8}}
	\put(20,-48){\mbox{\tt itstep, itstepv, enerkin\_ions, reset\_ions}}
	\put(11,-47){\line(1,0){8}}
	\put(20,-52){\mbox{\tt calc\_pseudo, calclocal, calc\_sic}}
	\put(11,-51){\line(1,0){8}}
	\put(20,-56){\mbox{\tt calc\_proj, calc\_projFine}}
	\put(11,-55){\line(1,0){8}}
	\put(20,-60){\mbox{\tt analyze\_elect, analyze\_ions, energ\_ions}}
	\put(11,-59){\line(1,0){8}}
	\put(20,-64){\mbox{\tt mpi\_barrier}}
	\put(11,-63){\line(1,0){8}}
      \end{picture}
  }}
  \caption{\label{fig:tree_dyn}
    Schematic calling tree for the static driver routine in {\tt dynamic.F90}.
    The routine in the framed box is explained in great detail in section \ref{cpc:sec:RTA}.}
\end{figure}


%\subsection{The RTA calling tree}\label{eq:treeRTA}

\begin{figure}[htbp]
  \centerline{\fbox{
      \begin{picture}(122,159)(5,-155)
	\put(10,0){\mbox{\tt RTA}}
	\put(11,-1){\line(0,-1){150}}
	\put(20,-4){\mbox{\tt srhomat}}
	\put(11,-3){\line(1,0){8}}
	\put(21,-5){\line(0,-1){6}}
	\put(30,-8){\mbox{\tt scalar}}
	\put(21,-7){\line(1,0){8}}
	\put(30,-12){\mbox{\tt cdiagmat}}
	\put(21,-11){\line(1,0){8}}
	\put(31,-13){\line(0,-1){2}}
	\put(40,-16){\mbox{\tt HEigensystem}}
	\put(31,-15){\line(1,0){8}}
	\put(90,-16){\mbox{$\leftrightarrow$ library routine}}
	\put(20,-20){\mbox{\tt eqstate}}
	\put(21,-21){\line(0,-1){74}}
	\put(11,-19){\line(1,0){8}}
	\put(30,-24){\mbox{\tt calcrhotot}}
	\put(21,-23){\line(1,0){8}}
	\put(30,-28){\mbox{\tt calc\_current}}
	\put(21,-27){\line(1,0){8}}
	\put(30,-32){\mbox{\tt calc\_Eref}}
	\put(21,-31){\line(1,0){8}}
	\put(30,-36){\mbox{\tt fermi1}}
	\put(31,-37){\line(0,-1){2}}
	\put(21,-35){\line(1,0){8}}
	\put(40,-40){\mbox{\tt occT1}}
	\put(31,-39){\line(1,0){8}}
	\put(30,-44){\mbox{\tt calc\_psi1}}
	\put(31,-45){\line(0,-1){38}}
	\put(21,-43){\line(1,0){8}}
	\put(40,-48){\mbox{\tt calc\_hamiltonien}}
	\put(31,-47){\line(1,0){8}}
	\put(40,-52){\mbox{\tt calc\_var}}
	\put(41,-53){\line(0,-1){2}}
	\put(50,-56){\mbox{\tt cproject}}
	\put(41,-55){\line(1,0){8}}
	\put(90,-56){\mbox{$\leftrightarrow$ \footnotesize obsolete?}}
	\put(40,-60){\mbox{\tt calcrhotot}}
	\put(31,-59){\line(1,0){8}}
	\put(40,-64){\mbox{\tt calc\_current}}
	\put(31,-63){\line(1,0){8}}
	\put(40,-68){\mbox{\tt cschmidt}}
	\put(31,-67){\line(1,0){8}}
	\put(90,-68){\mbox{$\leftrightarrow$ TDLDA package}}
	\put(40,-72){\mbox{\tt calc\_ekin}}
	\put(31,-71){\line(1,0){8}}
	\put(90,-72){\mbox{$\leftrightarrow$ TDLDA package}}
	\put(40,-76){\mbox{\tt nonlocalc}}
	\put(31,-75){\line(1,0){8}}
	\put(90,-76){\mbox{$\leftrightarrow$ TDLDA package}}
	\put(40,-80){\mbox{\tt fftf}}
	\put(31,-79){\line(1,0){8}}
	\put(90,-80){\mbox{$\leftrightarrow$ FFTW3 package}}
	\put(40,-84){\mbox{\tt fftback}}
	\put(31,-83){\line(1,0){8}}
	\put(90,-84){\mbox{$\leftrightarrow$ FFTW3 package}}
	\put(30,-88){\mbox{\tt coul\_mfield}}
	\put(21,-87){\line(1,0){8}}
	\put(90,-88){\mbox{$\leftrightarrow$ TDLDA package}}
	\put(30,-92){\mbox{\tt dyn\_mfield}}
	\put(21,-91){\line(1,0){8}}
	\put(90,-92){\mbox{$\leftrightarrow$ TDLDA package}}
	\put(30,-96){\mbox{\tt info}}
	\put(21,-95){\line(1,0){8}}
	\put(90,-96){\mbox{$\leftrightarrow$ TDLDA package}}
	\put(20,-100){\mbox{\tt occupT0}}
	\put(21,-101){\line(0,-1){2}}
	\put(11,-99){\line(1,0){8}}
	\put(30,-104){\mbox{\tt indexx}}
	\put(21,-103){\line(1,0){8}}
	\put(20,-108){\mbox{\tt calcrhoeq}}
	\put(21,-109){\line(0,-1){10}}
	\put(11,-107){\line(1,0){8}}
	\put(30,-112){\mbox{\tt cdiagmat}}
	\put(31,-113){\line(0,-1){2}}
	\put(21,-111){\line(1,0){8}}
	\put(40,-116){\mbox{\tt HEigensystem}}
	\put(31,-115){\line(1,0){8}}
	\put(90,-116){\mbox{$\leftrightarrow$ library routine}}
	\put(30,-120){\mbox{\tt  indexx}}
	\put(21,-119){\line(1,0){8}}
	\put(20,-124){\mbox{\tt calc\_Eref}}
	\put(11,-123){\line(1,0){8}}
	\put(20,-128){\mbox{\tt CorrectEnergy2}}
	\put(11,-127){\line(1,0){8}}
	\put(20,-132){\mbox{\tt OccupPerSpin}}
	\put(11,-131){\line(1,0){8}}
	\put(20,-136){\mbox{\tt temperature}}
	\put(11,-135){\line(1,0){8}}
	\put(21,-137){\line(0,-1){2}}
	\put(30,-140){\mbox{\tt lmdif1}}
	\put(21,-139){\line(1,0){8}}
	\put(20,-144){\mbox{\tt dyn\_mfield}}
	\put(11,-143){\line(1,0){8}}
	\put(90,-144){\mbox{$\leftrightarrow$ TDLDA package}}
	\put(20,-148){\mbox{\tt info}}
	\put(11,-147){\line(1,0){8}}
	\put(90,-148){\mbox{$\leftrightarrow$ TDLDA package}}
	\put(20,-152){\mbox{\tt analyze\_elect}}
	\put(11,-151){\line(1,0){8}}
	\put(90,-152){\mbox{$\leftrightarrow$ TDLDA package}}
      \end{picture}
  }}
  \caption{\label{fig:tree_rta}
    Schematic calling tree for the static driver routine in {\tt dynamic.F90}.}
\end{figure}
					
\newpage
\bibliographystyle{plain}
\bibliography{User_manual.bib}

\end{document}
